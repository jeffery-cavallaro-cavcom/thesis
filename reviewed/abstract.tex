Mechanical engineering design practices are increasingly moving towards a framework called \emph{axiomatic design}
(AD).  A key tenet of AD is to decrease the \emph{information content} of a design in order to increase the chance
of manufacturing success.  An important way to decrease information content is to fulfill multiple functional
requirements (FRs) by a single part: a process known as \emph{part consolidation}.  One possible method for
determining the minimum number of required parts is to represent a design by a graph, where the vertices are the
FRs and the edges represent the need to separate their endpoint FRs into separate parts.  The answer is then the
chromatic number of such a graph.  This research investigates the suitability of using two existing algorithms and
a new algorithm for finding the chromatic number of a graph in a part consolidation tool that can be used by
designers.  The runtime complexities and durations of the algorithms are compared empirically using the results
from a random graph analysis with binomial edge probability.  It was found that even though the algorithms are
quite different, they all execute in the same amount of time and are suitable for use in the desired design tool.

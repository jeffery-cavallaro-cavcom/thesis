\section{The Proposed Algorithm}\label{sec:algorithm}

The major advantages of the Zykov and Christofides algorithms described in the previous section are that they don't
depend on the connectedness of a graph, an example of a chromatic coloring is readily available, and the fact that
the algorithms can be coded rather easily to run on a computer.  Their major disadvantage is its their high runtime
complexity, which as was shown is inherent to the chromatic number problem.

Thus, the goals of the proposed algorithm are as follows:
\begin{enumerate}
\item It should not depend on whether the graph is connected or not.
\item An example of a chromatic coloring should be available.
\item It can be easily coded for execution on a computer.
\item It has better runtime performance than the existing algorithms.
\end{enumerate}

To accomplish these goals, the proposed algorithm loops on successively higher values of \(k\).  For each candidate
\(k\) value, a graph is assumed to be \colorable{k} and a modified version of a Zykov algorithm is executed to
either prove or disprove this assumption.  Since a candidate \(k\) value is known, certain reversible steps can be
applied to mutate \(G\) into simpler graphs with equivalent colorability and test for early termination of the
current Zykov tree.  The first \(k\) for which \(G\) (or one of its simplifications) is found to be \colorable{k}
is the chromatic number of \(G\).

One slight disadvantage of the proposed algorithm is that whereas the other algorithms readily provide examples of
actual chromatic colorings, the proposed algorithm requires a reverse traversal of its reversible steps in order to
construct such a coloring.  However, as was stated earlier, it is more important during the axiomatic design
analytical process to know the minimum number of parts as opposed to an actual FR allocation to those parts.

This algorithm was first proposed by the author and his advisor in collaboration with a team of mechanical
engineering researchers from SUNY Buffalo \cite{cavallaro}.  It accepts a graph \(G\) as input, provides \(\X(G)\)
as output, and is composed of an outer loop on values of \(k\) and a subroutine called by the outer loop to
determine if \(G\) is \colorable{k}.  The outer loop and called subroutine are summarized in the following
sections.  A complete description of the theorems that support the various steps in algorithm and the application
of the algorithm to a sample graph then follow.

\subsection{Outer Loop}\label{sec:sub:outer}

The outer loop accepts a graph \(G\) as input and returns \(\X(G)\).  It initially checks for some degenerate cases
and then loops on increasing values of \(k\).  For each value of \(k\), the called subroutine executes a modified
Zykov algorithm to determine if \(G\) is \colorable{k}.  The first such successful return identifies \(\X(G)\).

The steps of the outer loop are as follows:

\begin{enumerate}
\item \label{step:outer:null} If \(n=0\) then return \(0\), thus handling the degenerate case of a null graph.

\item \label{step:outer:one} If \(m=0\) then return \(1\), thus handling the degenerate case of an empty graph.

\item \label{step:outer:init} Initialize \(k\) to \(2\).

\item \label{step:outer:call} Call the subroutine to determine if \(G\) is \colorable{k}.  The subroutine returns a
  possibly simplified \(G\) called \(G'\) and a boolean value \(R\) that reports the result of the test.

\item \label{step:outer:result} If \(G\) is \colorable{k} (\(R=\) true) then return \(k\).

\item \label{step:outer:newg} Replace \(G\) with \(G'\).  As will be seen, doing this avoids needless reapplication of
  certain steps in the called subroutine.

\item \label{step:outer:incrk} Increment \(k\).

\item \label{step:outer:loop} Go to step~\ref{step:outer:call}.
\end{enumerate}

A flowchart of these steps is shown in \figurename~\ref{fig:outer}.

\begin{figure}[H]
  \centering
  \scalebox{0.75}{
    \begin{tikzpicture}[>=latex']
      \node (start) [draw,terminal] at (0,0) {START};
      \node (nullcheck) [draw,decision,below=of start] {\(n=0\)?};
      \node (nulldone) [draw,terminal,right=of nullcheck] {RETURN \(0\)};
      \node (emptycheck) [draw,decision,below=of nullcheck] {\(m=0\)?};
      \node (emptydone) [draw,terminal,right=of emptycheck] {RETURN \(1\)};
      \node (kinit) [draw,process,below=of emptycheck] {\(k=2\)};
      \node (kcheck) [draw,predproc,below=of kinit] {CALL \(G,k\)};
      \node (iskcolor) [draw,decision,below=of kcheck] {\(R=\)TRUE?};
      \node (kdone) [draw,terminal,right=of iskcolor] {RETURN \(k\)};
      \node (newg) [draw,process,below=of iskcolor] {\(G=G'\)};
      \node (kinc) [draw,process,below=of newg] {\(k=k+1\)};
      \node (belowinc) [coordinate,below=0.5cm of kinc] {};
      \node (leftinc) [coordinate,left=3cm of belowinc] {};
      \draw [->] (start) -- node [auto] {\(G\)} (nullcheck);
      \draw [->] (nullcheck) -- node [auto] {YES} (nulldone);
      \draw [->] (nullcheck) -- node [auto] {NO} (emptycheck);
      \draw [->] (emptycheck) -- node [auto] {YES} (emptydone);
      \draw [->] (emptycheck) -- node [auto] {NO} (kinit);
      \draw [->] (kinit) -- (kcheck);
      \draw [->] (kcheck) -- node [auto] {\(G',R\)} (iskcolor);
      \draw [->] (iskcolor) -- node [auto] {YES} (kdone);
      \draw [->] (iskcolor) -- node [auto] {NO} (newg);
      \draw [->] (newg) -- (kinc);
      \draw [->] (kinc) -- (belowinc) -- (leftinc) |- (kcheck);
    \end{tikzpicture}
  }
  \caption{Proposed algorithm outer loop.}
  \label{fig:outer}
\end{figure}

The outer loop is guaranteed to terminate because \(k\) will eventually be greater than or equal to \(n\).  Thus,
by Proposition~\ref{prop:coloring3}, the current state of \(G\) is \colorable{k}, causing the called subroutine to
return true.

\subsection{Called Subroutine}\label{sec:sub:called}

The called subroutine executes a modified version of a Zykov algorithm that determines whether a graph is
\colorable{k}.  It accepts the current state of \(G\) of order \(n\) and size \(m\) and the current value of
\(k\ge2\) as inputs.  It returns a possibly simplified version of \(G\) and a boolean value indicating whether or
not \(G\) is \colorable{k}.  Internally, various tests are applied to trim the corresponding Zykov tree or abandon
it all together based on the current value of \(k\).

The steps of the called subroutine and references to their associated theorems are as follows:

\begin{enumerate}
\item \label{step:sub:check} If \(n\le k\) then return true (Proposition~\ref{prop:coloring3}).

\item \label{step:sub:dencalc} Calculate a maximum edge threshold:
  \[a=\frac{n^2(k-1)}{2k}\]

\item \label{step:sub:density} If \(m>a\) then return false (Corollary~\ref{cor:density}).

\item \label{step:sub:smallcalc} Construct the set \(X\) of all vertices with degree less than \(k\):
  \[X=\setb{v\in V(G)}{\deg(v)<k}\]

\item \label{step:sub:small} If \(X\ne\emptyset\) then replace \(G\) with \(G-X\) and go to
  step~\ref{step:sub:check} (Corollary~\ref{cor:lowdeg}).

\item \label{step:sub:common} Calculate the common number of neighbors between each pair of vertices in \(G\),
  stopping if one vertex's neighborhood is found to be a subset of another.

\item \label{step:sub:subset} If \(G\) has vertices \(u\) and \(v\) such that \(N(u)\subseteq N(v)\) then replace
  \(G\) with \(G-u\) and go to step~\ref{step:sub:check} (Theorem~\ref{thm:subset}).

\item \label{step:sub:select} Let \(b\) be the smallest number of common neighbors between any pair of vertices in
  \(G\) as found in step~\ref{step:sub:common}:
  \[b=\min_{u,v\in V(G)}\abs{N(u)\cap N(v)}\]

\item \label{step:sub:ubcalc} Calculate an upper bound for the minimum number of common neighbors between any pair of
  vertices in \(G\):
  \[c=n-2-\frac{n-2}{k-1}\]

\item \label{step:sub:ubcheck} If \(b>c\) then return false (Corollary~\ref{cor:inter}).

\item \label{step:sub:select2} Select two non-adjacent vertices \(u,v\in V(G)\) with the smallest number of common
  neighbors as found in step~\ref{step:sub:common}.  It will be shown below that such a pair of vertices is
  guaranteed to exist in the current state of \(G\).

\item \label{step:sub:call1} Assume that \(u\) and \(v\) are assigned the same color by letting \(G'=G\cdot uv\).
  Recursively call this routine to see if \(G'\) is \colorable{k}.  If so, then return true
  (Theorem~\ref{thm:recurse}).

\item \label{step:sub:call2} Assume that \(u\) and \(v\) are assigned different colors by letting \(G'=G+uv\).
  Recursively call this subroutine to see if \(G'\) is \colorable{k}.  If so, then return true
  (Theorem~\ref{thm:recurse}).

\item \label{step:sub:fail} Since neither of the assumptions in steps \ref{step:sub:call1} nor \ref{step:sub:call2}
  hold, conclude that \(G\) is not \colorable{k} and return false.
\end{enumerate}

A flowchart of these steps is shown in \figurename~\ref{fig:called}.

\begin{figure}[H]
  \centering
  \scalebox{0.65}{
    \begin{tikzpicture}[>=latex']
      \node (start) [draw,terminal] at (0,0) {START};
      \node (donecheck) [draw,decision,below=of start] {\(n\le k\)?};
      \node (done) [draw,terminal,right=of donecheck] {RETURN \(G\),TRUE};
      \node (edgecalc) [draw,process,below=of donecheck] {\(a=\frac{n^2(k-1)}{2k}\)};
      \node (edgecheck) [draw,decision,below=of edgecalc] {\(m>a\)?};
      \node (edgefail) [draw,terminal,right=of edgecheck] {RETURN \(G\),FALSE};
      \node (nodecalc) [draw,process,below=of edgecheck] {\(X=\setb{v\in V(G)}{\deg(v)<k}\)};
      \node (nodecheck) [draw,decision,below=of nodecalc] {\(X\ne\emptyset\)?};
      \node (remnode) [draw,process,left=of nodecheck] {\(G=G-X\)};
      \node (join) [coordinate] at ($(remnode)-(2.5cm,0)$) {};
      \node (common) [draw,process,below=of nodecheck] {Calculate \(\abs{N(u)\cap N(v)}\)};
      \node (subcheck) [draw,decision,below=of common] {\(N(u)\subseteq N(v)\)?};
      \node (remsub) [draw,process,left=of subcheck] {\(G=G-u\)};
      \node (mininter) [draw,process,below=of subcheck] {\(\displaystyle b=\min_{u,v\in V(G)}\abs{N(u)\cap N(v)}\)};
      \node (intercalc) [draw,process,below=of mininter] {\(c=n-2-\frac{n-2}{k-1}\)};
      \node (intercheck) [draw,decision,below=of intercalc] {\(b>c\)?};
      \node (interfail) [draw,terminal,right=of intercheck] {RETURN \(G\),FALSE};
      \node (finduv) [draw,process,right=2.5cm of done] {\(\displaystyle \min_{uv\notin E(G)}\abs{N(u)\cap N(v)}\)};
      \node (save1) [draw,process,below=of finduv] {\(G'=G\cdot uv\)};
      \node (call1) [draw,predproc,below=of save1] {CALL \(G',k\)};
      \node (check1) [draw,decision,below=of call1] {\(R=\)TRUE?};
      \node (done1) [draw,terminal,right=of check1] {RETURN \(G\),TRUE};
      \node (save2) [draw,process,below=of check1] {\(G'=G+uv\)};
      \node (call2) [draw,predproc,below=of save2] {CALL \(G',k\)};
      \node (check2) [draw,decision,below=of call2] {\(R=\)TRUE?};
      \node (done2) [draw,terminal,right=of check2] {RETURN \(G\),TRUE};
      \node (fail) [draw,terminal,below=of check2] {RETURN \(G\),FALSE};
      \draw [->] (start) -- node [auto] {\(G,k\)} (donecheck);
      \draw [->] (donecheck) -- node [auto] {YES} (done);
      \draw [->] (donecheck) -- node [auto] {NO} (edgecalc);
      \draw [->] (edgecalc) -- (edgecheck);
      \draw [->] (edgecheck) -- node [auto] {YES} (edgefail);
      \draw [->] (edgecheck) -- node [auto] {NO} (nodecalc);
      \draw [->] (nodecalc) -- (nodecheck);
      \draw [->] (nodecheck) -- node [auto] {YES} (remnode);
      \draw [->] (remnode) -- (join) |- (donecheck);
      \draw [->] (nodecheck) -- node [auto] {NO} (common);
      \draw [->] (common) -- (subcheck);
      \draw [->] (subcheck) -- node [auto] {YES} (remsub);
      \draw (remsub) -| (join);
      \draw [->] (subcheck) -- node [auto] {NO} (mininter);
      \draw [->] (mininter) -- (intercalc);
      \draw [->] (intercalc) -- (intercheck);
      \draw [->] (intercheck) -- node [auto] {YES} (interfail);
      \draw [->] (intercheck) -- node [auto] {NO} ($(intercheck)-(0,2cm)$) -- ++(7cm,0) |- (finduv);
      \draw [->] (finduv) -- (save1);
      \draw [->] (save1) -- (call1);
      \draw [->] (call1) -- node [auto] {\(G'',R\)} (check1);
      \draw [->] (check1) -- node [auto] {YES} (done1);
      \draw [->] (check1) -- node [auto] {NO} (save2);
      \draw [->] (save2) -- (call2);
      \draw [->] (call2) -- node [auto] {\(G'',R\)} (check2);
      \draw [->] (check2) -- node [auto] {YES} (done2);
      \draw [->] (check2) -- node [auto] {NO} (fail);
    \end{tikzpicture}
  }
  \caption{Proposed algorithm called subroutine.}
  \label{fig:called}
\end{figure}

Step~\ref{step:sub:check} is the success termination condition.  Success occurs when \(G\) is simplified by
removing sufficent vertices (steps~\ref{step:sub:smallcalc}--\ref{step:sub:subset}) or when the outer loop has
sufficiently incremented \(k\) (step~\ref{step:outer:incrk}) such that \(n\le k\).

Steps~\ref{step:sub:smallcalc}--\ref{step:sub:subset} attempt to remove vertices to achieve a simpler graph that is
equivalently \colorable{k}.  Each time a vertex is removed, the Zykov branches associated with that vertex are
skipped.  Since these same steps would just be repeated for \(k+1\), the subroutine returns the current state of
the possibly simplified \(G\) to the outer loop as a starting point for the next candidate value of \(k\).

Steps~\ref{step:sub:dencalc}--\ref{step:sub:density} and \ref{step:sub:select}--\ref{step:sub:ubcheck} apply tests
that attempt to disprove that the current state of \(G\) is \colorable{k} for the current value of \(k\).  If so,
then the current Zykov tree is abandoned and the subroutine returns false.  This allows the outer loop to continue
with \(k+1\).

The remaining steps of the called subroutine, steps~\ref{step:sub:select2}--\ref{step:sub:fail}, constitute the
recursive portion of the modified Zykov algorithm.  The recursive calls are guaranteed to terminate because either
there will be sufficient vertex contractions such that \(n\le k\), resulting in a true return, or sufficient edge
additions such that the graph becomes complete and (as will be shown) is rejected by step~\ref{step:sub:density},
resulting in a false return.  Note that in the event of a false return, any modifications to the current state of
\(G\) resulting from the recursive calls are not returned to the outer loop.

\subsection{Supporting Theorems}\label{sec:sub:theorems}

This section contains the theorems that support the steps in the called subroutine.  Remember that the success
check of step~\ref{step:sub:check} is already supported by Proposition~\ref{prop:coloring3}.

\subsubsection{Maximum Edge Threshold}\label{sec:sub:sub:edges}

The maximum edge threshold test of steps~\ref{step:sub:dencalc} and \ref{step:sub:density} is supported by
Theorem~\ref{thm:density}.

\begin{theorem}[Maximum Edge Threshold]
  \label{thm:density}
  Let \(G\) be a graph of order \(n\) and size \(m\) and let \(k\in\N\).  If \(G\) is \colorable{k} then:
  \[m\le\frac{n^2(k-1)}{2k}\]
\end{theorem}

\begin{proof}
  Assume that \(G\) is \colorable{k}.  This means that \(V(G)\) can be distributed into \(k\) independent (some
  possibly empty) subsets.  Call these subsets \(A_1,\ldots A_k\) and let \(a_i=\abs*{A_i}\).  Thus, each \(v\in
  A_i\) can be adjacent to at most \(n-a_i\) other vertices in \(G\), and hence the maximum number of edges
  incident to vertices in \(A_i\) is given by: \(a_i(n-a_i)=na_i-a_i^2\).  Now, using Theorem~\ref{thm:first}, the
  maximum number of edges in \(G\) is given by:
  \[m\le\frac{1}{2}\sum_{i=1}^k(na_i-a_i^2)\]
  with the constraint:
  \[\sum_{i=1}^ka_i=n\]
  This problem can be solved using the Lagrange multiplier technique.  We start by defining:
  \begin{align*}
    F(a_1,\ldots,a_k) &= f(a_1,\ldots,a_k)-\l g(a_1,\ldots,a_k) \\
    &= \frac{1}{2}\sum_{i=1}^k(na_i-a_i^2)-\l\sum_{i=1}^ka_i \\
    &= \sum_{i=1}^k\left(\frac{1}{2}na_i-\frac{1}{2}a_i^2-\l a_i\right)
  \end{align*}
  Now, optimize by taking the gradient and setting the resulting vector equation equal to the zero vector:
  \[\vec{\nabla}F=\sum_{i=1}^k(\frac{n}{2}-a_i-\l)\hat{a_i}=\vec{0}\]
  This results in a system of \(k\) equations of the form:
  \[\frac{n}{2}-a_i-\l=0\]
  And so:
  \[a_i=\frac{n}{2}-\l\]
  Plugging this result back into the contraint:
  \[\sum_{i=1}^ka_i=\sum_{i=1}^k\left(\frac{n}{2}-\l\right)=k\left(\frac{n}{2}-\l\right)=n\]
  Solving for \(\l\) yields:
  \[\l=\frac{n}{2}-\frac{n}{k}\]
  And finally, to get \(a_i\) in terms of \(n\) and \(k\):
  \[a_i=\frac{n}{2}-\left(\frac{n}{2}-\frac{n}{k}\right)=\frac{n}{k}\]
  Therefore:
  \[m\le\frac{1}{2}\sum_{i=1}^k\left[n\left(\frac{n}{k}\right)-\left(\frac{n}{k}\right)^2\right]=
  \frac{k}{2}\left(\frac{n^2k-n^2}{k^2}\right)=\frac{n^2(k-1)}{2k}\]
\end{proof}

The called subroutine actually uses the contrapositive of this result, as stated in Corollary~\ref{cor:density}.

\begin{corollary}
  \label{cor:density}
  Let \(G\) be a graph of order \(n\) and size \(m\) and let \(k\in\N\).  If:
  \[m>\frac{n^2(k-1)}{2k}\]
  then \(G\) is not \colorable{k}.
\end{corollary}

Corollary~\ref{cor:density} is demonstrated by \figurename~\ref{fig:density}.  The shown graph \(G\) has \(n=4\),
\(m=5\), and \(\X(G)=3\).  Testing for \(k=2\):
\[a=\frac{4^2(2-1)}{2\cdot2}=4\]
But \(m=5>4=a\) and so we can conclude that \(G\) is not \colorable{2}.  However, testing for \(k=3\);
\[a=\frac{4^2(3-1)}{2\cdot3}=5.3\]
So \(m=5\ngtr5.3=a\) and thus \(G\) \emph{may} be \(3\)-colorable, since this test only provides a necessary and
not a sufficient condition.

\begin{figure}[H]
  \centering
  \begin{tikzpicture}
    \colorlet{c1}{green!25!white}
    \colorlet{c2}{blue!25!white}
    \colorlet{c3}{red!25!white}
    \begin{scope}[every node/.style={coordinate}]
      \cycleNnodes{4}{(0,0)}{0.75in}{135}{c};
    \end{scope}
    \begin{scope}[every node/.style={labeled node}]
      \node [fill=c1] (a) at (c1) {\(a\)};
      \node [fill=c2] (b) at (c2) {\(b\)};
      \node [fill=c1] (c) at (c3) {\(c\)};
      \node [fill=c3] (d) at (c4) {\(d\)};
    \end{scope}
    \draw (a) edge (b) edge (d);
    \draw (b) edge (c) edge (d);
    \draw (c) edge (d);
  \end{tikzpicture}

  \(G\)
  \caption{Corollary~\ref{cor:density} example.}
  \label{fig:density}
\end{figure}

In fact, the the test of Corollary~\ref{cor:density} will always fail for a complete graph when \(k<n\).  Since
\(k,n>0\):
\begin{align*}
  \frac{n(n-1)}{2}-\frac{n^2(k-1)}{2k} &= \frac{kn(n-1)-n^2(k-1)}{2k} \\
  &= \frac{kn^2-kn-kn^2+n^2}{2k} \\
  &= \frac{n^2-kn}{2k} \\
  &= \frac{n(n-k)}{2k} \\
  &>0\qquad(n>k)
\end{align*}

\subsubsection{Vertex Removal}\label{sec:sub:sub:vremove}

The theorems that support vertex removal make use of Lemma~\ref{lem:remone}.

\begin{lemma}
  \label{lem:remone}
  Let \(G\) be a graph and let \(v\in V(G)\).  If \(G\) is \colorable{k} then \(G-v\) is also \colorable{k}.
\end{lemma}

\begin{proof}
  Assume that \(G\) is \colorable{k}.  Let \(c:V(G)\to C\) be such a coloring, and so \(\abs{C}=k\).  Intuitively,
  removing \(v\) from \(G\) should not affect the proper coloring of the remaining vertices.  Thus, we should be
  able to construct a proper coloring for \(G-v\) based upon \(c\).  So consider the restricted coloring function
  \(c'=\restrict{c}{V(G-v)}\) and assume \(uw\in E(G-v)\).  Since \(c\) is proper:
  \[c'(u)=c(u)\ne c(w)=c'(w)\]
  Thus, \(c'\) is a proper coloring of \(G-v\) using at most \(k\) colors.

  Therefore \(G-v\) is \colorable{k}.
\end{proof}

Lemma~\ref{lem:remone} is demonstrated in \figurename~\ref{fig:remone}.  No matter which vertex is removed, the
resulting subgraph is still properly colored using at most four (in fact, three) colors.

\begin{figure}[H]
  \centering
  \scalebox{0.75}{
    \begin{tikzpicture}
      \colorlet{c1}{green!25!white}
      \colorlet{c2}{blue!25!white}
      \colorlet{c3}{red!25!white}
      \colorlet{c4}{yellow!25!white}
      \begin{scope}[every node/.style={coordinate}]
        \cycleNnodes{4}{(0,0)}{0.5in}{135}{x};
      \end{scope}
      \begin{scope} [every node/.style={labeled node}]
        \node [fill=c1] (v1) at (x1) {\(a\)};
        \node [fill=c2] (v2) at (x2) {\(b\)};
        \node [fill=c3] (v3) at (x3) {\(c\)};
        \node [fill=c4] (v4) at (x4) {\(d\)};
      \end{scope}
      \draw (v1) edge (v2) edge (v3) edge (v4);
      \draw (v2) edge (v3) edge (v4);
      \draw (v3) edge (v4);
    \end{tikzpicture}
  }

  \(G\)

  \bigskip

  \begin{minipage}{1.25in}
    \centering
    \scalebox{0.75}{
      \begin{tikzpicture}
        \colorlet{c1}{green!25!white}
        \colorlet{c2}{blue!25!white}
        \colorlet{c3}{red!25!white}
        \colorlet{c4}{yellow!25!white}
        \begin{scope}[every node/.style={coordinate}]
          \cycleNnodes{4}{(0,0)}{0.5in}{135}{x};
        \end{scope}
        \begin{scope} [every node/.style={labeled node}]
          \node [fill=c2] (v2) at (x2) {\(b\)};
          \node [fill=c3] (v3) at (x3) {\(c\)};
          \node [fill=c4] (v4) at (x4) {\(d\)};
        \end{scope}
        \draw (v2) edge (v3) edge (v4);
        \draw (v3) edge (v4);
      \end{tikzpicture}
    }

    \(G-a\)
  \end{minipage}
  \begin{minipage}{1.25in}
    \centering
    \scalebox{0.75}{
      \begin{tikzpicture}
        \colorlet{c1}{green!25!white}
        \colorlet{c2}{blue!25!white}
        \colorlet{c3}{red!25!white}
        \colorlet{c4}{yellow!25!white}
        \begin{scope}[every node/.style={coordinate}]
          \cycleNnodes{4}{(0,0)}{0.5in}{135}{x};
        \end{scope}
        \begin{scope} [every node/.style={labeled node}]
          \node [fill=c1] (v1) at (x1) {\(a\)};
          \node [fill=c3] (v3) at (x3) {\(c\)};
          \node [fill=c4] (v4) at (x4) {\(d\)};
        \end{scope}
        \draw (v1) edge (v3) edge (v4);
        \draw (v3) edge (v4);
      \end{tikzpicture}
    }

    \(G-b\)
  \end{minipage}
  \begin{minipage}{1.25in}
    \centering
    \scalebox{0.75}{
      \begin{tikzpicture}
        \colorlet{c1}{green!25!white}
        \colorlet{c2}{blue!25!white}
        \colorlet{c3}{red!25!white}
        \colorlet{c4}{yellow!25!white}
        \begin{scope}[every node/.style={coordinate}]
          \cycleNnodes{4}{(0,0)}{0.5in}{135}{x};
        \end{scope}
        \begin{scope} [every node/.style={labeled node}]
          \node [fill=c1] (v1) at (x1) {\(a\)};
          \node [fill=c2] (v2) at (x2) {\(b\)};
          \node [fill=c4] (v4) at (x4) {\(d\)};
        \end{scope}
        \draw (v1) edge (v2) edge (v4);
        \draw (v2) edge (v4);
      \end{tikzpicture}
    }

    \(G-c\)
  \end{minipage}
  \begin{minipage}{1.25in}
    \centering
    \scalebox{0.75}{
      \begin{tikzpicture}
        \colorlet{c1}{green!25!white}
        \colorlet{c2}{blue!25!white}
        \colorlet{c3}{red!25!white}
        \colorlet{c4}{yellow!25!white}
        \begin{scope}[every node/.style={coordinate}]
          \cycleNnodes{4}{(0,0)}{0.5in}{135}{x};
        \end{scope}
        \begin{scope} [every node/.style={labeled node}]
          \node [fill=c1] (v1) at (x1) {\(a\)};
          \node [fill=c2] (v2) at (x2) {\(b\)};
          \node [fill=c3] (v3) at (x3) {\(c\)};
        \end{scope}
        \draw (v1) edge (v2) edge (v3);
        \draw (v2) edge (v3);
      \end{tikzpicture}
    }

    \(G-d\)
  \end{minipage}
  \caption{Lemma~\ref{lem:remone} example.}
  \label{fig:remone}
\end{figure}

Steps~\ref{step:sub:smallcalc} and \ref{step:sub:small} remove vertices with degrees less than \(k\).  This is
supported by Theorem~\ref{thm:lowdeg}.

\begin{theorem}
  \label{thm:lowdeg}
  Let \(G\) be a graph and let \(v\in V(G)\) such that \(\deg(v)<k\) for some \(k\in\N\).  \(G\) is \colorable{k}
  if and only if \(G-v\) is \colorable{k}.
\end{theorem}

\begin{proof}
  Assume that \(G\) is \colorable{k}.  Therefore, by Lemma~\ref{lem:remone}, \(G-v\) is also \colorable{k}.

  For the converse, assume that \(G-v\) is \colorable{k}.  Let \(c:V(G-v)\to C\) be such a coloring, and so
  \(\abs{C}=k\).  By assumption, \(\deg(v)<k\), so \(v\) has at most \(k-1\) neighbors in \(G\), using at most
  \(k-1\) colors.  This means that there should be an additional color that can be assigned to \(v\) in \(G\) such
  that the coloring remains proper.  So let \(N(v)=\set{v_1,\ldots,v_r}\subseteq V(G-v)\) for some \(r<k\), and let
  \(c[N(v)]=\set{c_1,\ldots,c_s}\subset C\) for some \(s\le r<k\).  Since \(c[N(v)]\) is a proper subset of \(C\),
  select \(c_k\in C-c[N(v)]\) and define \(c':V(G)\to C\) as follows:
  \[c'(u)=\begin{cases}
  c(u), & u\ne v \\
  c_k, & u=v
  \end{cases}\]
  Now, assume that \(uw\in E(G)\) and consider the following two cases:
  \begin{description}
  \item[Case 1:] \(v\notin uw\)

    Since \(c\) is proper:
    \[c'(u)=c(u)\ne c(w)=c'(w)\]
  \item[Case 2:] \(v\in uw\)

    Assume without loss of generality (AWLOG) that \(u=v\).  This means that \(c'(v)=c_k\) and \(c'(w)=c(w)\in
    c[N(v)]\).  But \(c_k\notin c[N(v)]\) and so \(c'(v)\ne c'(w)\).
  \end{description}
  Thus, \(c'\) is a proper coloring of \(G\) using at most \(k\) colors.

  Therefore \(G\) is \colorable{k}.
\end{proof}

Theorem~\ref{thm:lowdeg} is demonstrated in \figurename~\ref{fig:lowdeg} for \(k=4\) and \(\deg(v)=3\).

\begin{figure}[H]
  \centering
  \begin{tikzpicture}
    \colorlet{c1}{green!25!white}
    \colorlet{c2}{blue!25!white}
    \colorlet{c3}{red!25!white}
    \colorlet{c4}{yellow!25!white}
    \begin{scope}[every node/.style={coordinate}]
      \cycleNnodes{4}{(0,0)}{1in}{135}{c};
    \end{scope}
    \begin{scope}[every node/.style={labeled node}]
      \node [fill=c1] (a) at (c1) {\(a\)};
      \node [fill=c2] (v) at (c2) {\(v\)};
      \node [fill=c3] (b) at (c3) {\(b\)};
      \node [fill=c4] (c) at (c4) {\(c\)};
      \node [fill=c2,below left=of a] (d) {\(d\)};
    \end{scope}
    \draw (a) edge (b) edge (c) edge (d);
    \draw [dashed,red] (v) edge (a) edge (b) edge (c);
    \draw (b) edge (c);
    \draw (c) edge (d);
  \end{tikzpicture}
  \caption{Theorem~\ref{thm:lowdeg} example.}
  \label{fig:lowdeg}
\end{figure}

The called subroutine actually removes all such vertices at once, which is supported by the inductive proof in
Corollary \ref{cor:lowdeg}.

\begin{corollary}
  \label{cor:lowdeg}
  Let \(G\) be a graph of order \(n\) and let \(X=\setb{v\in V(G)}{\deg(v)<k}\) for some \(k\in\N\).  \(G\) is
  \colorable{k} if and only if \(G-X\) is \colorable{k}.
\end{corollary}

\begin{proof}
  (by induction on \(\abs{X}\))
  \begin{description}
  \item[Base Case:] Let \(\abs{X}=0\).

    But \(G-X=G\) (trivial case).

  \item[Inductive Assumption:] Let \(\abs{X}=r\).

    Assume that \(G\) is \colorable{k} if and only if \(G-X\) is \colorable{k}.

  \item[Inductive Step:] Consider \(\abs{X}=r+1\).
    
    Since \(\abs{X}=r+1>0\), there exists \(v\in X\) such that \(\deg(v)<k\).  Let \(Y=X-\set{v}\) and note that
    \(\abs{Y}=\abs{X}-1=(r+1)-1=r\).  So, \(G\) is \colorable{k} if and only if \(G-v\) is \colorable{k} (Theorem
    \ref{thm:lowdeg}) if and only if \((G-v)-Y\) is \colorable{k} (inductive assumption).
  \end{description}

  Therefore, by the principle of induction, \(G\) is \colorable{k} if and only if \(G-X\) is \colorable{k}.
\end{proof}

Returning to the example in \figurename~\ref{fig:lowdeg}, note that \(X=\set{v,b}\) is the set of all vertices with
degree less than \(4\) and so both could be removed at once in accordance with Corollary \ref{cor:lowdeg}.
Furthermore, after these vertices are removed, the remaining vertices \(a\), \(b\), and \(c\) will all have degree
\(2\).  Since \(2<4\), the remaining vertices are subsequently removed, leaving \(n=0<4\), indicating that the
graph is indeed \colorable{4}.  This iterative collapsing of a graph is an ideal situation.

Step~\ref{step:sub:subset} removes vertices whose neighborhoods are subsets of other vertices.  This is supported
by Theorem~\ref{thm:subset}.

\begin{theorem}
  \label{thm:subset}
  Let \(G\) be a graph and let \(u,v\in V(G)\) such that \(N(u)\subseteq N(v)\).  \(G\) is \colorable{k} if and
  only if \(G-u\) is \colorable{k}.
\end{theorem}

\begin{proof}
  Assume that \(G\) is \colorable{k}.  Therefore, by Lemma \ref{lem:remone}, \(G-u\) is also \colorable{k}.

  For the converse, assume that \(G-u\) is \colorable{k}.  Let \(c:V(G-u)\to C\) be such a coloring, and so
  \(\abs{C}=k\).  By definition, \(u\notin N(u)\), and since, by assumption, \(N(u)\subseteq N(v)\), it is also the
  case that \(u\notin N(v)\).  Hence, \(u\) is not adjacent to \(v\) in \(G\).  But everything that is adjacent to
  \(u\) in \(G\) is also adjacent to \(v\) in both \(G-u\) and \(G\).  Since everything adjacent to \(v\) has a
  different color than \(v\), we should be able to assign \(u\) the same color as \(v\) in \(G\) in order to get a
  proper coloring of \(G\).

  Let \(S=N(u)\cap N(v)=\set{w_1,\ldots w_r}\subset V(G-u)\) for some \(r=\deg(u)\le \deg(v)\).  For all \(w\in
  S\), \(w\in N(v)\) meaning \(vw\in E(G-u)\).  And, because \(c\) is proper, it must be the case that \(c(w)\ne
  c(v)\).  So define \(c':V(G)\to C\) as follows:
  \[c'(w)=\begin{cases}
  c(w), & w\ne u \\
  c(v), & w=u
  \end{cases}\]
  Now, assume that \(wz\in E(G)\) and consider the following two cases:
  \begin{description}
  \item[Case 1:] \(u\notin wz\)

    Since \(c\) is proper:
    \[c'(w)=c(w)\ne c(z)=c'(z)\]
  \item[Case 2:] \(u\in wz\)

    Assume without loss of generality (AWLOG) that \(u=w\).  Since \(vz\in E(G)\) and \(c\) is proper:
    \[c'(u)=c(v)\ne c(z)=c'(z)\]
  \end{description}
  Thus, \(c'\) is a proper coloring of \(G\) using at most \(k\) colors.

  Therefore \(G\) is \colorable{k}.
\end{proof}

Theorem~\ref{thm:subset} is demonstrated in \figurename~\ref{fig:subset}.  Since \(N(u)\subseteq N(v)\), \(G\) and
\(G-u\) are equivalently colorable.  Futhermore, once \(u\) is removed, the degrees of vertices \(a\) and \(c\)
will have degree \(2\).  So if \(k=3\), those two vertices are subsequently removed by step \ref{step:sub:small}.
The remaining graph is of order \(2\), which then passes the success check of step \ref{step:sub:check} because
\(2<3\), so indeed the graph is \colorable{3}.  Once again, this iterative removal of vertices is very powerful.

\begin{figure}[H]
  \centering
  \begin{tikzpicture}[every node/.style={labeled node}]
    \colorlet{c1}{green!25!white}
    \colorlet{c2}{blue!25!white}
    \colorlet{c3}{red!25!white}
    \node [fill=c1] (b) at (0,0) {\(b\)};
    \node [fill=c2] (a) [above=of b] {\(a\)};
    \node [fill=c2] (c) [below=of b] {\(c\)};
    \node [fill=c3] (u) [left=of b] {\(u\)};
    \node [fill=c3] (v) [right=of b] {\(v\)};
    \draw [dashed,red] (u) edge (a) edge (b);
    \draw (v) edge (a) edge (b) edge (c);
    \draw (a) edge (b);
    \draw (c) edge (b);
  \end{tikzpicture}
  \caption{Theorem \ref{thm:subset} example.}
  \label{fig:subset}
\end{figure}

\subsubsection{Minimum Common Neighbor Upper Bound}\label{sec:sub:sub:common}

Steps~\ref{step:sub:select}--\ref{step:sub:ubcheck} establish an upper bound for the minimum common neighbor count
between any two vertices in a graph that is assumed to be \colorable{k}.  This limit is dependent on the following
facts that are guaranteed by previous steps:

\begin{enumerate}
\item \(2\le k<n\)
\item There are no \(u,v\in V(G)\) such that \(N(u)\subseteq N(v)\)
\end{enumerate}

The supporting theorem uses these facts along with Lemma~\ref{lem:neighbor} in its proof.

\begin{lemma}
  \label{lem:neighbor}
  Let \(G\) be a graph and let \(S\) be a non-empty independent subset of \(V(G)\).  If there exists a vertex
  \(v\in S\) such that \(v\) is adjacent to all vertices in \(V(G)-S\) (i.e., \(N(v)=V(G)-S\)) then for all
  vertices \(u\in S\) it is the case that \(N(u)\subseteq N(v)\).
\end{lemma}

\begin{proof}
  Assume that such a \(v\) exists and then assume that \(u\in S\).  If \(u=v\) then (trivially) \(N(v)=N(v)\), so
  assume \(u\ne v\).  Furthermore, since \(u,v\in S\) and \(S\) is independent (by assumption), it must be the case
  that \(u\) and \(v\) are not neighbors.

  \begin{description}
  \item[Case 1:] \(N(u)=\emptyset\).
      
    Therefore, by definition, \(N(u)=\emptyset\subseteq N(v)\).

  \item[Case 2:] \(N(u)\ne\emptyset\).

    Assume that \(w\in N(u)\).  This means that \(w\) is adjacent to \(u\) and hence \(w\notin S\), since \(S\) is
    an independent set.  So \(w\in V(G)-S\) and thus, by assumption, \(v\) is adjacent to \(w\) and we can conclude
    that \(w\in N(v)\).  Therefore \(N(u)\subseteq N(v)\).
  \end{description}

  Therefore, for all \(u\in S\), \(N(u)\subseteq N(v)\).
\end{proof}

Lemma~\ref{lem:neighbor} is demonstrated in \figurename~\ref{fig:neighbor}.  Note that since \(v\in S\) is adjacent
to every vertex in \(V(G)-S\), vertex \(u\in S\) can't help but be adjacent to some subset of \(N(v)\).

\begin{figure}[H]
  \centering
  \begin{tikzpicture}
    \draw (0in,0) ellipse (0.5in and 1in);
    \draw (3in,0) ellipse (1.5in and 1in);
    \begin{scope}[every node/.style={unlabeled node}]
      \node (v) at (0in,0.25in) {};
      \node (u) at (0in,-0.25in) {};
      \node (w1) at (2in,0) {};
      \node (w2) at (2.5in,0) {};
      \node (w3) at (3in,0) {};
      \node (w4) at (3.5in,0) {};
      \node (w5) at (4in,0) {};
    \end{scope}
    \node [left=1ex of v] {\(v\)};
    \node [left=1ex of u] {\(u\)};
    \node at (0,-1.25in) {\(S\)};
    \node at (3in,-1.25in) {\(V(G)-S\)};
    \draw (v) [bend left] edge (w1);
    \draw (v) [bend left] edge (w2);
    \draw (v) [bend left] edge (w3);
    \draw (v) [bend left] edge (w4);
    \draw (v) [bend left] edge (w5);
    \draw (u) [bend right] edge (w2);
    \draw (u) [bend right] edge (w4);
    \draw (u) [bend right] edge (w5);
  \end{tikzpicture}
  \caption{Lemma \ref{lem:neighbor} example.}
  \label{fig:neighbor}
\end{figure}

Theorem~\ref{thm:inter} establishes the desired upper bound.

\begin{theorem}
  \label{thm:inter}
  Let \(G\) be a graph of order \(n\) and size \(m\) such that there are no \(u,v\in V(G)\) where \(N(u)\subseteq
  N(v)\), and let \(k\in\N\) such that \(2\le k<n\).  If \(G\) is \colorable{k} then there exists two vertices
  \(w,z\in V(G)\) such that:
  \[\abs{N(w)\cap N(z)}\le n-2-\frac{n-2}{k-1}\]
\end{theorem}

\begin{proof}
  Assume that \(G\) is \colorable{k}.  This means that \(V(G)\) can be distributed into \(k\) independent (some
  possibly empty) subsets \(A_1,\ldots,A_k\) such that \(a_i=\abs*{A_i}\) and \(a_1\ge a_2\ge\cdots\ge a_k\).
  Since \(n>k\), by the pigeonhole principle, it must be the case that \(a_1\ge2\).  Assume that \(v\in A_1\).

  First, assume by way of contradiction (ABC) that \(v\) is adjacent to all other vertices in \(V(G)-A_1\).  Since
  \(a_1\ge2\), there exists \(u\in A_1\) such that \(u\ne v\) and \(u\) is not adjacent to \(v\).  Thus, by Lemma
  \ref{lem:neighbor}, \(N(u)\subseteq N(v)\), contradicting the assumption.  Note that this contradiction also
  eliminates the degenerate case where \(A_1=V(G)\); however, this case does not occur here because the graph would
  be an empty graph and would have been eliminated by previous steps.  Therefore, there exists some \(v'\in
  V(G)-A_1\) such that \(v\) is not adjacent to \(v'\).  Assume that \(v'\in A_i\) for some \(i\) such that
  \(1<i\le k\):

  \begin{description}
  \item [Case 1:] \(a_i=1\)

    By the pigeonhole principle:
    \[a_1\ge\ceil*{\frac{n-1}{k-1}}\ge\frac{n-1}{k-1}\]
    Now, assume by way of contradiction (ABC) that \(v'\) is adjacent to all vertices in \(V(G)-A_1-A_i\) and
    assume \(u\in N(v)\).  Then it must be the case that \(u\in V(G)-A_1-A_i\), and so \(u\) is adjacent to \(v'\),
    and thus \(u\in N(v')\).  Therefore \(N(v)\subseteq N(v')\), which contradicts the assumption.  This situation
    is demonstrated by \figurename~\ref{fig:aione}.

    \begin{figure}[H]
      \centering
      \begin{tikzpicture}
        \draw (0,0) ellipse (0.5in and 1in);
        \draw (2in,0) ellipse (0.5in and 1in);
        \draw (4in,0) ellipse (0.5in and 1in);
        \begin{scope}[every node/.style={unlabeled node}]
          \node (v) at (0,0) {};
          \node (u1) at (2in,0.5in) {};
          \node (u2) at (2in,0.25in) {};
          \node (u3) at (2in,0in) {};
          \node (u4) at (2in,-0.25in) {};
          \node (u5) at (2in,-0.5in) {};
          \node (vp) at (4in,0) {};
        \end{scope}
        \node [left=1ex of v] {\(v\)};
        \node [right=1ex of vp] {\(v'\)};
        \node at (0,-1.25in) {\(A_1\)};
        \node at (2in,-1.25in) {\(V(G)-A_1-A_i\)};
        \node at (4in,-1.25in) {\(A_i\)};
        \draw (v) edge (u1) edge (u3) edge (u4);
        \draw (vp) edge (u1) edge (u2) edge (u3) edge (u4) edge (u5);
      \end{tikzpicture}
      \caption{Case \(a_i=1\) contradiction.}
      \label{fig:aione}
    \end{figure}

    So there must exist some \(u\in V(G)-A_1-A_i\) such that \(u\) is not adjacent to \(v'\).  This results in the
    upper bound:
    \[\abs{N(v)\cap N(v')}\le n-\abs{\set{u,v'}}-a_1\le n-2-\frac{n-1}{k-1}\]
    Note that since \(v\in A_1\), it is already counted in \(a_1\).  Comparing this bound to the desired bound:
    \[\left(n-2-\frac{n-2}{k-1}\right)-\left(n-2-\frac{n-1}{k-1}\right)=\frac{(n-1)-(n-2)}{k-1}=\frac{1}{k-1}>0\]
    for \(k\ge2\).  Thus the new bound is tighter and so:
    \[\abs{N(v)\cap N(v')}\le n-2-\frac{n-1}{k-1}\le n-2-\frac{n-2}{k-1}\]
    
  \item [Case 2:] \(a_i=2\)

    By the pigeonhole principle:
    \[a_1\ge\ceil*{\frac{n-2}{k-1}}\ge\frac{n-2}{k-1}\]
    This results in the upper bound:
    \[\abs{N(v)\cap N(v')}\le n-a_i-a_1\le n-2-\frac{n-2}{k-1}\]
    
  \item [Case 3:] \(a_i\ge3\)

    By the pigeonhole principle:
    \[a_1\ge\ceil*{\frac{n-3}{k-1}}\ge\frac{n-3}{k-1}\]
    This results in the upper bound:
    \[\abs{N(v)\cap N(v')}\le n-a_i-a_1\le n-3-\frac{n-3}{k-1}\]
    Comparing this to the desired bound:
    \[\left(n-2-\frac{n-2}{k-1}\right)-\left(n-3-\frac{n-3}{k-1}\right)=1-\frac{(n-3)-(n-2)}{k-1}=1-\frac{1}{k-1}>0\]
    for \(k\ge2\).  Thus the new bound is tighter and so:
    \[\abs{N(v)\cap N(v')}\le n-3-\frac{n-3}{k-1}\le n-2-\frac{n-2}{k-1}\]
  \end{description}

  Therefore, there exists \(w,v\in V(G)\) such that:
  \[\abs{N(w)\cap N(z)}\le n-2-\frac{n-2}{k-1}\]
\end{proof}

The called subroutine actually uses the contrapositive of this result, as stated in Corollary~\ref{cor:inter}.

\begin{corollary}
  \label{cor:inter}
  Let \(G\) be a graph of order \(n\) and size \(m\) such that there are no \(u,v\in V(G)\) where
  \(N(u)\subseteq N(v)\), and let \(k\in\N\) such that \(2\le k<n\).  If for all \(w,z\in V(G)\) it is the case
  that:
  \[\abs{N(w)\cap N(z)}>n-2-\frac{n-2}{k-1}\]
  then \(G\) is not \colorable{k}.
\end{corollary}

Corollary~\ref{cor:inter} is demonstrated in \figurename~\ref{fig:inter}.  The shown graph has \(n=5\), is
\chromatic{3}, and has:
\[\min_{u,v\in V(G)}\abs{N(u)\cap N(v)}=1\]
Testing for \(k=2\):
\[5-2-\frac{5-2}{2-1}=0\]
But \(1>0\) and so we can conclude that \(G\) is not \colorable{2}.  However, testing for \(k=3\):
\[5-2-\frac{5-2}{3-1}=\frac{3}{2}\]
So \(1\ngtr\frac{3}{2}\) and thus \(G\) \emph{may} be \colorable{3}, since this test only provides a necessary and
not a sufficient condition.

\begin{figure}[H]
  \centering
  \begin{tikzpicture}
    \colorlet{c1}{green!25!white}
    \colorlet{c2}{blue!25!white}
    \colorlet{c3}{red!25!white}
    \begin{scope}[every node/.style={labeled node}]
      \node [fill=c1] (a) at (0,0) {\(a\)};
      \node [fill=c2,above left=of a] (b) {\(b\)};
      \node [fill=c3,above right=of a] (c) {\(c\)};
      \node [fill=c2,below left=of a] (d) {\(d\)};
      \node [fill=c3,below right=of a] (e) {\(e\)};
    \end{scope}
    \draw (a) -- (b) -- (c) -- (a) -- (d) -- (e) -- (a);
  \end{tikzpicture}

  \(G\)
  \caption{Corollary \ref{cor:inter} example.}
  \label{fig:inter}
\end{figure}

\subsubsection{Recursive Steps}

If nothing more can be done in the preceding steps then steps~\ref{step:sub:select2}--\ref{step:sub:fail} revert to
branching.  Step~\ref{step:sub:select2} selects two non-adjacent vertices with the smallest number of common
neighbors.  Such a pair must exist.  Otherwise, the current state of \(G\) is complete, which would have been
eliminated by step \ref{step:sub:density}.  The first recursive call (step~\ref{step:sub:call1}) assumes that the
two selected vertices have the same color, so they are contracted.  The second recursive call (step
\ref{step:sub:call2}) assumes that the two selected vertices have different colors, so they are joined by an added
edge.  Each call starts a new branch of the Zykov tree corresponding to the current value of \(k\).  If either call
returns true then it can be concluded that the input graph was indeed \colorable{k}.  Otherwise, it can be
concluded that the input graph is not \colorable{k} and the called subroutine returns the state of \(G\) prior to
the recursive calls to the outer loop.

These steps are supported by Theorem~\ref{thm:recurse}.

\begin{theorem}
  \label{thm:recurse}
  Let \(G\) be a graph of order \(n\ge2\) and let \(u,v\in G\) such that \(u\) and \(v\) are not adjacent.  \(G\)
  is \colorable{k} if and only if \(G\cdot uv\) or \(G+uv\) is \colorable{k}.
\end{theorem}

\begin{proof}
  Assume that \(G\) is \colorable{k}.  Let \(c:V(G)\to C\) be such a coloring, and so \(\abs{C}=k\).  There are two
  possibilities, corresponding to the two recursive choices:
  \begin{description}
  \item [Case 1:] \(u\) and \(v\) have the same color: \(c(u)=c(v)\).

    Let \(c_{uv}=c(u)=c(v)\in C\).  For all \(w\in N(u)\cup N(v)\), since \(w\) is adjacent to \(u\) and \(v\) and
    because \(c\) is proper, it must be the case that \(w\) is a different color than the color of \(u\) and \(v\):
    \(c(w)\ne c_{uv}\).  Let \(v'\) be the contracted vertex, so that \(N(v')=N(u)\cup N(v)\) and assign color
    \(c_{uv}\) to \(v'\).  Define \(c':V(G\cdot uv)\to C\) as follows:
    \[c'(w)=\begin{cases}
    c(w), & w\ne v' \\
    c_{uv}, & w=v'
    \end{cases}\]
    Now, assume that \(wz\in E(G\cdot uv)\) and consider the following two cases:
    \begin{description}
    \item[Case a:] \(v'\notin wz\)

      Since \(c\) is proper:
      \[c'(w)=c(w)\ne c(z)=c'(z)\]

    \item[Case b:] \(v'\in wz\)

      Assume without loss of generality that \(v'=w\).  Since \(z\in N(v')\):
      \[c'(v')=c_{uv}\ne c(z)=c'(z)\]
    \end{description}

    Thus, \(c'\) is a proper coloring of \(G\cdot uv\) using at most \(k\) colors.
      
    Therefore \(G\cdot uv\) is \colorable{k}.

  \item [Case 2:] \(u\) and \(v\) have the different colors: \(c(u)\ne c(v)\).

    By adding edge \(uv\), \(u\) and \(v\) become adjacent and thus must have different colors.  Thus, \(u\)
    and \(v\) can retain their same colors.  So define \(c':V(G+uv)\to C\) as follows:
    \[c'(w)=c(w)\]
    Now, assume \(wz\in E(G)\).  Since \(c\) is proper:
    \[c'(w)=c(w)\ne c(z)=c'(z)\]
    Thus, \(c'\) is a proper coloring of \(G+uv\) using at most \(k\) colors.

    Therefore \(G+uv\) is \colorable{k}.
  \end{description}

  Therefore \(G\cdot uv\) or \(G+uv\) is \colorable{k}.
    
  For the converse, assume \(G\cdot uv\) or \(G+uv\) is \colorable{k}.  Again, there are two cases:
  \begin{description}
  \item [Case 1:] \(G\cdot uv\) is \colorable{k}.

    Let \(c:V(G\cdot uv)\to C\) be such a coloring, and so \(\abs{C}=k\).  Let \(v'\) be the contracted vertex, and
    so \(N(v')=N(u)\cup N(v)\).  Let \(c(v')=c_{uv}\in C\).  For all \(w\in N(v')\) in \(G\cdot uv\), since \(w\) and
    \(v'\) are adjacent, they must have different colors: \(c(w)\ne c_{uv}\).

    Consider \(u\) and \(v\) in \(G\).  Since they are not adjacent, they can be assigned the same color.  So,
    defined \(c':V(G)\to C\) as follows:
    \[c'(w)=\begin{cases}
    c(w), & w\ne u,v \\
    c_{uv}, & w=u \\
    c_{uv}, & w=v
    \end{cases}\]
    Now, assume \(wz\in E(G)\) and consider the following cases:
    \begin{description}
    \item[Case a:] \(uv=wz\)

      This is not possible since, by assumption, \(u\) is not adjacent to \(v\).

    \item[Case b:] \(u\in wz\) and \(v\notin wz\)

      Assume without loss of generality (AWLOG) that \(u=w\).  Since \(z\in N(u)\) in \(G\), it must be the case that
      \(z\in N(v')\) in \(G\cdot uv\).  And so:
      \[c'(u)=c_{uv}\ne c(w)=c'(w)\]

    \item[Case c:] \(u\notin wz\) and \(v\in wz\)

      Assume without loss of generality (AWLOG) that \(v=w\).  Since \(z\in N(v)\) in \(G\), it must be the case that
      \(z\in N(v')\) in \(G\cdot uv\).  And so:
      \[c'(v)=c_{uv}\ne c(w)=c'(w)\]

    \item[Case d:] \(u,v\notin wz\)

      Since \(c\) is proper:
      \[c'(w)=c(w)\ne c(z)=c'(z)\]
    \end{description}

    Thus, \(c'\) is a proper coloring of \(G\) using at most \(k\) colors.

  \item [Case 2:] \(G+uv\) is \colorable{k}.

    Let \(c:V(G+uv)\to C\) be such a coloring, and so \(\abs{C}=k\).  Since \(u\) and \(v\) are adjacent in
    \(G+uv\), they must have different colors.  Once \(uv\) is removed in \(G\), \(u\) and \(v\) are no longer
    adjacent and so there are no requirements on their colors.  Thus, they can retain their original colors from
    \(G+uv\).  So define \(c':V(G)\to C\) as follows:
    \[c'(w)=c(w)\]
    Now assume \(wz\in E(G)\).  Since \(c\) is proper:
    \[c'(w)=c(w)\ne c(z)=c'(z)\]
    Thus, \(c'\) is a proper coloring of \(G\) using at most \(k\) colors.
  \end{description}

  Therefore \(G\) is \colorable{k}.
\end{proof}

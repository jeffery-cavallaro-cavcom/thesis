\section{The Proposed Algorithm}\label{sec:algorithm}

It was shown in \sectionname~\ref{sec:chromatic} that the Christofides algorithm with the Wang improvements
outperforms the Zykov algorithm.  But can some additional bounding conditions be added to the Zykov algorithm to
close the performance gap?  This section proposes a new Zykov-like algorithm that attempts to do just that.  An
early version of this new algorithm was first introduced by the author and his advisor in collaboration with a team
of mechanical engineering researchers from SUNY Buffalo \cite{cavallaro}.

The major advantages of the Christofides and the Zykov algorithms are that they do not depend on the connectedness
of a graph, an example of a chromatic coloring is readily available, and the fact that the algorithms can be coded
rather easily to run on a computer.  Their major disadvantage is their high runtime complexity, which is inherent
to the chromatic number problem.

Thus, the goals of the proposed algorithm are as follows:
\begin{enumerate}
\item It should not depend on whether the graph is connected or not.
\item An example of a chromatic coloring should be readily available.
\item It can be easily coded for execution on a computer.
\item It has better runtime performance than the well-known algorithms over the target range of less than
  \(20\) nodes with less than \(50\%\) edge density.
\end{enumerate}

To accomplish these goals, the proposed algorithm loops on successively higher values of \(k\).  For each candidate
\(k\) value, a graph \(G\) is assumed to be \colorable{k} and a modified version of a Zykov algorithm is executed
on \(G\) to either prove or disprove this assumption.  Since a candidate \(k\) value is known, certain reversible
steps can be applied to mutate \(G\) into simpler graphs with equivalent colorability and test for early
termination of the Zykov tree.  The first \(k\) for which \(G\) (or one of its simplifications) is found to be
\colorable{k} is the chromatic number of \(G\).

One slight disadvantage of the proposed algorithm is that whereas the other algorithms readily provide examples of
actual chromatic colorings, the proposed algorithm requires an additional step to construct a chromatic coloring: a
greedy algorithm with a particular sorting of the vertices.  The coloring step is discussed in detail in
\sectionname~\ref{sec:sub:coloring}.  Albeit additional work, this extra coloring step is P-time so it does not
have a significant impact on the performance of the proposed algorithm.

The proposed algorithm accepts a graph \(G\) as input and provides \(\X(G)\) and a chromatic coloring as output and
is composed of the following components:
\begin{enumerate}
\item A main routine that loops on increasing values of \(k\).
\item The Bron Kerbosch algorithm used by the main routine to determine a lower bound \(k_{min}\) for the chromatic
  number of \(G\) (\sectionname~\ref{sec:sub:sub:bron}).
\item The last-first with color interchange greedy coloring algorithm used by the main routine to determine an
  upper bound \(k_{max}\) for the chromatic number of \(G\) and a corresponding \coloring{k_{max}} of \(G\)
  (\sectionname~\ref{sec:sub:upper}).
\item A recursive subroutine that runs a modified Zykov algorithm with additional pruning to determine if
  \(G\) is \colorable{k}.
\item The Edwards Elphick algorithm used by the recursive subroutine to determine a lower bound for the chromatic
  number of a graph (\sectionname~\ref{sec:sub:sub:edwards});
\item A coloring routine called by the main routine to construct a chromatic coloring of \(G\) based on the results
  of the algorithm.
\end{enumerate}

The main routine, recursive subroutine, and coloring routine are summarized in the following sections.  A complete
description of the theorems that support the various steps in the algorithm and the application of the algorithm to
a sample graph then follow.

\subsection{The Main Routine}\label{sec:sub:main}

The main routine accepts a graph \(G\) as input and returns \(\X(G)\) and a chromatic coloring for \(G\).  It
initially computes a chromatic number lower bound using Bron Kerbosch and a chromatic number upper bound using
greedy last-first with color interchange.  The latter also provides a default coloring.  If the lower and upper
bounds match then the default coloring is accepted.  If not, then the main routine loops on increasing values of
\(k\), starting with the lower bound and going no further than the upper bound.  For each value of \(k\), the
recursive subroutine is called to execute a modified Zykov algorithm in order to determine if \(G\) is
\colorable{k}.  If \(k\) reaches the upper bound then the default coloring is accepted as chromatic.

Instead of walking a single Zykov tree for the whole graph \(G\), the Wang technique
(see~\sectionname~\ref{sec:sub:wang}) is used to mutate \(G\) into smaller graphs \(G_i\) by selecting a vertex
that occurs in the least number of MISs in \(G\).  Each of these MISs implies a set of vertex contractions and edge
additions that are applied to \(G\) to construct the corresponding \(G_i\).  Thus, the modified Zykov algorithm is
applied in sequence to these separate \(G_i\) for each \(k\) value.  The first successful return identifies
\(\X(G)\) and then the coloring routine is called to construct the final coloring based on the information from the
successful tree.

The steps of the main routine are as follows:

\begin{enumerate}
\item\label{step:outer:bron} Use the Bron Kerbosch algorithm to compute a chromatic lower bound \(k_{min}\)
  for the input \(G\).

\item\label{step:outer:greedy} Use the greedy last-first with color interchange algorithm to compute a chromatic
  number upper bound \(k_{max}\) and a default \coloring{k_{max}} for \(G\).

\item\label{step:outer:initk} Initialize \(k\) to \(k_{min}\).

\item\label{step:outer:match} If \(k_{min}=k_{max}\) then accept the default coloring found in
  step~\ref{step:outer:greedy} and go to step~\ref{step:outer:done}.

\item\label{step:outer:trees} Use the Bron algorithm on \(\bar{G}\) to find all the MISs in \(G\).  Select a vertex
  that occurs in the smallest number of MISs and then use those MISs to construct \(n\) graphs that are the roots
  of \(n\) Zykov trees.  Each tree is constructed from \(G\) by contracting the vertices in the MIS and adding
  edges between the contracted vertex and all vertices not in the MIS.  Each tree is also associated with an
  initially empty list of removed vertices \(S\).  The trees are sorted by decreasing MIS length since graphs
  resulting from larger MISs have fewer choices for branching and thus can be walked more expediently.

\item\label{step:outer:check} If \(k=k_{max}\) then accept the default coloring found in
  step~\ref{step:outer:greedy} and go to step~\ref{step:outer:done}.

\item\label{step:outer:call} Call the recursive subroutine on each tree to determine if its \(G\) is \colorable{k}.
  The recursive subroutine accepts the current tree's graph \(G\) and removed vertex list \(S\) as input and
  returns a Boolean result \(R\).  The called routine may simplify \(G\) and may append removed vertices to
  \(S\).  If a tree results in a solution (\(R=\)true) then go to step~\ref{step:outer:color}.

\item\label{step:outer:incrk} Increment \(k\) and go to step~\ref{step:outer:check}.

\item\label{step:outer:color} Call the coloring routine to construct the final coloring based on the successful
  tree's final state of \(G\) and \(S\).

\item\label{step:outer:done} Return the current \(k\) as the chromatic number and the found chromatic coloring.
\end{enumerate}

A flowchart of these steps is shown in \figurename~\ref{fig:outer}.

\begin{figure}[H]
  \centering
  \scalebox{0.75}{
    \begin{tikzpicture}[>=latex']
      \node (start) [draw,terminal] at (0,0) {START};
      \node (bron) [draw,predproc,below=of start]
            {\(\begin{array}{c} \text{BRON}\ G \\ \text{GREEDY}\ G \end{array}\)};
      \node (kinit) [draw,process,below=of bron] {\(k=k_{min}\)};
      \node (kcheck) [draw,decision,below=of kinit] {\(k_{min}=k_{max}\)?};
      \node (trees) [draw,predproc,below=of kcheck] {TREES \(G\)};
      \node (kcheck2) [draw,decision,below=of trees] {\(k=k_{max}\)?};
      \node (cdef) [draw,process,left=2cm of kcheck2] {\(c=c_{def}\)};
      \node (done) [draw,terminal,below=of cdef] {RETURN \(k,c\)};
      \node (iinit) [draw,process,below=of kcheck2] {\(i=1\)};
      \node (icheck) [draw,decision,below=of iinit] {\(i\le n\)?};
      \node (kinc) [draw,process,right=2cm of icheck] {\(k=k+1\)};
      \node (select) [draw,process,below=of icheck] {\(\begin{array}{c} G=G[i] \\ S=S[i] \end{array}\)};
      \node (zykov) [draw,predproc,below=of select] {ZYKOV \(G,S,k\)};
      \node (check) [draw,decision,below=of zykov] {\(R\)?};
      \node (iinc) [draw,process,below=of check] {\(i=i+1\)};
      \node (color) [draw,predproc,right=2cm of check] {COLOR \(G,S\)};
      \node (done2) [draw,terminal,below=of color] {RETURN \(k,c\)};
      \draw [->] (start) -- node [auto] {\(G\)} (bron);
      \draw [->] (bron) -- node [auto] {\(k_{min},k_{max},c_{def}\)} (kinit);
      \draw [->] (kinit) -- (kcheck);
      \draw [->] (kcheck) -- node [auto] {NO} (trees);
      \draw [->] (trees) -- node [auto] {\(G[1:n],S[1:n]\)} (kcheck2);
      \draw [->] (kcheck2) -- node [auto,swap] {YES} (cdef);
      \draw [->] (kcheck) -| node [auto,swap,pos=0.2] {YES} (cdef);
      \draw [->] (cdef) -- (done);
      \draw [->] (kcheck2) -- node [auto] {NO} (iinit);
      \draw [->] (iinit) -- (icheck);
      \draw [->] (icheck) -- node [auto] {NO} (kinc);
      \draw [->] (kinc) |- (kcheck2);
      \draw [->] (icheck) -- node [auto] {YES} (select);
      \draw [->] (select) -- (zykov);
      \draw [->] (zykov) -- node [auto] {\(G,S,R\)} (check);
      \draw [->] (check) -- node [auto] {NO} (iinc);
      \draw [->] (iinc) -- ++(-3cm,0) |- (icheck);
      \draw [->] (check) -- node [auto] {YES} (color);
      \draw [->] (color) -- node [auto] {\(c\)} (done2);
    \end{tikzpicture}
  }
  \caption{The proposed algorithm main routine.}
  \label{fig:outer}
\end{figure}

Note that in the case of the null or an empty graph, the upper and lower bounds for \(k\) will match and the main
routine will terminate immediately with the default greedy coloring.  Thus, the recursive subroutine is always
called with \(k\ge2\).  Also note that the main routine is guaranteed to terminate because \(k\) will eventually
reach \(k_{max}\) or the recursive subroutine will return true when a simplification occurs such that \(n(G)\le k\).

\subsection{The Recursive Subroutine}\label{sec:sub:called}

The recursive subroutine executes a modified version of the Zykov algorithm to determine whether a graph is
\colorable{k}.  It accepts a graph \(G\) of order \(n\) and size \(m\), a list of already removed vertices \(S\)
(in removed order), and the target value of \(k\ge2\) as inputs.  It returns a possibly simplified version of
\(G\), a possibly extended list of removed vertices \(S\), and a boolean value \(R\) indicating whether or not
\(G\) is \colorable{k}.  Internally, various tests are applied to prune the corresponding Zykov tree or abandon it
all together based on the current value of \(k\).

The steps of the recursive subroutine and references to their associated theorems are as follows:

\begin{enumerate}
\item\label{step:sub:kcheck} If \(n\le k\) set \(R\) to true and go to step~\ref{step:sub:done}
  (\theoremname~\ref{thm:nlek}).

\item\label{step:sub:ecalc} Calculate a maximum edge threshold:
  \[a=\frac{n^2(k-1)}{2k}\]

\item\label{step:sub:echeck} If \(m>a\) then set \(R\) to false and go to step~\ref{step:sub:done}
  (Corollary~\ref{cor:density}).

\item\label{step:sub:xcalc} Construct the set \(X\) of all vertices with degree less than \(k\):
  \[X=\setb{v\in V(G)}{\deg(v)<k}\]

\item\label{step:sub:xcheck} If \(X\ne\emptyset\) then replace \(G\) with \(G-X\), append \(X\) to \(S\), and go to
  step~\ref{step:sub:kcheck} (Corollary~\ref{cor:remall}).

\item\label{step:sub:common} Calculate the common number of neighbors between each pair of vertices in \(G\),
  stopping if one vertex's neighborhood is found to be a subset of another.

\item\label{step:sub:subset} If \(G\) has vertices \(u\) and \(v\) such that \(N(u)\subseteq N(v)\) then replace
  \(G\) with \(G\cdot uv\) and go to step~\ref{step:sub:kcheck} (Theorem~\ref{cor:subset}).

\item\label{step:sub:select} Let \(b\) be the smallest number of common neighbors between any pair of vertices in
  \(G\) based on the calculations in step~\ref{step:sub:common}:
  \[b=\min_{u,v\in V(G)}\abs{N(u)\cap N(v)}\]

\item\label{step:sub:ubcalc} Calculate an upper bound for the minimum number of common neighbors between any pair
  of vertices in \(G\):
  \[c=n-2-\frac{n-2}{k-1}\]

\item\label{step:sub:ubcheck} If \(b>c\) then set \(R\) to false and go to step~\ref{step:sub:done}.
  (Corollary~\ref{cor:inter}).

\item\label{step:sub:lower} Use the Edwards Elphick algorithm to calculate a chromatic number lower bound \(\ell\)
  for the current state of \(G\).

\item\label{step:sub:lcheck} If \(\ell>k\) then set \(R\) to false and go to step~\ref{step:sub:done}.

\item\label{step:sub:select2} Select two non-adjacent vertices \(u,v\in V(G)\) with the smallest number of common
  neighbors based on the calculations in step~\ref{step:sub:common}.  It will be shown below that such a pair of
  vertices is guaranteed to exist.

\item\label{step:sub:call1} Assume that \(u\) and \(v\) are assigned the same color by letting \(G'=G\cdot uv\).
  Also make a copy of the removed vertices list \(R'=R\).  Recursively call this routine using \(G'\), \(S'\), and
  \(k\) as inputs to see if \(G'\) is \colorable{k}.  Note that \(G'\) and \(S'\) may be modified.  If \(G'\) is
  \colorable{k} then replace \(G=G'\) and \(S=S'\), set \(R\) to true, and go to step~\ref{step:sub:done}
  (\theoremname~\ref{thm:recurse}).

\item\label{step:sub:call2} Assume that \(u\) and \(v\) are assigned different colors by letting \(G'=G+uv\).  Also
  make a copy of the removed vertices list \(R'=R\).  Recursively call this routine using \(G'\), \(S'\), and \(k\)
  as inputs to see if \(G'\) is \colorable{k}.  Note that \(G'\) and \(S'\) may be modified.  If \(G'\) is
  \colorable{k} then replace \(G=G'\) and \(S=S'\), set \(R\) to true, and go to step~\ref{step:sub:done}
  (\theoremname~\ref{thm:recurse}).

\item\label{step:sub:fail} Conclude that \(G\) is not \colorable{k} and set \(R\) to false.

\item\label{step:sub:done} Return the determine result \(R\), the possibly simplified \(G\), and the possibly
  extended list of removed vertices \(S\).
\end{enumerate}

A flowchart of these steps is shown in \figurename~\ref{fig:called}.

\begin{figure}[H]
  \centering
  \scalebox{0.55}{
    \begin{tikzpicture}[>=latex']
      \node (start) [draw,terminal] at (0,0) {START};
      \node (kcheck) [draw,decision,below=of start] {\(n\le k\)?};
      \node (rtrue1) [draw,process,right=2cm of kcheck] {\(R=\)TRUE};
      \node (return1) [draw,terminal,below=of rtrue1] {RETURN \(R,G,S\)};
      \node (ecalc) [draw,process,below=of kcheck] {\(a=\frac{n^2(k-1)}{2k}\)};
      \node (echeck) [draw,decision,below=of ecalc] {\(m\le a\)?};
      \node (xcalc) [draw,process,below=of echeck] {\(X=\setb{v\in V(G)}{\deg(v)<k}\)};
      \node (xcheck) [draw,decision,below=of xcalc] {\(X=\emptyset\)?};
      \node (remnode) [draw,process,left=of xcheck] {\(\begin{array}{c} G=G-X \\ S=S+X \end{array}\)};
      \node (xjoin) [coordinate] at ($(remnode)-(2cm,0)$) {};
      \node (common) [draw,process,below=of xcheck] {\(\abs{N(u)\cap N(v)}\)};
      \node (subcheck) [draw,decision,below=of common] {\(N(u)\subseteq N(v)\)?};
      \node (subcon) [draw,process,left=of subcheck] {\(G=G\cdot uv\)};
      \node (mincom) [draw,process,below=of subcheck] {\(b=\min\abs{N(u)\cap N(v)}\)};
      \node (mincalc) [draw,process,below=of mincom] {\(c=n-2-\frac{n-2}{k-1}\)};
      \node (mincheck) [draw,decision,below=of mincalc] {\(b\le c\)?};
      \node (mjoin) [coordinate] at ($(mincheck)+(4cm,0)$) {};
      \node (lbcalc) [draw,predproc,below=of mincheck] {EDWARDS \(G\)};
      \node (lbcheck) [draw,decision,below=of lbcalc] {\(\ell\le k\)?};
      \node (rfalse1) [draw,process,right=2cm of lbcheck] {\(R=\)FALSE};
      \node (return2) [draw, terminal,below=of rfalse1] {RETURN \(R,G,S\)};
      \node (save1) [draw,process,right=6cm of rtrue1] {\(\begin{array}{c} G'=G\cdot uv \\ S'=S \end{array}\)};
      \node (call1) [draw,predproc,below=of save1] {CALL \(G',S',k\)};
      \node (check1) [draw,decision,below=of call1] {\(R\)?};
      \node (save2) [draw,process,below=of check1] {\(\begin{array}{c} G'=G+uv \\ S'=S \end{array}\)};
      \node (call2) [draw,predproc,below=of save2] {CALL \(G',S',k\)};
      \node (check2) [draw,decision,below=of call2] {\(R\)?};
      \node (return3) [draw,terminal,below=of check2] {RETURN \(R,G,S\)};
      \node (update) [draw,process,right=3cm of check2] {\(\begin{array}{c} G=G' \\ S=S' \end{array}\)};
      \draw [->] (start) -- node [auto] {\(G,S,k\)} (kcheck);
      \draw [->] (kcheck) -- node [auto,pos=0.2] {YES} (rtrue1);
      \draw [->] (rtrue1) -- (return1);
      \draw [->] (kcheck) -- node [auto] {NO} (ecalc);
      \draw [->] (ecalc) -- (echeck);
      \draw [->] (echeck) -- node [auto] {YES} (xcalc);
      \draw [->] (xcalc) -- (xcheck);
      \draw [->] (xcheck) -- node [auto,swap] {NO} (remnode) -- (xjoin);
      \draw [->] (xcheck) -- node [auto] {YES} (common);
      \draw [->] (common) -- (subcheck);
      \draw [->] (subcheck) -- node [auto,swap] {YES} (subcon);
      \draw [->] (subcon) -| (xjoin) |- (kcheck);
      \draw [->] (subcheck) -- node [auto] {NO} (mincom);
      \draw [->] (mincom) -- (mincalc);
      \draw [->] (mincalc) -- (mincheck);
      \draw [->] (mincheck) -- node [auto] {YES} (lbcalc);
      \draw [->] (mincheck) -- node [auto,pos=0.2] {NO} (mjoin);
      \draw [->] (lbcalc) -- node [auto] {\(\ell\)} (lbcheck);
      \draw [->] (lbcheck) -- node [auto,pos=0.2] {NO} (rfalse1);
      \draw [->] (echeck) -| node [auto,pos=0.1] {NO} (mjoin) -- (rfalse1);
      \draw [->] (rfalse1) -- (return2);
      \draw [->] (lbcheck) -- node [auto,pos=0.2] {YES} ++(0,-3cm) -- ++(8cm,0) |- (save1);
      \draw [->] (save1) -- (call1);
      \draw [->] (call1) -- node [auto] {\(R,G',S'\)} (check1);
      \draw [->] (check1) -- node [auto] {NO} (save2);
      \draw [->] (save2) -- (call2);
      \draw [->] (call2) -- node [auto] {\(R,G',S'\)} (check2);
      \draw [->] (check2) -- node [auto] {NO} (return3);
      \draw [->] (check2) -- node [auto,pos=0.2] {YES} (update);
      \draw [->] (check1) -| node [auto,pos=0.1] {YES} (update);
      \draw [->] (update) |- (return3);
    \end{tikzpicture}
  }
  \caption{The proposed algorithm recursive subroutine.}
  \label{fig:called}
\end{figure}

Step~\ref{step:sub:kcheck} is the success condition.  Success occurs when \(G\) is simplified by removing or
contracting sufficent vertices (sub steps~\ref{step:sub:xcalc}--\ref{step:sub:subset}) and the main routine has
sufficiently incremented \(k\) (main step~\ref{step:outer:incrk}) such that \(n\le k\).

Steps~\ref{step:sub:xcalc}--\ref{step:sub:subset} attempt to simplify \(G\) using vertex removal and contraction in
order to achieve a simpler graph that is equivalently \colorable{k}.  Each time vertices are removed or contracted
the associated branches in the corresponding Zykov tree are pruned.  Since these same steps would just be repeated
for \(k+1\), the subroutine saves the simplified \(G\) as a starting point for the next candidate value of \(k\).

Steps~\ref{step:sub:ecalc}--\ref{step:sub:echeck} and \ref{step:sub:select}--\ref{step:sub:lcheck} apply tests
that attempt to disprove that the current state of \(G\) is \colorable{k} for the current value of \(k\).  If so,
then the current Zykov tree is abandoned and the subroutine returns false.

Steps~\ref{step:sub:select2}--\ref{step:sub:fail} constitute the recursive portion of the modified Zykov algorithm.
The recursive calls are guaranteed to terminate because either there will be sufficient vertex contractions such
that \(n\le k\), resulting in a true return, or sufficient edge additions such that the graph becomes complete and
(as will be shown) is rejected by step~\ref{step:sub:echeck}, resulting in a false return.  Note that in the event
of a false return, any modifications to the current states of \(G\) and \(S\) resulting from the recursive calls
are not returned to the main routine.

\subsection{The Coloring Routine}\label{sec:sub:coloring}

The recursive subroutine will eventually return true when applied to a particular Zykov tree using a particular
value of \(k\).  The final state of \(G\), which should be a complete graph of single and/or contracted vertices,
and the final state of \(S\), which is a list of removed single and/or contracted vertices in the order removed are
used to construct the final chromatic coloring.

According to \theoremname~\ref{thm:greedy}, there exists some ordering of the vertices such that the greedy algorithm
will produce an exact result for the chromatic number of a graph.  The removed vertices in reverse order is
such an ordering.  Consider the example shown in \figurename~\ref{fig:reverse}.  If \(k=4\) and vertex \(v\) is to
be removed since \(\deg(v)=3\), then in the reverse direction when \(v\) is added the fourth color is available for
\(v\).  In general, given a graph \(G-v\) such that \(deg(v)<k\) in \(G\), there will always be an available color
for \(v\) regardless of how \(N(v)\) is colored.  This result is formalized in \theoremname~\ref{thm:remone}.

\begin{figure}[H]
  \centering
  \begin{tikzpicture}
    \draw (0,0) circle [radius=2];
    \node at (0,-1.5) {\(G-v\)};
    \draw (0,0.5) ellipse (1 and 1.25);
    \node (a) [unlabeled node=green] at (0,1.5) {};
    \node (b) [unlabeled node=red] at (0,1.0) {};
    \node (c) [unlabeled node=blue] at (0,0.5) {};
    \node at (0,-0.25) {\(N(v)\)};
    \node (v) [unlabeled node] at (4,1) {};
    \node [below=1ex] at (v) {\(v\)};
    \draw (v) edge (a) edge (b) edge (c);
  \end{tikzpicture}
  \caption{Coloring a removed vertex example.}
  \label{fig:reverse}
\end{figure}

The coloring routine accepts a graph \(G\) that is \chromatic{k}, a complete graph \(G_i\) of order \(k\) that
results from running the modified Zykov algorithm on \(G\), and a removed vertex list \(R\) sorted by the order
removed as inputs.  It outputs a \chromatic{k} coloring of \(G\).  The steps of the coloring routine are as
follows:
\begin{enumerate}
\item Start with \(k\) empty color classes \(\set{c_1,\ldots,c_k}\).
\item Order the vertices in \(G_i\) into a list.  Note that these vertices represent single and/or contracted
  vertices from \(G\).
\item Append the vertices in \(S\) to the list in reverse order.  Note that these vertices also represent
  single and/or contracted vertices from \(G\).
\item Assign all of the vertices from \(G\) that are represented by \(v_j\in V(G_i)\) to color \(c_j\).
\item Greedy color the remaining vertices in the list.  No additional colors should be required.  Note that when
  determining color use by adjacent vertices, all of the single and/or contracted vertices from \(G\) represented
  by a vertex in the sorted list must be checked against all the single and/or contracted vertices from \(G\)
  represented by the previously colored vertices in the sorted list.
\end{enumerate}

Consider the example graph \(G\) and its reduced form \(G_1\) in \figurename~\ref{fig:crex}.

\begin{figure}[H]
  \begin{minipage}{2.5in}
    \centering
    \begin{tikzpicture}[every node/.style={labeled node}]
      \node (0) at (0,0) {\(0\)};
      \node (1) [left=of 0] {\(1\)};
      \node (2) [left=of 1] {\(2\)};
      \node (3) [below=of 0] {\(3\)};
      \node (4) [below=of 1] {\(4\)};
      \node (5) [below=of 2] {\(5\)};
      \draw (0) edge (1) edge (3);
      \draw (1) edge (2) edge (3) edge (4);
      \draw (3) edge (4);
      \draw (4) edge (5);
    \end{tikzpicture}

    \(G\)
  \end{minipage}
  \begin{minipage}{2in}
    \centering
    \begin{tikzpicture}[every node/.style={labeled node}]
      \node (1) at (0,0) {\(1\)};
      \node (04) [below=of 1] {\(04\)};
      \node (3) [left=of 04] {\(3\)};
      \draw (1) edge (04) edge (3);
      \draw (04) edge (3);
    \end{tikzpicture}

    \(G_1\)
  \end{minipage}
  \begin{minipage}{1in}
    \begin{tabular}{|c|}
      \hline
      \(S\) \\
      \hline
      \(2\) \\
      \(5\) \\
      \hline
    \end{tabular}
  \end{minipage}
  \caption{A coloring routine example.}
  \label{fig:crex}
\end{figure}

Since \(\X(G)=3\), start with three empty color classes: \(\set{c_1,c_2,c_3}\).  Next, list the vertices in
\(G_1\), followed by the removed vertices in reverse order: \(04,1,3,5,2\).  Now, assign the vertices from \(G_1\)
to the color classes:

\begin{center}
  \begin{tabular}{|c|c|}
    \hline
    \(c_1\) & \(0,4\) \\
    \hline
    \(c_2\) & \(1\) \\
    \hline
    \(c_3\) & \(3\) \\
    \hline
  \end{tabular}
\end{center}

Next, color the removed vertices.  Since \(5\) is adjacent to \(4\) but not to \(1\), \(5\) is assigned to \(c_2\).
Finally, since \(2\) is adjacent to \(1\) but to neither \(0\) nor \(4\), \(2\) is assigned to \(c_1\):

\begin{center}
  \begin{tabular}{|c|c|}
    \hline
    \(c_1\) & \(0,2,4\) \\
    \hline
    \(c_2\) & \(1,5\) \\
    \hline
    \(c_3\) & \(3\) \\
    \hline
  \end{tabular}
\end{center}

\bigskip

\subsection{Supporting Theorems}\label{sec:sub:theorems}

This section contains the theorems that support the steps in the recursive subroutine.  Remember that the success
check of step~\ref{step:sub:kcheck} is already supported by \theoremname~\ref{thm:nlek}.

\subsubsection{Maximum Edge Threshold}\label{sec:sub:sub:edges}

The maximum edge threshold test of steps~\ref{step:sub:ecalc} and \ref{step:sub:echeck} is supported by
Theorem~\ref{thm:density}.

\begin{theorem}[Maximum Edge Threshold]
  \label{thm:density}
  Let \(G\) be a graph of order \(n\) and size \(m\) and let \(k\in\N\).  If \(G\) is \colorable{k} then:
  \[m\le\frac{n^2(k-1)}{2k}\]
\end{theorem}

\begin{proof}
  Assume that \(G\) is \colorable{k}.  This means that \(V(G)\) can be distributed into \(k\) independent (some
  possibly empty) subsets.  Call these subsets \(A_1,\ldots A_k\) and let \(a_i=\abs*{A_i}\).  Thus, each \(v\in
  A_i\) can be adjacent to at most \(n-a_i\) other vertices in \(G\), and hence the maximum number of edges
  incident to vertices in \(A_i\) is given by: \(a_i(n-a_i)=na_i-a_i^2\).  Now, using Theorem~\ref{thm:first}, the
  maximum number of edges in \(G\) is given by:
  \[m\le\frac{1}{2}\sum_{i=1}^k(na_i-a_i^2)\]
  with the constraint:
  \[\sum_{i=1}^ka_i=n\]
  This problem can be solved using the Lagrange multiplier technique.  We start by defining:
  \begin{align*}
    F(a_1,\ldots,a_k) &= f(a_1,\ldots,a_k)-\l g(a_1,\ldots,a_k) \\
    &= \frac{1}{2}\sum_{i=1}^k(na_i-a_i^2)-\l\sum_{i=1}^ka_i \\
    &= \sum_{i=1}^k\left(\frac{1}{2}na_i-\frac{1}{2}a_i^2-\l a_i\right)
  \end{align*}
  Now, optimize by taking the gradient and setting the resulting vector equation equal to the zero vector:
  \[\vec{\nabla}F=\sum_{i=1}^k(\frac{n}{2}-a_i-\l)\hat{a_i}=\vec{0}\]
  This results in a system of \(k\) equations of the form:
  \[\frac{n}{2}-a_i-\l=0\]
  And so:
  \[a_i=\frac{n}{2}-\l\]
  Plugging this result back into the contraint:
  \[\sum_{i=1}^ka_i=\sum_{i=1}^k\left(\frac{n}{2}-\l\right)=k\left(\frac{n}{2}-\l\right)=n\]
  Solving for \(\l\) yields:
  \[\l=\frac{n}{2}-\frac{n}{k}\]
  And finally, to get \(a_i\) in terms of \(n\) and \(k\):
  \[a_i=\frac{n}{2}-\left(\frac{n}{2}-\frac{n}{k}\right)=\frac{n}{k}\]
  Therefore:
  \[m\le\frac{1}{2}\sum_{i=1}^k\left[n\left(\frac{n}{k}\right)-\left(\frac{n}{k}\right)^2\right]=
  \frac{k}{2}\left(\frac{n^2k-n^2}{k^2}\right)=\frac{n^2(k-1)}{2k}\]
\end{proof}

The recursive subroutine actually uses the contrapositive of this result, as stated in Corollary~\ref{cor:density}.

\begin{corollary}
  \label{cor:density}
  Let \(G\) be a graph of order \(n\) and size \(m\) and let \(k\in\N\).  If:
  \[m>\frac{n^2(k-1)}{2k}\]
  then \(G\) is not \colorable{k}.
\end{corollary}

Corollary~\ref{cor:density} is demonstrated by \figurename~\ref{fig:density}.  The shown graph \(G\) has \(n=4\),
\(m=5\), and \(\X(G)=3\).  Testing for \(k=2\):
\[a=\frac{4^2(2-1)}{2\cdot2}=4\]
But \(m=5>4=a\) and so we can conclude that \(G\) is not \colorable{2}.  However, testing for \(k=3\);
\[a=\frac{4^2(3-1)}{2\cdot3}=5.3\]
So \(m=5\ngtr5.3=a\) and thus \(G\) \emph{may} be \(3\)-colorable, since this test only provides a necessary and
not a sufficient condition.

\begin{figure}[H]
  \centering
  \begin{tikzpicture}
    \colorlet{c1}{green!25!white}
    \colorlet{c2}{blue!25!white}
    \colorlet{c3}{red!25!white}
    \begin{scope}[every node/.style={coordinate}]
      \cycleNnodes{4}{(0,0)}{0.75in}{135}{c};
    \end{scope}
    \begin{scope}[every node/.style={labeled node}]
      \node [fill=c1] (a) at (c1) {\(a\)};
      \node [fill=c2] (b) at (c2) {\(b\)};
      \node [fill=c1] (c) at (c3) {\(c\)};
      \node [fill=c3] (d) at (c4) {\(d\)};
    \end{scope}
    \draw (a) edge (b) edge (d);
    \draw (b) edge (c) edge (d);
    \draw (c) edge (d);
  \end{tikzpicture}

  \(G\)
  \caption{Corollary~\ref{cor:density} example.}
  \label{fig:density}
\end{figure}

In fact, the the test of Corollary~\ref{cor:density} will always fail for a complete graph when \(k<n\).  Since
\(k,n>0\):
\begin{align*}
  \frac{n(n-1)}{2}-\frac{n^2(k-1)}{2k} &= \frac{kn(n-1)-n^2(k-1)}{2k} \\
  &= \frac{kn^2-kn-kn^2+n^2}{2k} \\
  &= \frac{n^2-kn}{2k} \\
  &= \frac{n(n-k)}{2k} \\
  &>0\qquad(n>k)
\end{align*}

\subsubsection{Vertex Removal}\label{sec:sub:sub:vertex}

Steps~\ref{step:sub:xcalc} and \ref{step:sub:xcheck} remove vertices with degrees less than \(k\).  The idea is
that since a vertex's neighbors only use less than \(k\) colors there will always be an available color for the
vertex without adding a new color, regardless of how the vertex's neighbors are actually colored.  In order words,
these small degree vertices do not affect the overall colorability of their graph.  This fact is demonstrated by
\figurename~\ref{fig:remone}; no matter which vertex is removed, the resulting subgraph is still properly colored
using at most four (in fact, three) colors.

\begin{figure}[H]
  \centering
  \scalebox{0.75}{
    \begin{tikzpicture}
      \colorlet{c1}{green!25!white}
      \colorlet{c2}{blue!25!white}
      \colorlet{c3}{red!25!white}
      \colorlet{c4}{yellow!25!white}
      \begin{scope}[every node/.style={coordinate}]
        \cycleNnodes{4}{(0,0)}{0.5in}{135}{x};
      \end{scope}
      \begin{scope} [every node/.style={labeled node}]
        \node [fill=c1] (v1) at (x1) {\(a\)};
        \node [fill=c2] (v2) at (x2) {\(b\)};
        \node [fill=c3] (v3) at (x3) {\(c\)};
        \node [fill=c4] (v4) at (x4) {\(d\)};
      \end{scope}
      \draw (v1) edge (v2) edge (v3) edge (v4);
      \draw (v2) edge (v3) edge (v4);
      \draw (v3) edge (v4);
    \end{tikzpicture}
  }

  \(G\)

  \bigskip

  \begin{minipage}{1.25in}
    \centering
    \scalebox{0.75}{
      \begin{tikzpicture}
        \colorlet{c1}{green!25!white}
        \colorlet{c2}{blue!25!white}
        \colorlet{c3}{red!25!white}
        \colorlet{c4}{yellow!25!white}
        \begin{scope}[every node/.style={coordinate}]
          \cycleNnodes{4}{(0,0)}{0.5in}{135}{x};
        \end{scope}
        \begin{scope} [every node/.style={labeled node}]
          \node [fill=c2] (v2) at (x2) {\(b\)};
          \node [fill=c3] (v3) at (x3) {\(c\)};
          \node [fill=c4] (v4) at (x4) {\(d\)};
        \end{scope}
        \draw (v2) edge (v3) edge (v4);
        \draw (v3) edge (v4);
      \end{tikzpicture}
    }

    \(G-a\)
  \end{minipage}
  \begin{minipage}{1.25in}
    \centering
    \scalebox{0.75}{
      \begin{tikzpicture}
        \colorlet{c1}{green!25!white}
        \colorlet{c2}{blue!25!white}
        \colorlet{c3}{red!25!white}
        \colorlet{c4}{yellow!25!white}
        \begin{scope}[every node/.style={coordinate}]
          \cycleNnodes{4}{(0,0)}{0.5in}{135}{x};
        \end{scope}
        \begin{scope} [every node/.style={labeled node}]
          \node [fill=c1] (v1) at (x1) {\(a\)};
          \node [fill=c3] (v3) at (x3) {\(c\)};
          \node [fill=c4] (v4) at (x4) {\(d\)};
        \end{scope}
        \draw (v1) edge (v3) edge (v4);
        \draw (v3) edge (v4);
      \end{tikzpicture}
    }

    \(G-b\)
  \end{minipage}
  \begin{minipage}{1.25in}
    \centering
    \scalebox{0.75}{
      \begin{tikzpicture}
        \colorlet{c1}{green!25!white}
        \colorlet{c2}{blue!25!white}
        \colorlet{c3}{red!25!white}
        \colorlet{c4}{yellow!25!white}
        \begin{scope}[every node/.style={coordinate}]
          \cycleNnodes{4}{(0,0)}{0.5in}{135}{x};
        \end{scope}
        \begin{scope} [every node/.style={labeled node}]
          \node [fill=c1] (v1) at (x1) {\(a\)};
          \node [fill=c2] (v2) at (x2) {\(b\)};
          \node [fill=c4] (v4) at (x4) {\(d\)};
        \end{scope}
        \draw (v1) edge (v2) edge (v4);
        \draw (v2) edge (v4);
      \end{tikzpicture}
    }

    \(G-c\)
  \end{minipage}
  \begin{minipage}{1.25in}
    \centering
    \scalebox{0.75}{
      \begin{tikzpicture}
        \colorlet{c1}{green!25!white}
        \colorlet{c2}{blue!25!white}
        \colorlet{c3}{red!25!white}
        \colorlet{c4}{yellow!25!white}
        \begin{scope}[every node/.style={coordinate}]
          \cycleNnodes{4}{(0,0)}{0.5in}{135}{x};
        \end{scope}
        \begin{scope} [every node/.style={labeled node}]
          \node [fill=c1] (v1) at (x1) {\(a\)};
          \node [fill=c2] (v2) at (x2) {\(b\)};
          \node [fill=c3] (v3) at (x3) {\(c\)};
        \end{scope}
        \draw (v1) edge (v2) edge (v3);
        \draw (v2) edge (v3);
      \end{tikzpicture}
    }

    \(G-d\)
  \end{minipage}
  \caption{Vertex removal example.}
  \label{fig:remone}
\end{figure}

Vertex removal is supported by \theoremname~\ref{thm:remone}:

\begin{theorem}
  \label{thm:remone}
  Let \(G\) be a graph and let \(v\in V(G)\) such that \(\deg(v)<k\) for some \(k\in\N\).  \(G\) is \colorable{k}
  if and only if \(G-v\) is \colorable{k}.
\end{theorem}

\begin{proof}
  Assume that \(G\) is \colorable{k}.  By definition, there exists some coloring function \(c:V(G)\to C\) where
  \(\abs{C}=k\).  Consider the restricted coloring function \(c'=\restrict{c}{V(G-v)}\) and assume \(uw\in E(G-v)\)
  and hence \(u\ne w\).  Since \(c\) is proper:
  \[c'(u)=c(u)\ne c(w)=c'(w)\]
  This means that \(c':V(G-v)\to C\) is a proper coloring of \(G-v\) with \(\abs{C}=k\).

  Therefore \(G-v\) is \colorable{k}.

  For the converse, assume that \(G-v\) is \colorable{k}.  By definition, there exists some coloring function
  \(c:V(G-v)\to C\) where \(\abs{C}=k\).  By assumption, \(\deg(v)<k\), so \(v\) has at most \(k-1\) neighbors in
  \(G\), using at most \(k-1\) colors.  This means that there is an additional color that can be assigned to \(v\)
  in \(G\) such that the coloring remains proper (see~\figurename~\ref{fig:reverse}).  So let
  \(N(v)=\set{v_1,\ldots,v_r}\subseteq V(G-v)\) for some \(r<k\), and let \(c[N(v)]=\set{c_1,\ldots,c_s}\subset C\)
  for some \(s\le r<k\).  Since \(c[N(v)]\) is a proper subset of \(C\), select \(c_k\in C-c[N(v)]\) and define a
  coloring function \(c':V(G)\to C\) as follows:
  \[c'(u)=\begin{cases}
  c(u), & u\ne v \\
  c_k, & u=v
  \end{cases}\]

  Now, assume that \(uw\in E(G)\) and hence \(u\ne w\).
  \begin{description}
  \item[Case 1:] \(v\notin uw\)

    So \(u,w\in E(G-v)\) and since \(c\) is proper:
    \[c'(u)=c(u)\ne c(w)=c'(w)\]

  \item[Case 2:] \(v\in uw\)

    Assume without loss of generality that \(v=u\).  Thus, \(w\in V(G-v)\) and since \(c\) is proper:
    \[c'(v)=c_k\ne c(u)=c'(u)\]
  \end{description}

  This means that \(c':V(G)\to C\) is a proper coloring of \(G\) with \(\abs{C}=k\).

  Therefore \(G-v\) is \colorable{k}.
\end{proof}

Theorem~\ref{thm:remone} is demonstrated in \figurename~\ref{fig:lowdeg} for \(k=4\) and \(\deg(v)=3\).

\begin{figure}[H]
  \centering
  \begin{tikzpicture}
    \colorlet{c1}{green!25!white}
    \colorlet{c2}{blue!25!white}
    \colorlet{c3}{red!25!white}
    \colorlet{c4}{yellow!25!white}
    \begin{scope}[every node/.style={coordinate}]
      \cycleNnodes{4}{(0,0)}{1in}{135}{c};
    \end{scope}
    \begin{scope}[every node/.style={labeled node}]
      \node [fill=c1] (a) at (c1) {\(a\)};
      \node [fill=c2] (v) at (c2) {\(v\)};
      \node [fill=c3] (b) at (c3) {\(b\)};
      \node [fill=c4] (c) at (c4) {\(c\)};
      \node [fill=c2,below left=of a] (d) {\(d\)};
    \end{scope}
    \draw (a) edge (b) edge (c) edge (d);
    \draw [dashed,red] (v) edge (a) edge (b) edge (c);
    \draw (b) edge (c);
    \draw (c) edge (d);
  \end{tikzpicture}
  \caption{Theorem~\ref{thm:remone} example.}
  \label{fig:lowdeg}
\end{figure}

The recursive subroutine actually removes all such small degree vertices at once, which is supported by the
inductive proof in Corollary \ref{cor:remall}.

\begin{corollary}
  \label{cor:remall}
  Let \(G\) be a graph of order \(n\) and let \(X=\setb{v\in V(G)}{\deg(v)<k}\) for some \(k\in\N\).  \(G\) is
  \colorable{k} if and only if \(G-X\) is \colorable{k}.
\end{corollary}

\begin{proof}
  (by induction on \(\abs{X}\))
  \begin{description}
  \item[Base Case:] Let \(\abs{X}=0\).

    But \(G-X=G\) (trivial case).

  \item[Inductive Assumption:] Let \(\abs{X}=r\).

    Assume that \(G\) is \colorable{k} if and only if \(G-X\) is \colorable{k}.

  \item[Inductive Step:] Consider \(\abs{X}=r+1\).
    
    Since \(\abs{X}=r+1>0\), there exists \(v\in X\) such that \(\deg(v)<k\).  Let \(Y=X-\set{v}\) and note that
    \(\abs{Y}=\abs{X}-1=(r+1)-1=r\).  So, \(G\) is \colorable{k} if and only if \(G-v\) is \colorable{k} (Theorem
    \ref{thm:remone}) if and only if \((G-v)-Y\) is \colorable{k} (inductive assumption).
  \end{description}

  Therefore, by the principle of induction, \(G\) is \colorable{k} if and only if \(G-X\) is \colorable{k}.
\end{proof}

Returning to the example in \figurename~\ref{fig:lowdeg}, note that \(X=\set{v,b,d}\) is the set of all vertices
with degree less than \(4\) and so all three can be removed at once in accordance with Corollary \ref{cor:remall}.

\subsubsection{Neighborhood Subsets}\label{sec:sub:sub:subset}

Step~\ref{step:sub:subset} contracts vertices whose neighborhoods are subsets of other vertices.
\lemmaname~\ref{lem:subsame} states that in any \coloring{k} such vertices can be colored using the same color.

\begin{lemma}
  \label{lem:subsame}
  Let \(G\) be a graph such that:
  \begin{enumerate}
  \item \(n(G)\ge 2\)
  \item \(u,v\in V(G)\) and \(u\ne v\)
  \item \(N(u)\subseteq N(v)\), and hence \(uv\notin E(G)\)
  \end{enumerate}
  and let \(k\in\N\).  In any \coloring{k} of \(G\), \(u\) and \(v\) can be assigned the same color.
\end{lemma}

\begin{proof}
  Assume that the coloring function \(c:V(G)\to C\) is proper where \(\abs{C}=k\).  Since \(N(u)\subseteq N(v)\),
  it must be the case that \(c[N(u)]\subseteq c[N(v)]\).  But \(c(v)\notin c[N(v)]\) and hence \(c(v)\notin N(u)\).
  Therefore, \(c(v)\) is available for \(u\).
\end{proof}

Theorem~\ref{thm:consame} states that any two vertices using the same color can be contracted without affecting
colorability.

\begin{theorem}
  \label{thm:consame}
  Let \(G\) be a graph such that:
  \begin{enumerate}
  \item \(n(G)\ge 2\)
  \item \(u,v\in V(G)\) and \(u\ne v\)
  \item \(u\) and \(v\) are assigned the same color in any proper coloring of \(G\)
  \end{enumerate}
  and let \(k\in\N\).  \(G\) is \colorable{k} if and only if \(G\cdot uv\) is \colorable{k}.
\end{theorem}

\begin{proof}
  Assume that \(G\) is \colorable{k}.  By definition, there exists some coloring function \(c:V(G)\to C\) where
  \(c(u)=c(v)\) and \(\abs{C}=k\).  Let \(w\in V(G\cdot uv)\) be the identified vertex and define a coloring function
  \(c':V(G\cdot uv)\to C\) as follows:
  \[c'(z)=\begin{cases}
  c(z), & z\ne w \\
  c(u)=c(v), & z=w
  \end{cases}\]
  Now, assume that \(xy\in E(G\cdot uv)\) and hence \(x\ne y\).
  \begin{description}
  \item[Case 1:] \(w\notin xy\)

    So \(x,y\in E(G)\) and since \(c\) is proper:
    \[c'(x)=c(x)\ne c(y)=c'(y)\]

  \item[Case 2:] \(w\in xy\)

    Assume without loss of generality (AWLOG) that \(w=x\) and so \(w\ne y\).  Thus, \(y\in V(G)\),
    \(y\notin\set{u,v}\), and since \(c\) is proper:
    \[c'(w)=c(u)\ne c(y)=c'(y)\]
  \end{description}

  This means that \(c':V(G\cdot uv)\to C\) is a proper coloring of \(G\cdot uv\) with \(\abs{C}=k\).

  Therefore \(G\cdot v\) is \colorable{k}.

  For the converse, assume that \(G\cdot uv\) is \colorable{k} and let \(w\in V(G\cdot uv\) be the identified
  vertex.  By definition, there exists some coloring function \(c:V(G\cdot uv)\to C\) where \(\abs{C}=k\).  Define
  the coloring function \(c':V(G)\to C\) as follows:
  \[c'(z)=\begin{cases}
  c(z), & z\notin\set{u,v} \\
  c(w), & z\in\set{u,v}
  \end{cases}\]
  Now, assume that \(xy\in E(G)\) and hence \(x\ne y\).
  \begin{description}
  \item[Case 1:] \(u,v\notin xy\)

    So \(x,y\in E(G\cdot uv)\) and since \(c\) is proper:
    \[c'(x)=c(x)\ne c(y)=c'(y)\]

  \item[Case 2:] \(u\in xy\) and \(v\notin xy\) or \(u\notin xy\) and \(v\in xy\)

    Assume without loss of generality (AWLOG) that \(u=x\) and \(v\ne y\).  Thus, \(y\in V(G\cdot uv)\),
    \(y\ne w\), and since \(c\) is proper:
    \[c'(u)=c(w)\ne c(y)=c'(y)\]
  \end{description}

  This means that \(c':V(G)\to C\) is a proper coloring of \(G\) with \(\abs{C}=k\).

  Therefore \(G\) is \colorable{k}.
\end{proof}

Finally, \corollaryname~\ref{cor:subset} combines the previous two results.

\begin{corollary}
  \label{cor:subset}
  Let \(G\) be a graph such that:
  \begin{enumerate}
  \item \(n(G)\ge 2\)
  \item \(u,v\in V(G)\) and \(u\ne v\)
  \item \(N(u)\subseteq N(v)\), and hence \(uv\notin E(G)\)
  \end{enumerate}
  and let \(k\in\N\).  \(G\) is \colorable{k} if and only if \(G\cdot uv\) is \colorable{k}.
\end{corollary}

\begin{proof}
  By \lemmaname~\ref{lem:subsame}, \(u\) and \(v\) can be assigned the same color.  Therefore, by
  \theoremname~\ref{thm:consame}, \(G\) is \colorable{k} if and only if \(G\cdot uv\) is \colorable{k}.
\end{proof}

Corollary~\ref{cor:subset} is demonstrated in \figurename~\ref{fig:subset}.

\begin{figure}[H]
  \begin{minipage}{2.75in}
    \centering
    \begin{tikzpicture}
      \colorlet{c1}{green}
      \colorlet{c2}{blue}
      \colorlet{c3}{yellow}
      \colorlet{c4}{orange}
      \colorlet{c5}{red}
      \draw (0,0) circle (3);
      \node at (0,-2.75) {\(G\)};
      \draw (0,0) ellipse (1 and 2.5);
      \draw (0,0.5) ellipse (0.75 and 1.5);
      \node (a) [unlabeled node,fill=c1] at (0,1.75) {};
      \node (b) [unlabeled node,fill=c2] at (0,1) {};
      \node (c) [unlabeled node,fill=c3] at (0,0.25) {};
      \node at (0,-0.25) {\(N(u)\)};
      \node (d) [unlabeled node,fill=c4] at (0,-1.5) {};
      \node at (0,-2) {\(N(v)\)};
      \node [unlabeled node,fill=c5] (u) at (-2,0) {};
      \node [below=1ex] at (u) {\(u\)};
      \node [unlabeled node,fill=c5] (v) at (2,0) {};
      \node [below=1ex] at (v) {\(v\)};
      \draw (u) edge (a) edge (b) edge (c);
      \draw (v) edge (a) edge (b) edge (c) edge (d);
    \end{tikzpicture}
  \end{minipage}
  \begin{minipage}{2.75in}
    \centering
    \begin{tikzpicture}
      \colorlet{c1}{green}
      \colorlet{c2}{blue}
      \colorlet{c3}{yellow}
      \colorlet{c4}{orange}
      \colorlet{c5}{red}
      \draw (0,0) circle (3);
      \node at (0,-2.75) {\(G\cdot uv\)};
      \draw (0,0) ellipse (1 and 2.5);
      \node (a) [unlabeled node,fill=c1] at (0,1.75) {};
      \node (b) [unlabeled node,fill=c2] at (0,1) {};
      \node (c) [unlabeled node,fill=c3] at (0,0.25) {};
      \node (d) [unlabeled node,fill=c4] at (0,-1.5) {};
      \node at (0,-2) {\(N(w)\)};
      \node [unlabeled node,fill=c5] (v) at (2,0) {};
      \node [below=1ex] at (v) {\(w\)};
      \draw (v) edge (a) edge (b) edge (c) edge (d);
    \end{tikzpicture}
  \end{minipage}
  \caption{Demonstration of \theoremname~\ref{cor:subset}.}
  \label{fig:subset}
\end{figure}

In the original version of the proposed algorithm, if \(N(u)\subset N(v)\) then \(G\) was replaced by \(G-u\).
Although technically correct, simply removing \(u\) loses the important information that \(u\) and \(v\) are
assigned the same color, which would make the coloring routine more complicated.  Special thanks to
Graham~\cite{graham} for pointing out that contraction is sufficient.

\subsubsection{Minimum Common Neighbor Upper Bound}\label{sec:sub:sub:common}

Steps~\ref{step:sub:select}--\ref{step:sub:ubcheck} establish an upper bound for the minimum common neighbor count
between any two vertices in a graph that is assumed to be \colorable{k}.  This limit is dependent on the following
facts that are guaranteed by previous steps:

\begin{enumerate}
\item \(2\le k<n\)
\item There are no \(u,v\in V(G)\) such that \(N(u)\subseteq N(v)\)
\end{enumerate}

The supporting theorem uses these facts along with Lemma~\ref{lem:neighbor} in its proof.

\begin{lemma}
  \label{lem:neighbor}
  Let \(G\) be a graph and let \(S\) be a non-empty independent subset of \(V(G)\).  If there exists a vertex
  \(v\in S\) such that \(v\) is adjacent to all vertices in \(V(G)-S\) (i.e., \(N(v)=V(G)-S\)) then for all
  vertices \(u\in S\) it is the case that \(N(u)\subseteq N(v)\).
\end{lemma}

\begin{proof}
  Assume that such a \(v\) exists and then assume that \(u\in S\).  If \(u=v\) then (trivially) \(N(v)=N(v)\), so
  assume \(u\ne v\).  Furthermore, since \(u,v\in S\) and \(S\) is independent (by assumption), it must be the case
  that \(u\) and \(v\) are not neighbors.

  \begin{description}
  \item[Case 1:] \(N(u)=\emptyset\).
      
    Therefore, by definition, \(N(u)=\emptyset\subseteq N(v)\).

  \item[Case 2:] \(N(u)\ne\emptyset\).

    Assume that \(w\in N(u)\).  This means that \(w\) is adjacent to \(u\) and hence \(w\notin S\), since \(S\) is
    an independent set.  So \(w\in V(G)-S\) and thus, by assumption, \(v\) is adjacent to \(w\) and we can conclude
    that \(w\in N(v)\).  Therefore \(N(u)\subseteq N(v)\).
  \end{description}

  Therefore, for all \(u\in S\), \(N(u)\subseteq N(v)\).
\end{proof}

Lemma~\ref{lem:neighbor} is demonstrated in \figurename~\ref{fig:neighbor}.  Note that since \(v\in S\) is adjacent
to every vertex in \(V(G)-S\), vertex \(u\in S\) can't help but be adjacent to some subset of \(N(v)\).

\begin{figure}[H]
  \centering
  \begin{tikzpicture}
    \draw (0in,0) ellipse (0.5in and 1in);
    \draw (3in,0) ellipse (1.5in and 1in);
    \begin{scope}[every node/.style={unlabeled node}]
      \node (v) at (0in,0.25in) {};
      \node (u) at (0in,-0.25in) {};
      \node (w1) at (2in,0) {};
      \node (w2) at (2.5in,0) {};
      \node (w3) at (3in,0) {};
      \node (w4) at (3.5in,0) {};
      \node (w5) at (4in,0) {};
    \end{scope}
    \node [left=1ex of v] {\(v\)};
    \node [left=1ex of u] {\(u\)};
    \node at (0,-1.25in) {\(S\)};
    \node at (3in,-1.25in) {\(V(G)-S\)};
    \draw (v) [bend left] edge (w1);
    \draw (v) [bend left] edge (w2);
    \draw (v) [bend left] edge (w3);
    \draw (v) [bend left] edge (w4);
    \draw (v) [bend left] edge (w5);
    \draw (u) [bend right] edge (w2);
    \draw (u) [bend right] edge (w4);
    \draw (u) [bend right] edge (w5);
  \end{tikzpicture}
  \caption{Lemma \ref{lem:neighbor} example.}
  \label{fig:neighbor}
\end{figure}

Theorem~\ref{thm:inter} establishes the desired upper bound.

\begin{theorem}
  \label{thm:inter}
  Let \(G\) be a graph of order \(n\) and size \(m\) such that there are no \(u,v\in V(G)\) where \(N(u)\subseteq
  N(v)\), and let \(k\in\N\) such that \(2\le k<n\).  If \(G\) is \colorable{k} then there exists two vertices
  \(w,z\in V(G)\) such that:
  \[\abs{N(w)\cap N(z)}\le n-2-\frac{n-2}{k-1}\]
\end{theorem}

\begin{proof}
  Assume that \(G\) is \colorable{k}.  This means that \(V(G)\) can be distributed into \(k\) independent (some
  possibly empty) subsets \(A_1,\ldots,A_k\) such that \(a_i=\abs*{A_i}\) and \(a_1\ge a_2\ge\cdots\ge a_k\).
  Since \(n>k\), by the pigeonhole principle, it must be the case that \(a_1\ge2\).  Assume that \(v\in A_1\).

  First, assume by way of contradiction (ABC) that \(v\) is adjacent to all other vertices in \(V(G)-A_1\).  Since
  \(a_1\ge2\), there exists \(u\in A_1\) such that \(u\ne v\) and \(u\) is not adjacent to \(v\).  Thus, by Lemma
  \ref{lem:neighbor}, \(N(u)\subseteq N(v)\), contradicting the assumption.  Note that this contradiction also
  eliminates the degenerate case where \(A_1=V(G)\); however, this case does not occur here because the graph would
  be an empty graph and would have been eliminated by previous steps.  Therefore, there exists some \(v'\in
  V(G)-A_1\) such that \(v\) is not adjacent to \(v'\).  Assume that \(v'\in A_i\) for some \(i\) such that
  \(1<i\le k\):

  \begin{description}
  \item [Case 1:] \(a_i=1\)

    By the pigeonhole principle:
    \[a_1\ge\ceil*{\frac{n-1}{k-1}}\ge\frac{n-1}{k-1}\]
    Now, assume by way of contradiction (ABC) that \(v'\) is adjacent to all vertices in \(V(G)-A_1-A_i\) and
    assume \(u\in N(v)\).  Then it must be the case that \(u\in V(G)-A_1-A_i\), and so \(u\) is adjacent to \(v'\),
    and thus \(u\in N(v')\).  Therefore \(N(v)\subseteq N(v')\), which contradicts the assumption.  This situation
    is demonstrated by \figurename~\ref{fig:aione}.

    \begin{figure}[H]
      \centering
      \begin{tikzpicture}
        \draw (0,0) ellipse (0.5in and 1in);
        \draw (2in,0) ellipse (0.5in and 1in);
        \draw (4in,0) ellipse (0.5in and 1in);
        \begin{scope}[every node/.style={unlabeled node}]
          \node (v) at (0,0) {};
          \node (u1) at (2in,0.5in) {};
          \node (u2) at (2in,0.25in) {};
          \node (u3) at (2in,0in) {};
          \node (u4) at (2in,-0.25in) {};
          \node (u5) at (2in,-0.5in) {};
          \node (vp) at (4in,0) {};
        \end{scope}
        \node [left=1ex of v] {\(v\)};
        \node [right=1ex of vp] {\(v'\)};
        \node at (0,-1.25in) {\(A_1\)};
        \node at (2in,-1.25in) {\(V(G)-A_1-A_i\)};
        \node at (4in,-1.25in) {\(A_i\)};
        \draw (v) edge (u1) edge (u3) edge (u4);
        \draw (vp) edge (u1) edge (u2) edge (u3) edge (u4) edge (u5);
      \end{tikzpicture}
      \caption{Case \(a_i=1\) contradiction.}
      \label{fig:aione}
    \end{figure}

    So there must exist some \(u\in V(G)-A_1-A_i\) such that \(u\) is not adjacent to \(v'\).  This results in the
    upper bound:
    \[\abs{N(v)\cap N(v')}\le n-\abs{\set{u,v'}}-a_1\le n-2-\frac{n-1}{k-1}\]
    Note that since \(v\in A_1\), it is already counted in \(a_1\).  Comparing this bound to the desired bound:
    \[\left(n-2-\frac{n-2}{k-1}\right)-\left(n-2-\frac{n-1}{k-1}\right)=\frac{(n-1)-(n-2)}{k-1}=\frac{1}{k-1}>0\]
    for \(k\ge2\).  Thus the new bound is tighter and so:
    \[\abs{N(v)\cap N(v')}\le n-2-\frac{n-1}{k-1}\le n-2-\frac{n-2}{k-1}\]
    
  \item [Case 2:] \(a_i=2\)

    By the pigeonhole principle:
    \[a_1\ge\ceil*{\frac{n-2}{k-1}}\ge\frac{n-2}{k-1}\]
    This results in the upper bound:
    \[\abs{N(v)\cap N(v')}\le n-a_i-a_1\le n-2-\frac{n-2}{k-1}\]
    
  \item [Case 3:] \(a_i\ge3\)

    By the pigeonhole principle:
    \[a_1\ge\ceil*{\frac{n-a_i}{k-1}}\ge\frac{n-a_i}{k-1}\]
    This results in the upper bound:
    \[\abs{N(v)\cap N(v')}\le n-a_i-a_1\le n-a_i-\frac{n-a_i}{k-1}\]
    Comparing this to the desired bound:
    \[\left(n-2-\frac{n-2}{k-1}\right)-\left(n-a_i-\frac{n-a_i}{k-1}\right)=(a_i-2)\left(1-\frac{1}{k-1}\right)\ge0\]
    for \(k\ge2\) and \(a_i\ge3\).  Thus the new bound is as good or tighter and so:
    \[\abs{N(v)\cap N(v')}\le n-2-\frac{n-2}{k-1}\]
  \end{description}

  Therefore, there exists \(w,v\in V(G)\) such that:
  \[\abs{N(w)\cap N(z)}\le n-2-\frac{n-2}{k-1}\]
\end{proof}

The recursive subroutine actually uses the contrapositive of this result, as stated in Corollary~\ref{cor:inter}.

\begin{corollary}
  \label{cor:inter}
  Let \(G\) be a graph of order \(n\) and size \(m\) such that there are no \(u,v\in V(G)\) where
  \(N(u)\subseteq N(v)\), and let \(k\in\N\) such that \(2\le k<n\).  If for all \(w,z\in V(G)\) it is the case
  that:
  \[\abs{N(w)\cap N(z)}>n-2-\frac{n-2}{k-1}\]
  then \(G\) is not \colorable{k}.
\end{corollary}

Corollary~\ref{cor:inter} is demonstrated in \figurename~\ref{fig:inter}.  The shown graph has \(n=5\), is
\chromatic{3}, and has:
\[\min_{u,v\in V(G)}\abs{N(u)\cap N(v)}=1\]
Testing for \(k=2\):
\[5-2-\frac{5-2}{2-1}=0\]
But \(1>0\) and so we can conclude that \(G\) is not \colorable{2}.  However, testing for \(k=3\):
\[5-2-\frac{5-2}{3-1}=\frac{3}{2}\]
So \(1\ngtr\frac{3}{2}\) and thus \(G\) \emph{may} be \colorable{3}, since this test only provides a necessary and
not a sufficient condition.

\begin{figure}[H]
  \centering
  \begin{tikzpicture}
    \colorlet{c1}{green!25!white}
    \colorlet{c2}{blue!25!white}
    \colorlet{c3}{red!25!white}
    \begin{scope}[every node/.style={labeled node}]
      \node [fill=c1] (a) at (0,0) {\(a\)};
      \node [fill=c2,above left=of a] (b) {\(b\)};
      \node [fill=c3,above right=of a] (c) {\(c\)};
      \node [fill=c2,below left=of a] (d) {\(d\)};
      \node [fill=c3,below right=of a] (e) {\(e\)};
    \end{scope}
    \draw (a) -- (b) -- (c) -- (a) -- (d) -- (e) -- (a);
  \end{tikzpicture}

  \(G\)
  \caption{Corollary \ref{cor:inter} example.}
  \label{fig:inter}
\end{figure}

\subsubsection{Recursive Steps}

If nothing more can be done in the preceding steps then steps~\ref{step:sub:select2}--\ref{step:sub:fail} revert to
branching.  Step~\ref{step:sub:select2} selects two non-adjacent vertices with the smallest number of common
neighbors.  Such a pair must exist.  Otherwise, the current state of \(G\) is complete, which would have been
eliminated by step~\ref{step:sub:echeck}.  The first recursive call (step~\ref{step:sub:call1}) assumes that the
two selected vertices have the same color, so they are contracted.  The second recursive call (step
\ref{step:sub:call2}) assumes that the two selected vertices have different colors, so they are joined by an added
edge.  Each call starts a new branch of the Zykov tree corresponding to the current value of \(k\).  If either call
returns true then it can be concluded that the input graph was indeed \colorable{k}.  Otherwise, it can be
concluded that the input graph is not \colorable{k} and the recursive subroutine returns the state of \(G\) prior
to the recursive calls to the main routine.

These steps are supported by Theorem~\ref{thm:recurse}.

\begin{theorem}
  \label{thm:recurse}
  Let \(G\) be a graph of order \(n\ge2\) and let \(u,v\in G\) such that \(u\) and \(v\) are not adjacent.  \(G\)
  is \colorable{k} if and only if \(G\cdot uv\) or \(G+uv\) is \colorable{k}.
\end{theorem}

\begin{proof}
  Assume that \(G\) is \colorable{k}.  By definition, there exists some coloring function \(c:V(G)\to C\) where
  \(\abs{C}=k\).  There are two possibilities, corresponding to the two recursive choices: \(c(u)=c(v)\) (same
  color) or \(c(u)\ne c(v)\) (different colors).  The case \(c(u)=c(v)\) has already been proven by
  \theoremname~\ref{thm:consame}, so it remains to be proven that \(G\) is \colorable{k} if and only if \(G+uv\) is
  \colorable{k}.

  Assume \(c(u)\ne c(v)\).  By adding edge \(uv\), \(u\) and \(v\) become adjacent and thus must have different
  colors.  Thus, \(u\) and \(v\) can retain their same colors, the coloring is unchanged and remains proper, and
  therefore \(G+uv\) is \colorable{k}.
    
  For the converse, assume that \(G+uv\) is \colorable{k}.  Since \(u\) and \(v\) are adjacent in
  \(G+uv\), they must have different colors.  Once \(uv\) is removed in \(G\), \(u\) and \(v\) are no longer
  adjacent and so there are no requirements on their colors.  Thus, \(u\) and \(v\) can retain their same colors, the
  coloring is unchanged and remains proper, and therefore \(G\) is \colorable{k}.
\end{proof}

\subsection{An Example}\label{sec:sub:example}

In this section, the proposed algorithm will be applied to the Gr\"otzsch graph \(G\) with \(n=11\), \(m=20\), and
\(\X(M)=4\) (see~\sectionname~\ref{sec:sub:sub:mycielski}) shown in \figurename~\ref{fig:example}.

\begin{figure}[H]
  \centering
  \begin{tikzpicture}
    \cycleNnodes{5}{(0,0)}{1.25in}{90}{o};
    \cycleNnodes{5}{(0,0)}{0.75in}{90}{i};
    \begin{scope}[every node/.style={labeled node}]
      \node (0) at (o1) {\(a\)};
      \node (1) at (o2) {\(b\)};
      \node (2) at (o3) {\(c\)};
      \node (3) at (o5) {\(e\)};
      \node (4) at (o4) {\(d\)};
      \node (5) at (i1) {\(f\)};
      \node (6) at (i2) {\(g\)};
      \node (7) at (i3) {\(h\)};
      \node (8) at (i5) {\(j\)};
      \node (9) at (i4) {\(i\)};
      \node (10) at (0,0) {\(k\)};
    \end{scope}
    \draw (0) edge (1);
    \draw (2) edge (1);
    \draw (4) edge (2);
    \draw (3) edge (0);
    \draw (4) edge (3);
    \draw (5) edge (1);
    \draw (5) edge (3);
    \draw (10) edge (5);
    \draw (6) edge (0);
    \draw (6) edge (2);
    \draw (10) edge (6);
    \draw (7) edge (1);
    \draw (7) edge (4);
    \draw (10) edge (7);
    \draw (8) edge (0);
    \draw (8) edge (4);
    \draw (10) edge (8);
    \draw (9) edge (2);
    \draw (9) edge (3);
    \draw (10) edge (9);
  \end{tikzpicture}
  \caption{The Gr\"otzsch example: input graph.}
  \label{fig:example}
\end{figure}

The main routine first executes the Bron Kerbosch algorithm and finds that the chromatic number lower bound is
\(k_{min}=2\).  It then executes the greedy algorithm and finds that the chromatic number upper bound is
\(k_{max}=4\) with the following default coloring:
\[\begin{array}{l}
  \set{a, c, k} \\
  \set{b, e, g, j} \\
  \set{d, f, i} \\
  \set{h}
\end{array}\]
Since the lower and upper bounds do not match, the main routine continues.  The Bron Kerbosch algorithm is executed
on the complement of \(G\) to discover than \(G\) has \(16\) MISs.  Vertex \(a\) is found to occur in the least
number of these MISs: \(5\) times.  These \(5\) selected MISs, sorted by decreasing length, are as follows:
\[\begin{array}{l}
  \set{a, c, f, h} \\
  \set{a, d, f, i} \\
  \set{a, f, h, i} \\
  \set{a, c, k} \\
  \set{a, d, k}
\end{array}\]
The main routine creates \(5\) Zykov trees, one per selected MIS.  For brevity, the remainder of this example will
focus on the first tree; assume that the other \(4\) trees are processed similarly and that none of them produces a
solution.  The outer loop initializes \(k=2\) and since \(k=2<4=k_{max}\), the recursive subroutine is called on
the first tree with \(k=2\).

The starting graph for the first tree is constructed by contracting all of the vertices in the first MIS and then
adding edges between the contracted vertex and all other vertices.  The result is shown in
\figurename~\ref{fig:ex:ginit}.

\begin{figure}[H]
  \centering
  \begin{tikzpicture}
    \cycleNnodes{5}{(0,0)}{1.25in}{90}{o};
    \cycleNnodes{5}{(0,0)}{0.75in}{90}{i};
    \begin{scope}[every node/.style={labeled node}]
      \node (11) at (o1) {\(acfh\)};
      \node (1) at (o2) {\(b\)};
      \node (3) at (o5) {\(e\)};
      \node (4) at ($(o4)+(-0.5cm,0)$) {\(d\)};
      \node (6) at (i2) {\(g\)};
      \node (8) at (i5) {\(j\)};
      \node (9) at (i4) {\(i\)};
      \node (10) at (0,0) {\(k\)};
    \end{scope}
    \draw (11) edge (1);
    \draw (4) edge (11);
    \draw (3) edge (11);
    \draw (4) edge (3);
    \draw (10) edge (11);
    \draw (6) edge (11);
    \draw (10) edge (6);
    \draw (8) edge (11);
    \draw (8) edge (4);
    \draw (10) edge (8);
    \draw (9) edge (11);
    \draw (9) edge (3);
    \draw (10) edge (9);
  \end{tikzpicture}
  \caption{The Gr\"otzsch example: first tree initial graph.}
  \label{fig:ex:ginit}
\end{figure}

Since \(n=8\ge2=4\), the called routine continues and calculates a maximum edge threshold of \(a=16\).  Since
\(m=13<=16=a\) the called routine continues and finds that \(\deg(b)=1<2=k\) and so vertex \(b\) is removed and
added to the removed vertex list \(R\).  The resulting graph is shown in \figurename~\ref{fig:ex:remb}.

\begin{figure}[H]
  \centering
  \begin{tikzpicture}
    \cycleNnodes{5}{(0,0)}{1.25in}{90}{o};
    \cycleNnodes{5}{(0,0)}{0.75in}{90}{i};
    \begin{scope}[every node/.style={labeled node}]
      \node (11) at (o1) {\(acfh\)};
      \node (3) at (o5) {\(e\)};
      \node (4) at ($(o4)+(-0.5cm,0)$) {\(d\)};
      \node (6) at (i2) {\(g\)};
      \node (8) at (i5) {\(j\)};
      \node (9) at (i4) {\(i\)};
      \node (10) at (0,0) {\(k\)};
    \end{scope}
    \draw (4) edge (11);
    \draw (3) edge (11);
    \draw (4) edge (3);
    \draw (10) edge (11);
    \draw (6) edge (11);
    \draw (10) edge (6);
    \draw (8) edge (11);
    \draw (8) edge (4);
    \draw (10) edge (8);
    \draw (9) edge (11);
    \draw (9) edge (3);
    \draw (10) edge (9);
  \end{tikzpicture}
  \caption{The Gr\"otzsch example: vertex \(b\) removed.}
  \label{fig:ex:remb}
\end{figure}

Now, \(n=7\ge2=k\) so the called routine continues.  The new maximum edge threshold is \(a=12.25\) and since
\(m=12\le12.25=a\) the called routine continues.  There are no more small degree vertices; however it is found that
\(N(g)\subseteq N(j)\) and so these two vertices are contracted.  The resulting graph is shown in
\figurename~\ref{fig:ex:congj}.

\begin{figure}[H]
  \centering
  \begin{tikzpicture}
    \cycleNnodes{5}{(0,0)}{1.25in}{90}{o};
    \cycleNnodes{5}{(0,0)}{0.75in}{90}{i};
    \begin{scope}[every node/.style={labeled node}]
      \node (11) at (o1) {\(acfh\)};
      \node (3) at (o5) {\(e\)};
      \node (4) at ($(o4)+(-0.5cm,-1cm)$) {\(d\)};
      \node (12) at (i2) {\(gj\)};
      \node (9) at (i4) {\(i\)};
      \node (10) at (0,0) {\(k\)};
    \end{scope}
    \draw (4) edge (11);
    \draw (3) edge (11);
    \draw (4) edge (3);
    \draw (10) edge (11);
    \draw (12) edge (11);
    \draw (10) edge (12);
    \draw (12) edge (4);
    \draw (9) edge (11);
    \draw (9) edge (3);
    \draw (10) edge (9);
  \end{tikzpicture}
  \caption{The Gr\"otzsch example: vertices \(g\) and \(j\) contracted.}
  \label{fig:ex:congj}
\end{figure}

Now, \(n=6\ge2=k\) so the called routine continues.  The new maximum edge threshold is \(a=9\) and since
\(m=10>9=a\) the maximum edge check fails.  The called routine returns with false, the current graph, and a removed
vertex list of \(R=(b)\).  The remaining \(4\) trees similarly fail, so the main routine increments \(k=3\) and
since \(k=3<4=k_{max}\) the recursive subroutine is recalled on the first tree.

Since \(n=6>3=k\) the called routine continues.  The new maximum edge threshold is \(a=12\) and since
\(m=10<=12=a\) the called routine continues.  No small degree vertices or neighborhood subsets are found, so
vertices \(d\) and \(e\) are selected as having the smallest number of common neighbors: \(b=1\).  The minimum
common neighbors upper bound is calculated to be \(c=2\) and since \(b=1<2=c\) the called routine continues.
The Edwards Elphick algorithm is executed to determine that the new chromatic number lower bound is \(3\), and
since \(3<=3=k\), the called routine continues.

At this point all bounding efforts have failed so it is time to start branching.  Since all nonadjacent vertices
share \(2\) common neigbors, vertices \(e\) and \(k\) are selected and are contracted.  The resulting graph is
shown in \figurename~\ref{fig:ex:conek}.  Copies are made of this graph and the removed vertex list and a recursive
call is made with these values.

\begin{figure}[H]
  \centering
  \begin{tikzpicture}
    \cycleNnodes{5}{(0,0)}{1.25in}{90}{o};
    \cycleNnodes{5}{(0,0)}{0.75in}{90}{i};
    \begin{scope}[every node/.style={labeled node}]
      \node (11) at (o1) {\(acfh\)};
      \node (13) at (o5) {\(ek\)};
      \node (4) at ($(o4)+(-0.5cm,-1cm)$) {\(d\)};
      \node (12) at (i2) {\(gj\)};
      \node (9) at (i4) {\(i\)};
    \end{scope}
    \draw (4) edge (11);
    \draw (13) edge (11);
    \draw (4) edge (13);
    \draw (12) edge (11);
    \draw (13) edge (12);
    \draw (12) edge (4);
    \draw (9) edge (11);
    \draw (9) edge (13);
  \end{tikzpicture}
  \caption{The Gr\"otzsch example: vertices \(e\) and \(k\) contracted.}
  \label{fig:ex:conek}
\end{figure}

Now, \(n=5\ge3=k\) so the called routine continues.  The new maximum edge threshold is \(a=8.33\) and since
\(m=8\le8.33=a\) the called routine continues.  Since \(deg(i)=2<3=k\), vertex \(i\) is removed and the removed
vertex list is now \(R=(b,i)\).  The resulting graph is shown in \figurename~\ref{fig:ex:remi}.

\begin{figure}[H]
  \centering
  \begin{tikzpicture}
    \cycleNnodes{5}{(0,0)}{1.25in}{90}{o};
    \cycleNnodes{5}{(0,0)}{0.75in}{90}{i};
    \begin{scope}[every node/.style={labeled node}]
      \node (11) at (o1) {\(acfh\)};
      \node (13) at (o5) {\(ek\)};
      \node (4) at (o4) {\(d\)};
      \node (12) at (i2) {\(gj\)};
    \end{scope}
    \draw (4) edge (11);
    \draw (13) edge (11);
    \draw (4) edge (13);
    \draw (12) edge (11);
    \draw (13) edge (12);
    \draw (12) edge (4);
  \end{tikzpicture}
  \caption{The Gr\"otzsch example: vertex \(i\) removed.}
  \label{fig:ex:remi}
\end{figure}

The current state is now a leaf node with a complete graph of order 4.  Since \(n=4>3=k\) the called routine
continues; however, the new maximum edge threshold is \(a=5.33\) and since \(m=6>5.33\) the edge threshold test
fails as expected and the recursive call returns false.  Modifications to the graph and removed vertex list
resulting from the contraction are discarded and the graph of \figurename~\ref{fig:ex:congj} remains the current graph
for the first tree with a removed vertex list of just \(R=(b)\).

For the next branch, the edge \(ek\) is added.  The resulting graph is shown in \figurename~\ref{fig:ex:addek}.
Copies are made of this graph and the removed vertex list and a recursive call is made with these values.

\begin{figure}[H]
  \centering
  \begin{tikzpicture}
    \cycleNnodes{5}{(0,0)}{1.25in}{90}{o};
    \cycleNnodes{5}{(0,0)}{0.75in}{90}{i};
    \begin{scope}[every node/.style={labeled node}]
      \node (11) at (o1) {\(acfh\)};
      \node (3) at (o5) {\(e\)};
      \node (4) at ($(o4)+(-0.5cm,-1cm)$) {\(d\)};
      \node (12) at (i2) {\(gj\)};
      \node (9) at (i4) {\(i\)};
      \node (10) at (0,0) {\(k\)};
    \end{scope}
    \draw (4) edge (11);
    \draw (3) edge (11);
    \draw (4) edge (3);
    \draw (10) edge (11);
    \draw (12) edge (11);
    \draw (10) edge (12);
    \draw (12) edge (4);
    \draw (9) edge (11);
    \draw (9) edge (3);
    \draw (10) edge (9);
    \draw (3) edge (10);
  \end{tikzpicture}
  \caption{The Gr\"otzsch example: edge \(ek\) added.}
  \label{fig:ex:addek}
\end{figure}

Now, \(n=6>3=k\) so the called routine continues and calculates the new maximum edge threshold \(a=12\).  Since
\(m=11<12=a\) the called routine continues.  No small degree vertices are found; however, it is found that
\(N(gj)\subseteq N(e)\), so these vertices are contracted.  The resulting graph is shown in
\figurename~\ref{fig:ex:congje}.

\begin{figure}[H]
  \centering
  \begin{tikzpicture}
    \cycleNnodes{5}{(0,0)}{1.25in}{90}{o};
    \cycleNnodes{5}{(0,0)}{0.75in}{90}{i};
    \begin{scope}[every node/.style={labeled node}]
      \node (11) at (o1) {\(acfh\)};
      \node (13) at (o5) {\(egj\)};
      \node (4) at ($(o4) + (-0.5cm,0)$) {\(d\)};
      \node (9) at (i4) {\(i\)};
      \node (10) at (0,0) {\(k\)};
    \end{scope}
    \draw (4) edge (11);
    \draw (13) edge (11);
    \draw (4) edge (13);
    \draw (10) edge (11);
    \draw (10) edge (13);
    \draw (9) edge (11);
    \draw (9) edge (13);
    \draw (10) edge (9);
  \end{tikzpicture}
  \caption{The Gr\"otzsch example: vertices \(gj\) and \(e\) contracted.}
  \label{fig:ex:congje}
\end{figure}

Now, \(n=5>3=k\) so the called routine continues and calculates the new edge threshold \(a=8.33\).  Since
\(m=8<8.33=a\) the called routine continues and finds that \(\deg(d)=2<3=k\) so vertex \(d\) is removed.  The
resulting graph is shown in \figurename~\ref{fig:ex:remd} and the new removed vertex list is now \(R=(b,d)\).

\begin{figure}[H]
  \centering
  \begin{tikzpicture}
    \cycleNnodes{5}{(0,0)}{1.25in}{90}{o};
    \cycleNnodes{5}{(0,0)}{0.75in}{90}{i};
    \begin{scope}[every node/.style={labeled node}]
      \node (11) at (o1) {\(acfh\)};
      \node (13) at (o5) {\(egj\)};
      \node (9) at (i4) {\(i\)};
      \node (10) at (0,0) {\(k\)};
    \end{scope}
    \draw (13) edge (11);
    \draw (10) edge (11);
    \draw (10) edge (13);
    \draw (9) edge (11);
    \draw (9) edge (13);
    \draw (10) edge (9);
  \end{tikzpicture}
  \caption{The Gr\"otzsch example: vertex \(d\) removed.}
  \label{fig:ex:remd}
\end{figure}

Once again, a leaf node with a complete graph of order \(4\) is achieved.  Since \(n=4>3=k\) the called routine
continues.  The new maximum edge threshold is \(a=5.33\) and since \(m=6>5.33=a\) the edge threshold test fails as
expected.  All of the changes to the graph and removed vertex list applied by this branch are discarded and the
recursive call returns false.

At this point, both branches have failed and so the called routine returns false to the main routine with the graph
shown in \figurename~\ref{fig:ex:congj} and a removed vertex list of just \(R=(b)\).  The remaining \(4\) trees
similarly fail, so the main routine increments \(k=4\) and since \(k=4=4=k_{max}\) the default \coloring{4} is
accepted as the chromatic coloring for the input graph.

\tablename~\ref{tab:ex:results} compares the performance of the proposed, Christofides/Wang, and Zykov algorithms
when applied to the Gr\"{o}tzsch graph, which has \(n=11\) and \(m=20\).  The table shows the number of calls to
the recursive routine and the mean time duration for \(100\) trials.  The proposed algorithm has a slight advantage
over Christofides/Wang and a somewhat larger advantage over Zykov.

\begin{table}[H]
  \centering
  \caption{Gr\"{o}tzsch Graph Algorithm Performance}
  \label{tab:ex:results}
  \begin{tabular}{|c|c|c|}
    \hline
    ALGORITHM & CALLS & DURATION(s) \\
    \hline
    proposed & \(20\) & \(0.0013\) \\
    \hline
    Christofides/Wang & \(40\) & \(0.0015\) \\
    \hline
    Zykov & \(175\) & \(00062\) \\
    \hline
  \end{tabular}
\end{table}

\tablename~\ref{tab:ex:results2} compares the performance of the three algorithms when applied to the \chromatic{5}
Mycielski graph, which has \(n=23\) and \(m=71\). The proposed algorithm still has a very slight advantage over
Christofides/Wang; however, the Zykov algorithm performance is horrible.  This is most probably due to the
unbalanced nature of the graph: the proposed algorithm can eliminate the lower-degree vertices; however, the Zykov
algorithm is forced to branch on them.

\begin{table}[H]
  \centering
  \caption{Mycielski \chromatic{5} Graph Algorithm Performance}
  \label{tab:ex:results2}
  \begin{tabular}{|c|c|c|}
    \hline
    ALGORITHM & CALLS & DURATION(s) \\
    \hline
    proposed & \(723\) & \(0.0999\) \\
    \hline
    Christofides/Wang & \(1420\) & \(0.1078\) \\
    \hline
    Zykov & 2848167 & \(>250\) \\
    \hline
  \end{tabular}
\end{table}

\sectionname~\ref{sec:sub:duration} contains additional runtime duration comparisons of the three algorithms.


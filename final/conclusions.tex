\section{Conclusions}\label{sec:conclusions}

The primary goal of this research was to select a chromatic coloring algorithm that could be used in a part
consolidation tool for AD designers.  The tool needs to handle input FR design graphs with up to \(20\)
vertices and moderate (\(50\%\)) edge density and must be able to deliver an answer in under one minute.

Three algorithms were considered:
\begin{enumerate}
\item The Christofides Algorithm~(\sectionname~\ref{sec:sub:christofides}) with the
  Wang improvements~(\sectionname~\ref{sec:sub:wang})
\item The Zykov algorithm~(\sectionname~\ref{sec:sub:zykov})
\item A new proposed algorithm~(\sectionname~\ref{sec:algorithm})
\end{enumerate}

These three algorithms were originally compared using empirical runtime complexity values obtained from a random
graph analysis.  For the purposes of this analysis, a step was defined to be a call to a routine that performs all
of the necessary processing for a state in an algorithms state tree.  The results are summarized in
\tablename~\ref{tab:rtresults}.

\begin{table}[H]
  \centering
  \caption{Runtime Complexity Comparison}
  \label{tab:rtresults}
  \begin{tabular}{|c|c|}
    \hline
    ALGORITHM & RUNTIME COMPLEXITY \\
    \hline
    Christofides/Wang & \(\BO(1.0045^{n^2})\) \\
    \hline
    Zykov & \(\BO(3.3^n)\) \\
    \hline
    Proposed & \(\BO(1.0103^{n^2})\) \\
    \hline
  \end{tabular}
\end{table}

Assuming that number of states processed translates to runtime duration, tt would seem that the Christofides/Wang
algorithm would have a slight speed advantage over the proposed algorithm and a large speed advantage over the
plain Zykobv algorithm.  It was noted that the initial chromatic number lower and upper bound estimate test that was
presented as part of the new proposed algorithm could be added to the two well-known algorithms.  Thus, a runtime
duration test was performed limited to random graphs where the estimated lower and upper bounds do not match.  The
results presented in \sectionname~\ref{sec:random} showed that the three algorithms took exactly the same amount of
time to produce a solution, which was well under the one minute requirement.

However, it was found in \sectionname~\ref{sec:sub:example} that there are some graphs where the proposed and the
Christofides/Wang algorithms have a performance advantage.  Nevertheless, it must be admitted that Chistofides/Wang
is a much simpler algorithm, and that the various steps in the proposed algorithm do not seem to deliver a decisive
performance advantage over the target range.  Thus, the simplicity of Christofides/Wang makes it a better choice
for the desired design tool.

\section{Graph Theory}\label{sec:theory}

This section presents the concepts, definitions, and theorems from the field of graph theory that are needed in the
development of the proposed algorithm.  This material is primarily taken from the undergraduate graph theory text
by Chartrand and Zhang (2012)~\cite{chartrand} and the graduate graph theory text by West (2001)~\cite{west}.

\subsection{Simple Graphs}\label{sec:sub:simple}

The problem of part consolidation is best served by a class of graphs called \emph{simple graphs}.  A \emph{simple
  graph} is a mathematical object represented by an ordered pair \(G=(V,E)\) consisting of a finite and non-empty
set of \emph{vertices} (also called \emph{nodes}): \(V(G)\), and a finite and possibly empty set of edges:
\(E(G)\).  Each edge is represented by a two-element subset of \(V(G)\) called the \emph{endpoints} of the edge:
\(E(G)\subseteq\ps_2\left(V(G)\right)\).  For the remainder of this work, the use of the term ``graph'' implies a
``simple graph.''  Thus, a part consolidation problem can be represented by a graph whose vertices are the
functional requirements (FRs) of the design and whose edges indicate which endpoint FRs should never be combined
into a single part.

The choice of two-element subsets of \(V(G)\) for the edges has certain ramifications that are indeed characteristics
that differentiate a simple graph from other classes of graphs:
\begin{enumerate}
\item Every two vertices of a graph are the endpoints of at most one edge; there are no so-called
  \emph{multiple} edges between two vertices.
\item The two endpoint vertices of an edge are always distinct; there are no so-called \emph{loop} edges on a
  single vertex.
\item The two endpoint vertices are unordered, suggesting that an edge provides a bidirectional connection between
  its endpoint vertices.
\end{enumerate}
When referring to the edges in a graph, the common notation of juxtaposition of the vertices will be used instead of
the set syntax.  Thus, edge \(\set{u,v}\) is simply referred to as \(uv\) or \(vu\).

Graphs are often portrayed visually using labeled or filled circles for the vertices and lines for the edges such
that each edge line is drawn between its two endpoint vertices.  An example graph is shown in
\figurename~\ref{fig:exgraph}.

\begin{figure}[H]
  \begin{minipage}{2.75in}
    \vspace{0in}
    \centering
    \begin{tikzpicture}[node distance=1cm,every node/.style={labeled node}]
      \node (E) at (0,0) {\(e\)};
      \node (A) [above left=of E] {\(a\)};
      \node (B) [above right=of E] {\(b\)};
      \node (C) [below right=of E] {\(c\)};
      \node (D) [below left=of E] {\(d\)};
      \draw (A) edge (B);
      \draw (B) edge (E);
      \draw (E) edge (A);
      \draw (A) edge (D);
    \end{tikzpicture}
  \end{minipage}
  \begin{minipage}{2.75in}
    \vspace{0in}
    \centering
    \begin{tikzpicture}[node distance=1.75cm,every node/.style={unlabeled node}]
      \node (E) at (0,0) {};
      \node (A) [above left=of E] {};
      \node (B) [above right=of E] {};
      \node (C) [below right=of E] {};
      \node (D) [below left=of E] {};
      \draw (A) edge (B);
      \draw (B) edge (E);
      \draw (E) edge (A);
      \draw (A) edge (D);
    \end{tikzpicture}
  \end{minipage}
  \begin{gather*}
    V(G)=\set{a,b,c,d,e} \\
    E(G)=\set[\big]{ab,ad,ae,be}
  \end{gather*}
  \vspace{-2\baselineskip}
  \caption{An example graph (labeled and unlabeled).}
  \label{fig:exgraph}
\end{figure}

When two vertices are the endpoints of the same edge, the vertices are said to be \emph{adjacent} or are called
\emph{neighbors}, and the edge is said to \emph{join} its two endpoint vertices.  Furthermore, an edge is said to
be \emph{incident} to its endpoint vertices.  In the example graph of \figurename~\ref{fig:exgraph}, vertex \(a\)
is adjacent to vertices \(b\), \(d\), and \(e\); however, it is not adjacent to vertex \(c\).

As demonstrated by vertex \(c\) in \figurename~\ref{fig:exgraph}, there is no requirement that every vertex in a
graph be an endpoint for some edge.  In fact, a vertex that is not incident to any edge is called an
\emph{isolated} vertex.

We can also speak of adjacent edges, which are edges that share exactly one endpoint.  Note that two edges cannot
share both of their endpoints --- otherwise they would be multiple edges, which are not allowed in simple graphs.
In the example graph of \figurename~\ref{fig:exgraph}, edge \(ab\) is adjacent to edges \(ad\) and \(ae\) via
common vertex \(a\), and \(be\) via common vertex \(b\).

\subsection{Order and Size}\label{sec:sub:ordersize}

Two of the most important characteristics of a graph are its \emph{order} and its \emph{size}.  The \emph{order} of
a graph \(G\), denoted by \(n(G)\) or just \(n\) when \(G\) is unambiguous, is the number of vertices in \(G\):
\(n=\abs{V(G)}\).  The \emph{size} of a graph \(G\), denoted by \(m(G)\) or just \(m\) when \(G\) is unambiguous,
is the number of edges in \(G\): \(m=\abs{E(G)}\).  In the example graph of \figurename~\ref{fig:exgraph}: \(n=5\)
and \(m=4\).

Since every two vertices can have at most one edge between them, the number of edges has an upper bound:

\begin{theorem}
  Let \(G\) be a graph of order \(n\) and size \(m\):
  \[m\le\frac{n(n-1)}{2}\]
\end{theorem}

\begin{proof}
  Since each pair of distinct vertices in \(V(G)\) can have zero or one edges joining them, the maximum number of
  possible edges is \(\binom{n}{2}\), and so:
  \[m\le\binom{n}{2}=\frac{n!}{2!(n-2)!}=\frac{n(n-1)}{2}\]
\end{proof}

Some choices of graph order and size lead to certain degenerate cases that serve as important termination cases for
the the proposed algorithm:
\begin{itemize}
\item The \emph{null} graph is the non-graph with no vertices \((n=m=0)\).
\item The \emph{trivial} graph is the graph with exactly one vertex and no edges \((n=1,m=0)\).  Otherwise
  \((n>1)\), a graph is called \emph{non-trivial}.
\item An \emph{empty} graph is a graph containing no edges \((m=0)\).
\item A \emph{complete} graph is a graph containing every possible edge \(\left(m=\frac{n(n-1)}{2}\right)\).
\end{itemize}

Note that both the null and trivial graphs are empty.

\subsection{Graph Relations}\label{sec:sub:relations}

In addition to its vertices and edges, a graph may be associated with one or more relations.  Each relation has
\(V(G)\) or \(E(G)\) as its domain and is used to associate vertices or edges with problem-specific attributes such
as labels or colors.  Note that there are no particular limitations on the nature of such relations --- everything
from a basic relation to a bijective function are possible.  Some authors include these relations and their
codomains as part of the graph tuple; however, since these extra tuple elements don't affect the structure of a
graph, we will not do so.

In practice, when a graph theory problem requires a particular vertex or edge attribute, the presence of some
corresponding relation \(\sR\) is assumed and we say something like, ``vertex \(v\) has attribute \(a\),'' instead
of the more formal, ``vertex \(v\) has attribute \(\sR(v)\).''

The following sections describe the two relations used by the proposed algorithm.

\subsubsection{Labels}\label{sec:sub:sub:labels}

One possible relation associated with a graph \(G\) is a bijective function \(\ell:V(G)\to L\) that assigns to each
vertex a unique identifying label.  The codomain \(L\) is the set of available labels.  When such a function is
present, the graph is said to be a \emph{labeled} graph and the vertices are considered to be distinct.  Otherwise,
a graph is said to be \emph{unlabeled} and the vertices are considered to be identical (only the structure of the
graph matters).

The vertices in a labeled graph are typically drawn as open circles containing the corresponding labels, whereas
the vertices in an unlabeled graph are typically drawn as filled circles.  This is demonstrated in the example
graph of \figurename~\ref{fig:exgraph}: the graph on the left is labeled and the graph on the right is unlabeled.

Since the labeling function \(\ell\) is bijective, a vertex \(v\in V(G)\) with label ``a'' can be identified by
\(v\) or \(\ell^{-1}(a)\).  In practice, the presence of a labeling function is assumed for a labeled graph and so
a vertex is freely identified by its label.  This is important to note when a proof includes a phrase such as,
``let \(v\in V(G)\ldots\)'' since \(v\) may be a reference to any vertex in \(V(G)\) or may call out a specific
vertex by its label; the intention is usually clear from the context.

The design graphs that act as the inputs to the proposed algorithm are labeled graphs, where the labels represent
the various functional requirements: \(\FR_1,\FR_2,\FR_3,\ldots,\FR_n\).

\subsubsection{Vertex Color}\label{sec:sub:sub:color}

Other graph theory problems require that a graph's vertices be distributed into some number of sets based on some
problem-specific criteria.  Usually, this distribution is a true partition (no empty sets), but this is not
required depending on the problem.  One popular method of performing this distribution on a graph \(G\) is by using
a \emph{coloring} function \(c:V(G)\to C\), where \(C\) is a set of \emph{colors}.  Vertices with the same color
are assigned to the same set in the distribution.  Although the elements of \(C\) are usually actual colors
(\txtclr{red}, \txtclr{green}, \txtclr{blue}, etc.), a graph coloring problem is free to select any value type for
the color attribute.  Note that there is no assumption that \(c\) is surjective, so the codomain \(C\) may contain
unused colors, which correspond to empty sets in the distribution.

A coloring \(c:V(G)\to C\) on a graph \(G\) is called \emph{proper} when no two adjacent vertices in \(G\) are
assigned the same color: for all \(u,v\in V(G)\), if \(uv\in E(G)\) then \(c(u)\ne c(v)\).  Otherwise, \(c\) is
called \emph{improper}.  A proper coloring with \(\abs{C}=k\) is called a \emph{\coloring{k}} of \(G\) and \(G\) is
said to be \emph{\colorable{k}}, meaning the actual coloring (range of \(c\)) uses \emph{at most} \(k\) colors.

An example of a \coloring{4} is shown in \figurename~\ref{fig:exproper}.

\begin{figure}[H]
  \begin{minipage}{2.75in}
    \centering
    \begin{tikzpicture}
      \colorlet{c1}{green!50!white}
      \colorlet{c2}{blue!50!white}
      \colorlet{c3}{red!50!white}
      \colorlet{c4}{orange!50!white}
      \begin{scope}[node distance=2cm,every node/.style={labeled node}]
        \node (E) [fill=c3] at (0,0) {\(e\)};
        \node (A) [above left=of E,fill=c1] {\(a\)};
        \node (B) [above right=of E,fill=c2] {\(b\)};
        \node (C) [below right=of E,fill=c1] {\(c\)};
        \node (D) [below left=of E,fill=c4] {\(d\)};
      \end{scope}
      \draw (A) edge (B);
      \draw (B) edge (E);
      \draw (E) edge (A);
      \draw (A) edge (D);
      \draw (E) edge (C);
    \end{tikzpicture}
  \end{minipage}
  \begin{minipage}{2.75in}
    \centering
    \(C=\set{\mtxtclr{green},\mtxtclr{blue},\mtxtclr{red},\mtxtclr{orange}}\)
    \begin{align*}
      c(a) &= \mtxtclr{green} \\
      c(b) &= \mtxtclr{blue} \\
      c(c) &= \mtxtclr{green} \\
      c(d) &= \mtxtclr{orange} \\
      c(e) &= \mtxtclr{red}
    \end{align*}
  \end{minipage}
  \caption{A graph with a \coloring{4}.}
  \label{fig:exproper}
\end{figure}

Since there is no requirement that a coloring \(c\) be surjective, the codomain \(C\) may contain unused colors.
For example, the coloring shown in \figurename~\ref{fig:exproper} is surjective, but we can add an unused color to
\(C\):
\[C=\set{\mtxtclr{green},\mtxtclr{blue},\mtxtclr{red},\mtxtclr{orange},\mtxtclr{brown}}\]
Now, \(c\) is no longer surjective, and according to the definition, \(G\) is \colorable{5} --- the coloring \(c\)
uses at most 5 colors (actually only 4), which is the cardinality of the codomain.  This fact is generalized by
\theoremname~\ref{thm:coloring}.

\begin{theorem}
  \label{thm:coloring}
  Let \(G\) be a graph and let \(r\in\N\).  If \(G\) is \colorable{k} then \(G\) is \colorable{(k+r)}.
\end{theorem}

\begin{proof}
  Although this conclusion is fairly intuitive, it is always best to construct a proper coloring function under the
  given conditions so that the result is based on the definition.  So start by assuming that \(G\) is of order
  \(n\) and is \colorable{k}.  This means that there exists a coloring function \(c:V(G)\to C\) that is proper with
  \(\abs{C}=k\).  Let \(V(G)=\set{v_1,\ldots,v_n}\) and let \(C=\set{c_1,\cdots,c_k}\).  Now, let
  \(C'=\set{c_1,\cdots,c_{k+r}}\) and define \(c':V(G)\to C'\) by:
  \[c'(v)=c(v)\]
  Assume that \(u\) and \(v\) are two nonadjacent vertices in \(G\): \(uv\notin E(G)\).  Since \(c\) is proper:
  \[c'(u)=c(u)\ne c(v)=c'(v)\]
  and so \(c'\) is proper with \(\abs{C'}=k+r\).

  Therefore, \(G\) is \colorable{k+r}.
\end{proof}

Furthermore, for a graph \(G\) of order \(n\), if \(n\le k\) then we can conclude that \(G\) is \colorable{k},
since there are sufficient colors to assign each vertex its own unique color.  This result is stated in
\theoremname~\ref{thm:nlek}, which will turn out to be an important termination case for the proposed algorithm.

\begin{theorem}
  \label{thm:nlek}
  Let \(G\) be a graph of order \(n\) and let \(k\in\N\).  If \(n\le k\) then \(G\) is \colorable{k}.
\end{theorem}

\begin{proof}
  Assume \(n\le k\).  Let \(V(G)=\set{v_1,\ldots,v_n}\) and let \(C=\set{c_1,\ldots,c_k}\).  Now, define
  \(c:V(G)\to C\) by:
  \[c(v_i)=c_i\]
  which is possible since, by assumption, \(n\le k\).  Finally, assume that \(v_i\) and \(v_j\) are two nonadjacent
  vertices in \(G\): \(v_iv_j\notin E(G)\).  Since the \(c_i\) are distinct:
  \[c(v_i)=c_i\ne c_j=c(v_j)\]
  and so \(c\) is proper with \(\abs{C}=k\).

  Therefore, \(G\) is \colorable{k}.
\end{proof}

Since \(k\in\N\), by the well-ordering principle there exists some minimum \(k\) such that a graph \(G\) is
\colorable{k}.  This minimum \(k\) is called the \emph{chromatic number} of \(G\), denoted by \(\X(G)\).  A
\coloring{k} for a graph \(G\) where \(k=\X(G)\) is called a \emph{\chromatic{k}} coloring of \(G\).

Returning to the example \coloring{4} of \figurename~\ref{fig:exproper}, note that vertex \(d\) can be colored blue
and then orange can be excluded from the codomain, resulting in a \coloring{3}.  This is shown in
\figurename~\ref{fig:exchromatic}.  Since there is no way to use less than 3 colors to obtain a proper coloring of
the graph, the coloring is \chromatic{3}.  Note that when a coloring is chromatic, there are no unused colors
(empty sets) and hence the distribution is a true partition.

\begin{figure}[H]
  \begin{minipage}{2.75in}
    \centering
    \begin{tikzpicture}
      \colorlet{c1}{green!50!white}
      \colorlet{c2}{blue!50!white}
      \colorlet{c3}{red!50!white}
      \begin{scope}[node distance=2cm,every node/.style={labeled node}]
        \node (E) [fill=c3] at (0,0) {\(e\)};
        \node (A) [above left=of E,fill=c1] {\(a\)};
        \node (B) [above right=of E,fill=c2] {\(b\)};
        \node (C) [below right=of E,fill=c1] {\(c\)};
        \node (D) [below left=of E,fill=c2] {\(d\)};
      \end{scope}
      \draw (A) edge (B);
      \draw (B) edge (E);
      \draw (E) edge (A);
      \draw (A) edge (D);
      \draw (E) edge (C);
    \end{tikzpicture}
  \end{minipage}
  \begin{minipage}{2.75in}
    \[C=\set{\mtxtclr{green},\mtxtclr{blue},\mtxtclr{red}}\]
    \begin{align*}
      c(a) &= \mtxtclr{green} \\
      c(b) &= \mtxtclr{blue} \\
      c(c) &= \mtxtclr{green} \\
      c(d) &= \mtxtclr{blue} \\
      c(e) &= \mtxtclr{red}
    \end{align*}
  \end{minipage}
  \caption{A Graph with a \chromatic{3} coloring.}
  \label{fig:exchromatic}
\end{figure}

\subsection{Subgraphs}\label{sec:sub:subgraphs}

The basic strategy of the proposed algorithm is to arrive at a solution by mutating an input graph into simpler
graphs such that a solution is more easily determined.  The algorithm utilizes three particular mutators: vertex
deletion, edge addition, and vertex contraction.  Before describing these mutators, it will be helpful to describe
what is meant by graph equality and a \emph{subgraph} of a graph.

To say that graph \(G\) is \emph{equal} to graph \(H\), denoted by \(G=H\), means that the \emph{exact same graph}
is given two names: \(G\) and \(H\).  It is specifically \emph{not} a comparison between two different graphs.  Two
different graphs that have the same structure, meaning there exists an adjacency-preserving bijection between the
vertices of the two graphs, are referred to as being \emph{isomorphic}, denoted by \(G\cong H\), and are not
considered to be equal.  Of course, if \(G=H\) then \(G\cong H\); however, the converse is usually not true.  In
fact, \(G=H\) if and only if \(V(G)=V(H)\) and \(E(G)=E(H)\).

To say that \(H\) is a \emph{subgraph} of a graph \(G\), denoted by \(H\subseteq G\), means that \(V(H)\subseteq
V(G)\) and \(E(H)\subseteq E(G)\).  Thus, \(H\) can be achieved by removing zero or more vertices and/or edges from
\(G\), and \(G\) can be achieved by adding zero or more vertices and/or edges to \(H\).  Once again, \(H\) is not a
different graph.  If \(H\) is a different graph then one can say that it is isomorphic to a subgraph of \(G\), but
not a subgraph of \(G\) itself.  By definition, \(G\subseteq G\) and the null graph is a subgraph of every graph.

When \(G\) and \(H\) differ by at least one vertex or edge then \(H\) is called a \emph{proper subgraph} of \(G\),
denoted by \(H\subset G\).  In fact, \(H\subset G\) if and only if \(H\subseteq G\) but \(H\ne G\), meaning
\(V(H)\subset V(G)\) or \(E(H)\subset E(G)\).  When \(H\) and \(G\) differ by edges only: \(V(H)=V(G)\) and
\(E(H)\subseteq E(G)\), then \(H\) is called a \emph{spanning subgraph} of \(G\).

The concept of subgraphs is demonstrated by graphs \(G\), \(H\), and \(F\) in \figurename~\ref{fig:subgraphs}.
\(H\) is a proper subgraph of \(G\) by removing vertices \(c\) and \(d\) and edges \(ad\) and \(be\).  \(F\) is a
proper spanning subgraph of \(G\) because \(F\) contains all of the vertices in \(G\) but is missing edges \(ab\)
and \(be\).

\begin{figure}[H]
  \begin{minipage}{1.75in}
    \centering
    \begin{tikzpicture}[node distance=1cm,every node/.style={labeled node}]
      \node (E) at (0,0) {\(e\)};
      \node (A) [above left=of E] {\(a\)};
      \node (B) [above right=of E] {\(b\)};
      \node (C) [below right=of E] {\(c\)};
      \node (D) [below left=of E] {\(d\)};
      \draw (A) edge (B);
      \draw (B) edge (E);
      \draw (E) edge (A);
      \draw (A) edge (D);
    \end{tikzpicture}

    \(G\)
  \end{minipage}
  \begin{minipage}{1.75in}
    \centering
    \begin{tikzpicture}[node distance=1cm,every node/.style={labeled node}]
      \node (E) at (0,0) {\(e\)};
      \node (A) [above left=of E] {\(a\)};
      \node (B) [above right=of E] {\(b\)};
      \node (C) [below right=of E,color=white] {};
      \node (D) [below left=of E,color=white] {};
      \draw (A) edge (B);
      \draw (E) edge (A);
    \end{tikzpicture}

    \(H\subset G\) (proper)
  \end{minipage}
  \begin{minipage}{1.75in}
    \centering
    \begin{tikzpicture}[node distance=1cm,every node/.style={labeled node}]
      \node (E) at (0,0) {\(e\)};
      \node (A) [above left=of E] {\(a\)};
      \node (B) [above right=of E] {\(b\)};
      \node (C) [below right=of E] {\(c\)};
      \node (D) [below left=of E] {\(d\)};
      \draw (E) edge (A);
      \draw (A) edge (D);
    \end{tikzpicture}

    \(F\subset G\) (spanning)
  \end{minipage}
  \caption{Subgraph examples.}
  \label{fig:subgraphs}
\end{figure}

An \emph{induced} subgraph is a special type of subgraph.  Let \(G\) be a graph and let \(S\subseteq V(G)\).  The
subgraph of \(G\) \emph{induced} by \(S\), denoted by \(G[S]\), is a subgraph \(H\) such that \(V(H)=S\) and for
every \(u,v\in S\), if \(u\) and \(v\) are adjacent in \(G\) then they are also adjacent in \(H\).  Such a subgraph
\(H\) is called an \emph{induced subgraph} of \(G\).

In the examples of \figurename~\ref{fig:subgraphs}, \(H\) is not an induced subgraph of \(G\) because it is missing
edge \(be\).  Likewise, a proper spanning subgraph like \(F\) can never be induced due to missing edges.  In fact,
the only induced spanning subgraph of a graph is the graph itself.  \figurename~\ref{fig:induced} adds edge \(be\)
so that \(H\) is now an induced subgraph of \(G\).

\begin{figure}[H]
  \begin{minipage}{2.75in}
    \centering
    \begin{tikzpicture}[node distance=1cm,every node/.style={labeled node}]
      \node (E) at (0,0) {\(e\)};
      \node (A) [above left=of E] {\(a\)};
      \node (B) [above right=of E] {\(b\)};
      \node (C) [below right=of E] {\(c\)};
      \node (D) [below left=of E] {\(d\)};
      \draw (A) edge (B);
      \draw (B) edge (E);
      \draw (E) edge (A);
      \draw (A) edge (D);
    \end{tikzpicture}

    \(G\)
  \end{minipage}
  \begin{minipage}{2.75in}
    \centering
    \begin{tikzpicture}[node distance=1cm,every node/.style={labeled node}]
      \node (E) at (0,0) {\(e\)};
      \node (A) [above left=of E] {\(a\)};
      \node (B) [above right=of E] {\(b\)};
      \node (C) [below right=of E,color=white] {};
      \node (D) [below left=of E,color=white] {};
      \draw (A) edge (B);
      \draw (B) edge (E);
      \draw (E) edge (A);
    \end{tikzpicture}

    \(H=G[\set{a,b,e}]\)
  \end{minipage}
  \caption{An induced subgraph example.}
  \label{fig:induced}
\end{figure}

\subsection{Mutators}\label{sec:sub:mutators}

The following sections describe the graph mutators used by the proposed algorithm.

\subsubsection{Vertex Removal}\label{sec:sub:sub:vremove}

Let \(G\) be a graph and let \(S\subseteq V(G)\).  The induced subgraph obtained by removing all of the vertices in
\(S\) (and their incident edges) is denoted by:
\[G-S=G[V(G)-S]\]
If \(S\ne\emptyset\) then \(G-S\) is a proper subgraph of \(G\).  If \(S=V(G)\) then the result is the null graph.

\figurename~\ref{fig:vremove} shows an example of vertex removal: vertices \(c\) and \(e\) are removed, along with
their incident edges \(ae\) and \(be\).

\begin{figure}[H]
  \begin{minipage}{2.75in}
    \centering
    \begin{tikzpicture}[node distance=1cm,every node/.style={labeled node}]
      \node [red] (E) at (0,0) {\(e\)};
      \node (A) [above left=of E] {\(a\)};
      \node (B) [above right=of E] {\(b\)};
      \node [red] (C) [below right=of E] {\(c\)};
      \node (D) [below left=of E] {\(d\)};
      \draw (A) edge (B);
      \draw [red] (B) edge (E);
      \draw [red] (E) edge (A);
      \draw (A) edge (D);
    \end{tikzpicture}

    \(G\)
  \end{minipage}
  \begin{minipage}{3in}
    \centering
    \begin{tikzpicture}[node distance=1cm,every node/.style={labeled node}]
      \node (A) [above left=of E] {\(a\)};
      \node (B) [above right=of E] {\(b\)};
      \node (C) [below right=of E,white] {\(c\)};
      \node (D) [below left=of E] {\(d\)};
      \draw (A) edge (B);
      \draw (A) edge (D);
    \end{tikzpicture}

    \(G-\set{c,e}\)
  \end{minipage}
  \caption{A vertex removal example.}
  \label{fig:vremove}
\end{figure}

If \(S\) consists of a single vertex \(v\) then the alternate syntax \(G-v\) is used instead of \(G-\set{v}\).

The proposed algorithm uses vertex removal to simplify a graph that is assumed to be \colorable{k} into a smaller
graph that is also \colorable{k}.

\subsubsection{Edge Addition}\label{sec:sub:sub:eadd}

Let \(G\) be a graph and let \(u,v\in V(G)\) such that \(uv\notin E(G)\).  The graph \(G+uv\) is the graph with the
same vertices as \(G\) and with edge set \(E(G)\cup\set{uv}\).  Note that \(G\) is a proper spanning subgraph of
\(G+uv\).

\figurename~\ref{fig:eadd} shows an example of edge addition: edge \(cd\) is added.

\begin{figure}[H]
  \begin{minipage}{2.75in}
    \centering
    \begin{tikzpicture}[node distance=1cm,every node/.style={labeled node}]
      \node (E) at (0,0) {\(e\)};
      \node (A) [above left=of E] {\(a\)};
      \node (B) [above right=of E] {\(b\)};
      \node (C) [below right=of E] {\(c\)};
      \node (D) [below left=of E] {\(d\)};
      \draw (A) edge (B);
      \draw (B) edge (E);
      \draw (E) edge (A);
      \draw (A) edge (D);
    \end{tikzpicture}

    \(G\)
  \end{minipage}
  \begin{minipage}{2.75in}
    \centering
    \begin{tikzpicture}[node distance=1cm,every node/.style={labeled node}]
      \node (E) at (0,0) {\(e\)};
      \node (A) [above left=of E] {\(a\)};
      \node (B) [above right=of E] {\(b\)};
      \node (C) [below right=of E] {\(c\)};
      \node (D) [below left=of E] {\(d\)};
      \draw (A) edge (B);
      \draw (B) edge (E);
      \draw (E) edge (A);
      \draw (A) edge (D);
      \draw [green] (C) edge (D);
    \end{tikzpicture}

    \(G+cd\)
  \end{minipage}
  \caption{An edge addition example.}
  \label{fig:eadd}
\end{figure}

The proposed algorithm uses edge addition to prevent two non-adjacent FRs from being consolidated into the same
part.

\subsubsection{Edge Removal}\label{sec:sub:sub:eremove}

The proposed algorithm does not use edge removal; however, a number of related algorithms do rely on this mutator
so it is presented here.  Let \(G\) be a graph and let \(X\subseteq E(G)\).  The spanning subgraph obtained by
removing all of the edges in \(X\) is denoted by:
\[G-X=H\left(V(G),E(G)-X\right)\]
Thus, only edges are removed --- no vertices are removed.  If \(X\ne\emptyset\) then \(G-X\) is a proper subgraph of
\(G\).  If \(X=E(G)\) then the result is an empty graph.

\figurename~\ref{fig:eremove} shows an example of edge removal: edges \(ae\) and \(be\) are removed.

\begin{figure}[H]
  \begin{minipage}{2.75in}
    \centering
    \begin{tikzpicture}[node distance=1cm,every node/.style={labeled node}]
      \node (E) at (0,0) {\(e\)};
      \node (A) [above left=of E] {\(a\)};
      \node (B) [above right=of E] {\(b\)};
      \node (C) [below right=of E] {\(c\)};
      \node (D) [below left=of E] {\(d\)};
      \draw (A) edge (B);
      \draw [red] (B) edge (E);
      \draw [red] (E) edge (A);
      \draw (A) edge (D);
    \end{tikzpicture}

    \(G\)
  \end{minipage}
  \begin{minipage}{2.75in}
    \centering
    \begin{tikzpicture}[node distance=1cm,every node/.style={labeled node}]
      \node (E) at (0,0) {\(e\)};
      \node (A) [above left=of E] {\(a\)};
      \node (B) [above right=of E] {\(b\)};
      \node (C) [below right=of E] {\(c\)};
      \node (D) [below left=of E] {\(d\)};
      \draw (A) edge (B);
      \draw (A) edge (D);
    \end{tikzpicture}

    \(G-\set{ae,be}\)
  \end{minipage}
  \caption{An Edge removal example.}
  \label{fig:eremove}
\end{figure}

If \(X\) consists of a single edge \(e\) then the alternate syntax \(G-e\) is used instead of \(G-\set{e}\).

\subsubsection{Vertex Contraction}\label{sec:sub:sub:contract}

Vertex contraction is a bit different because it does not involve subgraphs.  Let \(G\) be a graph and let \(u,v\in
V(G)\).  The graph \(G\cdot uv\) is constructed by identifying \(u\) and \(v\) as one vertex (i.e., merging them).
Any edge between the two vertices is discarded.  Any other edges that were incident to the two vertices become
incident to the new single vertex.  Note that this may require supression of multiple edges to preserve the nature
of a simple graph.

\figurename~\ref{fig:contract} shows an example of vertex contraction: vertices \(a\) and \(b\) are contracted into
a single vertex.  Since edges \(ae\) and \(be\) would result in multiple edges between \(a\) and \(e\), one of the
edges is discarded.  Edges \(bc\) and \(bd\) also become incident to the single vertex.

\begin{figure}[H]
  \begin{minipage}{2.75in}
    \centering
    \begin{tikzpicture}[every node/.style={labeled node}]
      \cycleVnodes{\(a\),\(b\),\(c\),\(d\),\(e\)}{(0,0)}{1in}{90}{}
      \draw (1) edge (5);
      \draw (2) edge (3) edge (4) edge (5);
      \draw (3) edge (4);
      \draw (4) edge (5);
      \draw [dashed,red,->] (2) edge (1);
    \end{tikzpicture}

    \(G\)
  \end{minipage}
  \begin{minipage}{2.75in}
    \centering
    \begin{tikzpicture}
      \begin{scope}[every node/.style={coordinate}]
        \cycleNnodes{5}{(0,0)}{1in}{90}{}
      \end{scope}
      \begin{scope}[every node/.style={labeled node}]
        \node (AB) at (1) {\(ab\)};
        \node (C) at (3) {\(c\)};
        \node (D) at (4) {\(d\)};
        \node (E) at (5) {\(e\)};
      \end{scope}
      \draw (AB) edge (C) edge (D) edge (E);
      \draw (C) edge (D);
      \draw (D) edge (E);
    \end{tikzpicture}

    \(G\cdot ab\)
  \end{minipage}
  \caption{A vertex contraction example.}
  \label{fig:contract}
\end{figure}

For the operation \(G\cdot uv\), if \(uv\in E(G)\) then the operation is also referred to as \emph{edge
  contraction}.  If \(uv\notin E(G)\) then the operation is also referred to as \emph{vertex identification}.  The
proposed algorithm uses vertex identification to consolidate two non-adjacent FRs into the same part.

\subsubsection{Graph Complement}\label{sec:sub:sub:complement}

One final important graph mutator is the \emph{complement} of a graph.  For a graph \(G\), the \emph{complement} of
\(G\), denoted by \(\bar{G}\), is the graph with the same vertex set as \(G\): \(V(G)=V(\bar{G})\), and with edge
set \(E(\bar{G})=\ps_2\left(V(G)\right)-E(G)\); if \(u,v\in V(G)\) are adjacent in \(G\) \(\left(uv\in
E(G)\right)\) then they are not adjacent in \(\bar{G}\) \(\left(uv\notin E(\bar{G})\right)\).

An example of a graph complement operation is shown in \figurename~\ref{fig:complement}.

\begin{figure}[H]
  \begin{minipage}{2.75in}
    \centering
    \begin{tikzpicture}[every node/.style={labeled node}]
      \cycleVnodes{\(a\),\(b\),\(c\),\(d\)}{(0,0)}{0.75in}{135}{};
      \draw (1) edge (2) edge (3) edge (4);
      \draw [red, dashed] (2) edge (3);
      \draw [red, dashed] (2) edge (4);
      \draw [red, dashed] (3) edge (4);
    \end{tikzpicture}

    \(G\)
  \end{minipage}
  \begin{minipage}{2.75in}
    \centering
    \begin{tikzpicture}[every node/.style={labeled node}]
      \cycleVnodes{\(a\),\(b\),\(c\),\(d\)}{(0,0)}{0.75in}{135}{};
      \draw [red, dashed] (1) edge (2) edge (3) edge (4);
      \draw (2) edge (3);
      \draw (2) edge (4);
      \draw (3) edge (4);
    \end{tikzpicture}

    \(\bar{G}\)
  \end{minipage}
  \caption{A graph complement example.}
  \label{fig:complement}
\end{figure}

Some important properties of the complement of a graph are stated in \propname~\ref{prop:compprops}.

\begin{proposition}
  \label{prop:compprops}
  Let \(G\) be a graph of order \(n\) and size \(m\):
  \begin{enumerate}
  \item \(\bar{\bar{G}}=G\)
  \item \(G\) is empty if and only if \(\bar{G}\) is complete.
  \item \(n(\bar{G})=n(G)\)
  \item \(m(\bar{G})=\frac{n(n-1)}{2}-m(G)\)
  \end{enumerate}
\end{proposition}

For example, in \figurename~\ref{fig:complement}, since \(n(G)=4\) and \(m(G)=3\), it is the case that
\(n(\bar{G})=4\) and:
\[m(\bar{G})=\frac{4(4-1)}{2}-3=6-3=3\]

\subsection{Independent Sets}\label{sec:sub:independent}

The primary purpose of a \coloring{k} of a graph \(G\) is to distribute the vertices of \(G\) into \(k\) so-called
\emph{independent} (some possibly empty) sets.  For a graph \(G\), an \emph{independent} set \(S\subseteq V(G)\),
sometimes called a \emph{stable} set, is a set of pairwise non-adjacent vertices in \(G\): for all \(u,v\in S\),
\(uv\notin E(G)\).  By definition, the empty set is an independent set of every graph \(G\).  A \emph{maximal}
independent set of a graph \(G\), sometimes referred to as a \emph{MIS} of \(G\), is an independent set of \(G\)
that cannot be extended by an additional vertex in \(V(G)\); MISs of \(G\) are never proper subsets of other
independent sets in \(G\).  The cardinality of the largest possible MIS in a graph \(G\), denoted by \(\a(G)\), is
called the \emph{independence number} for \(G\).

Consider the example in \figurename~\ref{fig:independent}.  Although the graph contains independent sets of sizes
\(1\) and \(2\), none of these are maximal.  All of the maximal independent sets are of sizes \(3\) and \(4\), and
so \(\a(G)=4\).

\begin{figure}[H]
  \begin{minipage}{3.25in}
    \centering
    \begin{tikzpicture}[every node/.style={labeled node}]
      \node (a) at (0,0) {\(a\)};
      \node (b) at (2,0) {\(b\)};
      \node (c) at (4,0) {\(c\)};
      \node (d) at (6,0) {\(d\)};
      \node (e) at (5,-2) {\(e\)};
      \node (f) at (3,-2) {\(f\)};
      \node (g) at (1,-2) {\(g\)};
      \node (h) at (3,2) {\(h\)};
      \draw (a) edge (f) edge (g) edge (h);
      \draw (b) edge (f) edge (h);
      \draw (c) edge (e) edge (f) edge (h);
      \draw (d) edge (e) edge (f) edge (h);
    \end{tikzpicture}
  \end{minipage}
  \begin{minipage}{2in}
    \centering
    \begin{tabular}{c|c}
      MIS & SIZE \\
      \hline
      \(\set{a, b, c, d}\) & \(4\) \\
      \(\set{a, b, e}\) & \(3\) \\
      \(\set{b, c, d, g}\) & \(4\) \\
      \(\set{b, e, g}\) & \(3\) \\
      \(\set{e, f, g, h}\) & \(4\)
    \end{tabular}

    \bigskip

    \(\a(G)=4\)
  \end{minipage}
  \caption{The independent sets of an example graph.}
  \label{fig:independent}
\end{figure}

Since a \chromatic{k} coloring of a graph \(G\) is surjective, there are no unused colors (empty sets) and so the
coloring partitions the vertices of \(G\) into exactly \(k\) non-empty independent sets.  The goal of the proposed
algorithm is to find a chromatic coloring of a design graph so that the resulting independent sets indicate how to
consolidate the FRs into a minimum number of parts: one part per independent set.

\subsection{Cliques}\label{sec:sub:cliques}

A \emph{clique} is a complete subgraph of a graph.  A clique of order \(k\) in a graph \(G\) is called a \clique{k}
of \(G\).  A \emph{maximal} clique in a graph \(G\) is a clique in \(G\) that cannot be extended by an additional
vertex in \(V(G)\); maximal cliques in \(G\) are never proper subgraphs of other cliques in \(G\).  The order of
the largest possible maximal clique in a graph \(G\), denoted by \(\w(G)\), is called the \emph{clique number} for
\(G\).

Consider the example in \figurename~\ref{fig:clique}.  Although the graph contains cliques of orders \(1\) and
\(2\), none of these are maximal.  All of the maximal cliques are of orders \(3\) and \(4\), and so \(\w(G)=4\).

\begin{figure}[H]
  \begin{minipage}{3.25in}
    \centering
    \scalebox{0.75}{
      \begin{tikzpicture}[every node/.style=labeled node]
        \completeV{\(a\),\(b\),\(c\),\(d\)}{(0,0)}{0.75in}{135}{a};
        \completeV{\(e\),\(f\),\(g\)}{(5,0)}{0.6in}{120}{b};
        \node (h) [right=of b2] {\(h\)};
        \draw (a2) edge (b1);
        \draw (a3) edge (b3);
        \draw (b2) edge (h);
        \draw (a1) -- ($(a1)+(0,1cm)$) -| (b1);
        \draw (a4) -- ($(a4)-(0,1cm)$) -| (b3);
        \draw (a2) -- (b3);
        \draw (b1) -| (h);
        \draw (b3) -| (h);
      \end{tikzpicture}
    }

    \(G\)
  \end{minipage}
  \begin{minipage}{2.5in}
    \centering
    \begin{tabular}{c|c}
      MAXIMAL CLIQUE & ORDER \\
      \hline
      \(G[\set{a, b, c, d}]\) & \(4\) \\
      \(G[\set{a, b, e}]\) & \(3\) \\
      \(G[\set{b, c, d, g}]\) & \(4\) \\
      \(G[\set{b, e, g}]\) & \(3\) \\
      \(G[\set{e, f, g, h}]\) & \(4\)
    \end{tabular}

    \bigskip

    \(\w(G)=4\)
  \end{minipage}
  \caption{The maximal cliques of an example graph.}
  \label{fig:clique}
\end{figure}

Since it is true that two nonadjacent vertices in a graph must be adjacent in the graph's complement, there is an
important relationship between the independent sets of a graph and the cliques in the graph's complement.  This
relationship is stated in \theoremname~\ref{thm:isclique}.

\begin{theorem}
  \label{thm:isclique}
  Let \(G\) be a graph and let \(S\subseteq V(G)\).  \(S\) is an independent set in \(G\) if and only if
  \(\bar{G}[S]\) is a clique in \(\bar{G}\).  Furthermore, \(S\) is maximal in \(G\) if and only if \(\bar{G}[S]\)
  is maximal in \(\bar{G}\) and so \(\a(G)=\w(\bar{G})\).
\end{theorem}

\begin{proof}
  By definition, for all \(u,v\in V(G)\), \(u\) is not adjacent to \(v\) in \(G\) if and only if \(u\) and \(v\)
  are adjacent in \(\bar{G}\), and so \(G[S]\) is empty if and only if \(\bar{G}[S]\) is complete.  Therefore,
  \(S\) is an independent set in \(G\) if and only if \(\bar{G}[S]\) is a clique in \(\bar{G}\).

  Furthermore, assume that \(S\) is maximal in \(G\) but assume by way of contradiction that \(\bar{G}[S]\) is
  not maximal.  Then there exists \(v\in V(\bar{G})\) such that \(v\notin S\) and \(\bar{G}[S\cup\set{v}]\) is a
  clique in \(\bar{G}\), and thus \(S\cup\set{v}\) is an independent set in \(G\).  But \(S\subset S\cup\set{v}\),
  violating the maximality of \(S\).  Therefore \(\bar{G}[S]\) is maximal in \(\bar{G}\).

  Similarly, assume that \(\bar{G}[S]\) is maximal in \(\bar{G}\) but assume by way of contradiction that \(S\) is
  not maximal in \(G\).  Then there exists \(v\in V(G)\) such that \(v\notin S\) and \(S\cup\set{v}\) is an
  independent set in \(G\), and thus \(\bar{G}[S\cup\set{v}]\) is a clique in \(\bar{G}\).  But
  \(\bar{G}[S]\subset\bar{G}[S\cup\set{v}]\), violating the maximality of \(\bar{G}[S]\).  Therefore \(S\) is
  maximal in \(G\).
\end{proof}

Indeed, the graphs in \figurename~\ref{fig:independent} and \figurename~\ref{fig:clique} are complements and,
as expected, every MIS in \figurename~\ref{fig:independent} is a maximal clique in \figurename~\ref{fig:clique}.

Since a \clique{k} of a graph \(G\) needs at least \(k\) colors in any proper coloring of \(G\), the clique number
of \(G\) provides a nice lower bound for the chromatic number of \(G\).  Unfortunately, the clique number problem
is known to be inherently intractable as well~\cite{corneil}.  Thus, there are many attempts in the literature to
find a good lower bound for the clique number of a graph \(G\), usually denoted by \(\w'(G)\).  If such a lower
bound is known then the conclusion of \propname~\ref{prop:clique} holds:

\begin{proposition}
  \label{prop:clique}
  Let \(G\) be a graph with clique number lower bound \(\w'(G)\):
  \[\w'(G)\le\w(G)\le\X(G)\]
\end{proposition}

\subsection{Connected Graphs}\label{sec:sub:connected}

The edges of a graph suggest the ability to ``walk'' from one vertex to another along the edges.  A graph where this
is possible for any two vertices is called a \emph{connected} graph.  The concept of connectedness is an
important topic in graph theory; however, an ideal coloring algorithm should work regardless of the connected
nature of an input graph.  The concept of connectedness and how it impacts coloring is described in this section.

\subsubsection{Walks}\label{sec:sub:sub:walks}

The undirected edges in a simple graph suggest bidirectional connectivity between their endpoint vertices.  This
leads to the idea of ``traveling'' between two vertices in a graph by following the edges joining intermediate
adjacent vertices.  Such a journey is referred to as a \emph{walk}.

A \(u-v\) \emph{walk} \(W\) in a graph \(G\) is a finite sequence of vertices \(w_i\in V(G)\) starting with
\(u=w_0\) and ending with \(v=w_k\):
\[W=(u=w_0,w_1,\ldots,w_k=v)\]
such that \(w_iw_{i+1}\in E(G)\) for \(0\le i<k\).  To say that \(W\) is \emph{open} means that \(u\ne v\).  To say
that \(W\) is \emph{closed} means that \(u=v\).  The \emph{length} \(k\) of \(W\) is the number of edges traversed:
\(k=\abs{W}\).  A \emph{trivial} walk is a walk of zero length --- i.e, a single vertex: \(W=(u)\).

The bidirectional nature of the edges in a simple graph suggests the following proposition:

\begin{proposition}
  Let \(G\) be a graph and let \(u-v\) be a walk of length \(k\) in \(G\).  \(G\) contains a \(v-u\) walk of length
  \(k\) in \(G\) by traversing \(u-v\) in the opposite direction.
\end{proposition}

An example of two walks of length \(4\) is shown in \figurename~\ref{fig:walks}.  \(W_1\) is an open walk because
it starts and ends on distinct vertices, whereas \(W_2\) is a closed walk because it starts and ends on the same
vertex.

\begin{figure}[H]
  \begin{minipage}{2.75in}
    \centering
    \begin{tikzpicture}[every node/.style={labeled node}]
      \cycleVnodes{\(a\),\(b\),\(c\),\(d\),\(e\)}{(0,0)}{0.75in}{90}{}
      \draw (1) edge (2) edge (3) edge (4) edge (5);
      \draw (2) edge (3) edge (5);
      \draw (3) edge (4);
    \end{tikzpicture}
  \end{minipage}
  \begin{minipage}{2.75in}
    \(W_1=(a,b,e,a,c)\ \text{is open}\)

    \(W_2=(a,e,b,c,a)\ \text{is closed}\)

    \bigskip

    \(\abs{W_1}=\abs{W_2}=4\)
  \end{minipage}
  \caption{Open and closed walks.}
  \label{fig:walks}
\end{figure}

Vertices and edges are allowed to be repeated during a walk.  Certain special walks can be defined by restricting
such repeats:

\begin{center}
  \begin{tabular}{lll}
    \emph{trail} & An open walk with no repeating edges & \((a,b,c,a,e)\) \\
    \\
    \emph{path} & A trail with no repeating vertices & \((a,e,b,c)\) \\
    \\
    \emph{circuit} & A closed trail & \((a,b,e,a,c,d,a)\) \\
    \\
    \emph{cycle} & A closed path & \((a,e,b,c,a)\)
  \end{tabular}
\end{center}

The example special walks stated above refer to the graph in \figurename~\ref{fig:walks}.

\subsubsection{Paths}\label{sec:sub:sub:walkpath}

When discussing the connectedness of a graph, the main concern is the existence of paths between vertices.  Let
\(G\) be a graph and let \(u,v\in V(G)\).  To say that \(u\) and \(v\) are \emph{connected} means that \(G\)
contains a \(u-v\) path.

But if there exists a \(u-v\) walk in a graph \(G\), does this also mean that there exists a \(u-v\) path in \(G\)
--- i.e., a walk with no repeating edges or vertices?  The answer is yes, as shown by the following theorem:

\begin{theorem}
  Let \(G\) be a graph and let \(u,v\in V(G)\).  If \(G\) contains a \(u-v\) walk of length \(k\) then \(G\)
  contains a \(u-v\) path of length \(\ell\le k\).
\end{theorem}

\begin{proof}
  Assume that \(G\) contains at least one \(u-v\) walk of length \(k\) and consider the set of all possible \(u-v\)
  walks in \(G\); their lengths form a non-empty set of positive integers.  By the well-ordering principle, there
  exists a \(u-v\) walk \(P\) of minimum length \(\ell\le k\):
  \[P=(u=w_0,\ldots,w_{\ell}=v)\]
  We claim that \(P\) is a path.

  Assume by way of contradiction that \(P\) is not a path, and thus \(P\) has at least one repeating vertex.  Let
  \(w_i=w_j\) for some \(0\le i<j\le\ell\) be such a repeating vertex.  There are two possibilities:
  
  \begin{description}
  \item Case 1: The walk ends on a repeated vertex (\(j=\ell\)).  This is demonstrated in
    \figurename~\ref{fig:rend}.

    \begin{figure}[H]
      \centering
      \begin{tikzpicture}
        \colorlet{cin}{green}
        \begin{scope}[every node/.style={unlabeled node},node distance=1in]
          \node (w0) at (0,0) {};
          \node (w1) [right=of w0] {};
          \node (wi) [right=of w1] {};
          \node (wi1) [right=of wi] {};
          \node (wj2) [right=of wi1] {};
          \node (wj1) [below=of wi] {};
        \end{scope}
        \draw [cin] (w0) edge (w1);
        \draw [dashed,cin] (w1) edge (wi);
        \draw (wi) edge (wi1);
        \draw [dashed] (wi1) edge (wj2);
        \draw (wj2) edge (wj1);
        \draw (wj1) edge (wi);
        \node [above] at (w0) {\(u=w_0\)};
        \node [above] at (w1) {\(w_1\)};
        \node [above] at (wi) {\(w_i=w_j=w_{\ell}=v\)};
        \node [above] at (wi1) {\(w_{i+1}\)};
        \node [above] at (wj2) {\(w_{j-2}\)};
        \node [below] at (wj1) {\(w_{j-1}\)};
      \end{tikzpicture}
      \caption{Repeated vertex at end case.}
      \label{fig:rend}
    \end{figure}

    Let \(P'=(u=w_0,w_1,\ldots,w_i=v)\) be the walk shown in green in \figurename~\ref{fig:rend}.  \(P'\) is a
    \(u-v\) walk of length \(i<\ell\) in \(G\).

  \item Case 2: A repeated vertex occurs inside the walk (\(j<\ell\)).  This is demonstrated in
    \figurename~\ref{fig:rmiddle}.
    
    \begin{figure}[H]
      \centering
      \begin{tikzpicture}
        \colorlet{cin}{green}
        \begin{scope}[every node/.style={unlabeled node},node distance=1in]
          \node (w0) at (0,0) {};
          \node (w1) [right=of w0] {};
          \node (wi) [right=of w1] {};
          \node (wip1) [below right=of wi] {};
          \node (wjm1) [below left=of wi] {};
          \node (wjp1) [right=of wi] {};
          \node (wl) [right=of wjp1] {};
        \end{scope}
        \draw [cin] (w0) edge (w1);
        \draw [dotted,cin] (w1) edge (wi);
        \draw (wi) edge (wip1);
        \draw [dotted] (wip1) edge (wjm1);
        \draw (wjm1) edge (wi);
        \draw [cin] (wi) edge (wjp1);
        \draw [dotted,cin] (wjp1) edge (wl);
        \draw [dotted,cin] (wjp1) edge (wl);
        \node [above] at (w0) {\(u=w_0\)};
        \node [above] at (w1) {\(w_1\)};
        \node [above] at (wi) {\(w_i=w_j\)};
        \node [below] at (wip1) {\(w_{i+1}\)};
        \node [below] at (wjm1) {\(w_{j-1}\)};
        \node [above] at (wjp1) {\(w_{j+1}\)};
        \node [above] at (wl) {\(w_{\ell}=v\)};
      \end{tikzpicture}
      \caption{Repeated vertex inside case.}
      \label{fig:rmiddle}
    \end{figure}

  Let \(P'=(u=w_0,w_1,\ldots,w_i,w_{j+1},\ldots,w_{\ell}=v)\) be the walk shown in green in the
  \figurename~\ref{fig:rmiddle}.  \(P'\) is a \(u-v\) walk of length \(\ell-(j-i)<\ell\) in \(G\).
  \end{description}

  Both cases contradict the minimality of the length of \(P\).

  \(\therefore P\) is a \(u-v\) path of length \(\ell\le k\) in \(G\).
\end{proof}

\subsubsection{Connected}\label{sec:sub:sub:connected}

A \emph{connected} graph \(G\) is a graph whose vertices are all connected: for all \(u,v\in V(G)\) there exists a
\(u-v\) path.  Otherwise, \(G\) is said to be \emph{disconnected}.  Examples of connected and disconnected graphs
are shown in figure \ref{fig:connect}.

\begin{figure}[H]
  \begin{minipage}[t]{2.75in}
    \centering
    \begin{tikzpicture}[every node/.style={labeled node}]
      \cycleVnodes{\(a\),\(b\),\(c\),\(d\)}{(0,0)}{0.75in}{135}{}
      \draw (1) edge (2) edge (3) edge (4);
    \end{tikzpicture}

    \bigskip

    \begin{tabular}{c}
      \((a,b)\) \\
      \((a,c)\) \\
      \((a,d)\) \\
      \((b,a,c)\) \\
      \((b,a,d)\) \\
      \((c,a,d)\)
    \end{tabular}

    \bigskip
    
    CONNECTED
  \end{minipage}
  \begin{minipage}[t]{3in}
    \centering
    \begin{tikzpicture}[every node/.style={labeled node},node distance=2cm]
      \cycleVnodes{\(a\),\(b\),\(c\)}{(0,0)}{0.75in}{90}{l}
      \draw (l1) edge (l2) edge (l3);
      \draw (l2) edge (l3);
      \pathVnodes{\(d\),\(e\)}{(3,0.5)}{right}{r};
      \draw (r1) edge (r2);
    \end{tikzpicture}

    \bigskip
    
    No path from any of \(a,b,c\) to any of \(d,e\)

    \bigskip

    DISCONNECTED
  \end{minipage}
  \caption{Connected and disconnected graphs.}
  \label{fig:connect}
\end{figure}

By definition, the trivial graph is connected since the single vertex is connected to itself by a trivial path (of
length \(0\)).

\subsubsection{Components}\label{sec:sub:sub:components}

It would seem that a disconnected graph is composed of some number of connected subgraphs that partition the
graph's vertex set under a connected equivalence relation.  Each such subgraph is called a \emph{component} of the
graph.

Let \(G\) be a graph and let \(\SG\) be the set of all connected subgraphs of \(G\).  To say that a graph
\(H\in\SG\) is a \emph{component} of a \(G\) means that \(H\) is not a subgraph of any other connected subgraph of
\(\SG\): for every \(F\in\SG-\set{H}\) it is the case that \(H\not\subset F\).  The number of distinct components
in \(G\) is denoted by k(G), or just \(k\) if \(G\) is unambiguous.  For a connected graph: \(k(G)=1\).

Each component of a graph \(G\) is denoted by \(G_i\) where \(1\le i\le k(G)\).  We also use union notation to
denote that \(G\) is composed of its component parts:
\[G=\bigcup_{0\le i\le k(G)}G_i\]
Furthermore the \(G_i\) are induced by the vertex equivalence classes of the connectedness relation:

\begin{theorem}
  Let \(G\) be a graph with component \(G_i\).  \(G_i\) is an induced subgraph of \(G\).
\end{theorem}

\begin{proof}
  By definition, \(G_i\) is a maximal connected subgraph of \(G\).  So assume by way of contradiction that
  \(G_i\) is not an induced subgraph of \(G\).  Thus, \(G_i\) is missing some edges that when added would result in a
  connected induced subgraph \(H\) of \(G\).  But then \(G_i\subset H\), contradicting the maximality of \(G_i\).

  \(\therefore G_i\) is an induced subgraph of \(G\).
\end{proof}

\subsubsection{Impact on Coloring}\label{sec:sub:sub:impact}

The impact of disconnectedness on coloring depends on the selected algorithm.  One might assume that the selected
algorithm should be run on each component individually in order to determine each \(\X(G_i)\) and then, as pointed
out by Zykov (1949)~\cite{zykov}, conclude that the maximum such value is sufficient for \(\X(G)\):
\[\X(G)=\max_{1\le i\le k(G)}\X(G_i)\]
For example, consider the disconnected graph in \figurename~\ref{fig:connect}.  The graph contains two components,
so number the components from left-to-right:
\begin{gather*}
  \X(G_1)=3 \\
  \X(G_2)=2 \\
  \\
  \X(G)=\max\{3,2\}=3
\end{gather*}

Using this technique requires application of an initial algorithm to partition the graph into components.  Such an
algorithm is well-known and is described by Hopcroft and Tarjan (1973)~\cite{hopcroft}.  The algorithm is
recursive.  It starts by pushing a randomly selected vertex on the stack and walking the vertex's incident edges,
removing each edge as it is traversed.  As each unmarked vertex is encountered, it is assigned to the current
component.  Vertices with incident edges are pushed onto the stack and newly isolated vertices are popped off the
stack.  Once the stack is empty, any previously unmarked vertex is selected to start the next component and the
process continues until all vertices are marked.  Given a graph \(G\) of order \(n\) and size \(m\), this algorithm
runs in \(\max(n,m)\) steps.

Alternatively, an ideal coloring algorithm could be run on the entire graph at once regardless of the number of
components in the graph.  The proposed algorithm is such a solution, and therefore saves the needless work of
partitioning the graph into components first.

\subsection{Vertex Degree}\label{sec:sub:degree}

Besides a graph's order and size, the next most important parameter is the so-called \emph{degree} of each vertex.
In order to define the degree of a vertex, we need to define what is meant by a vertex's \emph{neighborhood} first.
Let \(G\) be a graph and let \(u\in V(G)\).  If \(v\in V(G)\) is adjacent to \(u\) then \(u\) and \(v\) are called
\emph{neighbors}.  Note that for simple graphs, a vertex is never a neighbor of itself.  The \emph{neighborhood} of
\(u\), denoted by \(N(u)\), is the set of all the neighbors of \(u\) in \(G\):
\[N(u)=\setb{v\in V(G)}{uv\in E(G)}\]
The \emph{degree} of \(u\), denoted by \(\deg_G(u)\) or just \(\deg(u)\) if \(G\) is unambiguous, is then defined
to be the cardinality of its neighborhood: \(\deg(u)=\abs{N(u)}\).  Thus, the degree of a vertex can be viewed as
the number of neighbor vertices or the number of incident edges.

When considering the degrees of all the vertices in a graph, the following limits are helpful:
\begin{gather*}
  \dmin(G)=\min_{v\in V(G)}\deg(v) \\
  \dmax(G)=\max_{v\in V(G)}\deg(v)
\end{gather*}

Therefore, we can state the conclusion of \propname~\ref{prop:degree}:

\begin{proposition}
  \label{prop:degree}
  Let \(G\) be a graph of order \(n\).  For every vertex \(v\in G\):
\[0\le\dmin(G)\le\deg(v)\le\dmax(G)\le n-1\]
\end{proposition}

Intuitively, as \(\dmin(G)\) increases, a graph becomes denser (more edges) resulting in more adjacencies, making it
harder to find a proper coloring at lower values of \(k\).

Vertices can be classified based on their degree, as shown in \tablename~\ref{tab:degree}.

\begin{table}[H]
  \centering
  \caption{Classifying Vertex \(u\) in Graph \(G\)}
  \label{tab:degree}
  \begin{tabular}{|c|l|}
    \hline
    \(\deg(u)\) & TYPE \\
    \hline
    \(0\) & isolated \\
    \(1\) & pendant, end, leaf \\
    \(n-1\) & universal \\
    even & even \\
    odd & odd \\
    \hline
  \end{tabular}
\end{table}

\emph{Isolated} vertices have degree \(0\); they are not adjacent to any other vertex in \(G\).  \emph{Pendant}
(also called \emph{end} or \emph{leaf}) vertices have degree \(1\); they are adjacent to exactly one other vertex
in \(G\).  \emph{Universal} vertices are adjacent to every other vertex in \(G\).  \emph{Even} vertices are adjacent
to an even number of vertices in \(G\) and \emph{odd} vertices are adjacent to an odd number of vertices in \(G\).
Note that if \(G\) has a universal vertex then it cannot have an isolated vertex, and vice-versa.

The degrees of the vertices in a graph and the number of edges in the graph are related by the so-called First
Theorem of Graph Theory:

\begin{theorem}[First Theorem of Graph Theory]
  \label{thm:first}
  Let \(G\) be a graph of size \(m\):
  \[\sum_{v\in V(G)}\deg(v)=2m\]
\end{theorem}

\begin{proof}
  When summing all the degrees, each edge is counted twice: once for each endpoint.
\end{proof}

These concepts are demonstrated by the graph in \figurename~\ref{fig:degree}.

\begin{figure}[H]
  \begin{minipage}{3.25in}
    \centering
    \begin{tikzpicture}[every node/.style={labeled node}]
      \cycleV{\(v_2\),\(v_3\),\(v_4\),\(v_5\),\(v_6\),\(v_7\)}{(0,0)}{1in}{0}{c};
      \node (v1) at (0,0) {\(v_1\)};
      \node (v8) at (150:1.25in) {\(v8\)};
      \foreach \i in {c1,c2,c3,c4,c5,c6}{
        \draw (v1) edge (\i);
      }
      \draw (c1) edge (c5);
      \draw (c2) edge (c4);
      \draw (v1) edge (v8);
    \end{tikzpicture}

    \bigskip

    \begin{tabular}{ll}
      \(n=8\) & \(m=15=\frac{30}{2}\) \\
      \(\dmin(G)=1\) & \(\dmax(G)=7\)
    \end{tabular}
  \end{minipage}
  \begin{minipage}{2.5in}
    \begin{tabular}{c|c|l}
      vertex & degree & type \\
      \hline
      \(v_1\) & 7 & universal,odd \\
      \(v_2\) & 4 & even \\
      \(v_3\) & 4 & even \\
      \(v_4\) & 3 & odd \\
      \(v_5\) & 4 & even \\
      \(v_6\) & 4 & even \\
      \(v_7\) & 3 & odd \\
      \(v_8\) & 1 & pendant,odd \\
      \hline
      total & 30 &
    \end{tabular}
  \end{minipage}
  \caption{Vertex degrees and the first theorem of graph theory.}
  \label{fig:degree}
\end{figure}

\subsection{Special Graphs}\label{sec:sub:special}

The following sections described some special classes of graphs that are important to the execution of the proposed
algorithm.

\subsubsection{Empty Graphs}\label{sec:sub:sub:empty}

An \emph{empty} graph of order \(n\), denoted by \(E_n\), is a graph with one or more vertices (\(n>1\)) and no
edges (\(m=0\)).  An empty graph is connected if and only if \(n=1\).  Examples of empty graphs are shown in
\figurename~\ref{fig:empty}.

\begin{figure}[H]
  \begin{minipage}{1in}
    \centering
    \begin{tikzpicture}[every node/.style=unlabeled node]
      \node at (0,0) {};
    \end{tikzpicture}

    \(E_1\)
  \end{minipage}
  \begin{minipage}{2.5in}
    \centering
    \begin{tikzpicture}[every node/.style=unlabeled node]
      \pathNnodes{4}{(0,0)}{right}{};
    \end{tikzpicture}

    \(E_4\)
  \end{minipage}
  \begin{minipage}{2in}
    \centering
    \begin{tikzpicture}[every node/.style=unlabeled node]
      \pathNnodes{3}{(0,0)}{right}{};
      \pathNnodes{3}{(0,1)}{right}{};
      \pathNnodes{3}{(0,2)}{right}{};
    \end{tikzpicture}

    \(E_9\)
  \end{minipage}
  \caption{Empty graphs.}
  \label{fig:empty}
\end{figure}

The null graph (\(n=0\)) is denoted by \(E_0\) and is defined to be \chromatic{0}.  All other empty graphs are
\chromatic{1} and thus are important termination conditions for the proposed algorithm.

\subsubsection{Paths}\label{sec:sub:sub:paths}

A \emph{path} graph of order \(n\) and length \(n-1\), denoted by \(P_n\), is a connected graph consisting of a
single open path.  Examples of path graphs are shown in \figurename~\ref{fig:path}.

\begin{figure}[H]
  \begin{minipage}{1in}
    \centering
    \begin{tikzpicture}[every node/.style=unlabeled node]
      \node at (0,0) {};
    \end{tikzpicture}

    \(P_1\)
  \end{minipage}
  \begin{minipage}{2.5in}
    \centering
    \begin{tikzpicture}[every node/.style=unlabeled node]
      \pathN{4}{(0,0)}{right}{};
    \end{tikzpicture}

    \(P_4\)
  \end{minipage}
  \begin{minipage}{2in}
    \centering
    \begin{tikzpicture}[every node/.style=unlabeled node]
      \pathNnodes{3}{(0,0)}{right}{b};
      \pathNnodes{3}{(0,1)}{right}{m};
      \pathNnodes{3}{(0,2)}{right}{t};
      \draw (t1) to (t2) to (t3) to (m3) to (m2) to (m1) to (b1) to (b2) to (b3);
    \end{tikzpicture}

    \(P_9\)
  \end{minipage}
  \caption{Path graphs.}
  \label{fig:path}
\end{figure}

Note that \(P_1=E_1\) is \chromatic{1}, whereas \(P_{n>1}\) is \chromatic{2}.

Paths are not particularly important to the proposed algorithm; however, they are used in the definition of cycles.

\subsubsection{Cycles}\label{sec:sub:sub:cycles}

A \emph{cycle} graph of order \(n\) and length \(n\) for \(n\ge3\), denoted by \(C_n\), is a connected graph
consisting of a single closed path.  When \(n\) is odd then \(C_n\) is called an \emph{odd} cycle and when \(n\) is
even then \(C_n\) is called an \emph{even} cycle.

Examples of cycle graphs are shown in \figurename~\ref{fig:cycle}.

\begin{figure}[H]
  \begin{minipage}{1.75in}
    \centering
    \begin{tikzpicture}[every node/.style=unlabeled node]
      \cycleN{3}{(0,0)}{0.5in}{90}{};
    \end{tikzpicture}

    \(C_3\) (odd)
  \end{minipage}
  \begin{minipage}{1.75in}
    \centering
    \begin{tikzpicture}[every node/.style=unlabeled node]
      \cycleN{4}{(0,0)}{0.5in}{135}{};
    \end{tikzpicture}

    \(C_4\) (even)
  \end{minipage}
  \begin{minipage}{1.75in}
    \centering
    \begin{tikzpicture}[every node/.style=unlabeled node]
      \cycleN{9}{(0,0)}{0.5in}{90}{};
    \end{tikzpicture}

    \(C_9\) (odd)
  \end{minipage}
  \caption{Cycle graphs.}
  \label{fig:cycle}
\end{figure}

Note that even cycles are \(2\)-chromatic; however, odd cycles are \(3\)-chromatic.

Cycles are not particularly important to the proposed algorithm; however, they are used in the definition of trees,
which are important to the later analysis of coloring algorithms.

\subsubsection{Complete Graphs}\label{sec:sub:sub:complete}

A \emph{complete} graph of order \(n\) and size \(\frac{n(n-1)}{2}\), denoted by \(K_n\), is a connected graph that
contains every possible edge: \(E(G)=\ps_2(V(G))\).  Thus, all of the vertices in a complete graph are universal.

Examples of complete graphs are shown in \figurename~\ref{fig:complete}.

\begin{figure}[H]
  \begin{minipage}{1.75in}
    \centering
    \begin{tikzpicture}[every node/.style=unlabeled node]
      \node at (0,0) {};
    \end{tikzpicture}

    \(K_1\)
  \end{minipage}
  \begin{minipage}{1.75in}
    \centering
    \begin{tikzpicture}[every node/.style=unlabeled node]
      \completeN{4}{(0,0)}{0.5in}{135}{};
    \end{tikzpicture}

    \(K_4\)
  \end{minipage}
  \begin{minipage}{1.75in}
    \centering
    \begin{tikzpicture}[every node/.style=unlabeled node]
      \completeN{9}{(0,0)}{0.5in}{90}{};
    \end{tikzpicture}

    \(K_9\)
  \end{minipage}
  \caption{Complete graphs.}
  \label{fig:complete}
\end{figure}

Note that \(K_1=P_1=E_1\).

Since all of the vertices in a complete graph are adjacent to each other, each vertex requires a separate color in
order to achieve a proper coloring.  Thus, \(K_n\) is \(n\)-chromatic and is also an important termination
condition for the proposed algorithm.

\subsubsection{Trees}\label{sec:sub:sub:trees}

A \emph{tree} is a connected graph that contains no cycles as subgraphs.  Typically, one vertex of the tree is
selected as the \emph{root} vertex and then the tree is depicted in layers that contain vertices that are
equidistant from the root vertex.  Thus, the bottom layer is composed entirely of pendant vertices, but pendant
vertices can exist in the other layers as well.  Such pendant vertices are usually referred to as \emph{leaves} in
this context.

An example tree is shown in \figurename~\ref{fig:tree}.  The root vertex \(r\) is shown in red and the leaf
vertices \(b,e,g,h,i,j,k\) are shown in green.

\begin{figure}[H]
  \centering
  \begin{tikzpicture}[every node/.style={labeled node}]
    \colorlet{cr}{red!25!white}
    \colorlet{cl}{green!25!white}
    \node [fill=cr] (r) at (0,0) {\(r\)};
    \node (v11) at (-3,-2) {\(a\)};
    \node [fill=cl] (v12) at (0,-2) {\(b\)};
    \node (v13) at (3,-2) {\(c\)};
    \node (v21) at (-4.5,-4) {\(d\)};
    \node [fill=cl] (v22) at (-3,-4) {\(e\)};
    \node (v23) at (-1.5,-4) {\(f\)};
    \node [fill=cl] (v24) at (1.75,-4) {\(g\)};
    \node [fill=cl] (v25) at (4.25,-4) {\(h\)};
    \node [fill=cl] (v31) at (-4.5,-6) {\(i\)};
    \node [fill=cl] (v32) at (-2.25,-6) {\(j\)};
    \node [fill=cl] (v33) at (-0.75,-6) {\(k\)};
    \draw (r) edge (v11) edge (v12) edge (v13);
    \draw (v11) edge (v21) edge (v22) edge (v23);
    \draw (v21) edge (v31);
    \draw (v23) edge (v32) edge (v33);
    \draw (v13) edge (v24) edge (v25);
  \end{tikzpicture}
  \caption{A tree organized from root to leaves.}
  \label{fig:tree}
\end{figure}

Trees are important because they can be used to track so-called ``branch-and-bound'' algorithms; each vertex
represents a branch choice of the algorithm and thus a particular state of the problem.  All such states can be
visited using a so-called \emph{depth-first} walk.  In the example in \figurename~\ref{fig:tree}, such a
depth-first walk would be:
\[(r,a,d,i,d,a,e,a,f,j,f,k,f,a,r,b,r,c,g,c,h,c,r)\]
Note that this walk guarantees that each vertex is visited at least once.

When such a tree is applied to the problem of exhaustively finding the chromatic number of a graph via a sequence
of vertex contraction and edge addition choices, the tree is called a \emph{Zykov} tree and the algorithm is called
a \emph{Zykov} algorithm~\cite{mcdiarmid}.  Zykov algorithms are described in detail in
\sectionname~\ref{sec:chromatic}.  In fact, the proposed algorithm is a variation of the standard Zykov algorithm.

\subsection{The Adjacency Matrix}\label{sec:sub:adjacency}

For a graph \(G\) of order \(n\), the adjacency matrix \(A\) for \(G\) is the \(n\times n\) matrix such that:
\[a_{ij}=\begin{cases}
0, & v_iv_j\notin E(G) \\
1, & v_iv_j\in E(G)
\end{cases}\]
In the case of a simple graph:
\begin{enumerate}
\item The \(a_{ij}\) values are limited to \(0\) and \(1\) in order to avoid multiple edges.
\item The diagonal values \(a_{ii}\) are always \(0\) in order to avoid loops.
\item \(A\) is symmetric due to the bidirectional nature of the edges.
\end{enumerate}

An example graph and its adjacency matrix are shown in \figurename~\ref{fig:adjacency}.

\begin{figure}[H]
  \begin{minipage}{2.5in}
    \centering
    \begin{tikzpicture}[every node/.style={labeled node}]
      \cycleV{\(1\),\(2\),\(3\),\(4\)}{(0,0)}{0.75in}{135}{};
      \draw (1) edge (3);
    \end{tikzpicture}
  \end{minipage}
  \begin{minipage}{2.5in}
    \centering
    \[\begin{bmatrix}
      0 & 1 & 1 & 1 \\
      1 & 0 & 1 & 0 \\
      1 & 1 & 0 & 1 \\
      1 & 0 & 1 & 1
    \end{bmatrix}\]
  \end{minipage}
  \caption{A graph and its adjacency matrix.}
  \label{fig:adjacency}
\end{figure}

The degree of vertex \(v_i\) can be calculated by summing the \(i^{th}\) row or the \(i^{th}\) column:
\[\deg(v_i)=\sum_{k=1}^na_{ik}=\sum_{k=1}^na_{ki}\]
The minimum \((\dmin(G))\) and maximum \((\dmax(G))\) degree values can then be calculated by selecting the minimum
and maximum calculated degree values.

The adjacency matrix is extremely important to graph algorithms since it provides an instant report of vertex
adjacency.  Furthermore, as the adjacency matrix is being constructed for a graph, it is easy to calculate each
vertex degree as well as the minimum and maximum degree and cache these values for later use.

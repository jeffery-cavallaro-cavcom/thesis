Mechanical engineering design practices are increasingly moving towards a framework called \emph{axiomatic design},
which starts with a set of independent \emph{functional requirements} ({\FR}s) for a manufactured product.  A key
tenet of axiomatic design is to decrease the \emph{information content} of a design in order to increase the chance
of manufacturing success.  One important way to decrease information content is to fulfill multiple {\FR}s by a
single part: a process known as \emph{part consolidation}.  Thus, an important parameter when comparing two
candidate designs is the minimum number of parts needed to satisfy all of the {\FR}s.  One possible method for
determining the minimum number of parts is to represent the problem by a graph, where the vertices are the FRs and
the edges represent the need to separate their endpoint FRs into separate parts.  The answer then becomes the
solution to a vertex coloring problem: finding the chromatic number of such a graph.  Unfortunately, the chromatic
number problem is known to be NP-hard.  This research investigates a new algorithm that determines the chromatic
number for a graph and compares the new algorithm's computer runtime performance to existing Zykov branch
and bound algorithms via random graph analysis.  

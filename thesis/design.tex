\section{Axiomatic Design}

The axiomatic design framework was developed in the late \(20^{th}\) century by Professor Nam P. Suh while at MIT
and the NSF \cite{suh}.  This was in response to concern in the engineering community that \emph{design} was being
practiced almost exclusively as an ad-hoc creative endeavor with very little in the way of scientific discipline.
In the words of Professor Suh:
\begin{quote}
  It [design] might have preceding the development of natural sciences by scores of centuries.  Yet, to this day,
  design is being done intuitively as an art.  It is one of the few technical areas where experience is more
  important than formal education.
\end{quote}
It is important to note that Professor Suh was not making these claims in an educational vacuum, but in the shadow
of several recent major design failures such as the Union Carbide plant disaster in India, nuclear power plant
accidents at Three Mile Island and Chenobyl, and the Challenger space shuttle O-ring failure.  Furthermore,
Professor Suh asserts that design-related issues resulting in production problems and operating failures were
increasingly happening in everything from consumer products to big-ticket items.

\emph{Design} is defined as the process by which it is determined \emph{what} needs to be achieved and then
\emph{how} to achieve it.  Thus, the decisions on what do to are just as important as how to do it.  Creativity is
the process by which experience and intuition are used to generate solutions to perceived needs.  This includes
pattern matching to and adapting existing solutions and synthesizing new solutions.  Thus, creativity plays a vital
role in design.  Since different designers may approach the same problem differently, their level of creativity may
lead to very different, yet plausible, solutions.  Thus, there needs to be a design-agnostic method for comparing
different designs with the goal of selecting the best one.

This discussion will sound very familiar to mathematicians, since creativity is very important for solving math
problems, and in particular, for writing proofs.  Starting with the work of Peano in the \(19^{th}\) century, the
field of mathematics has established various tests on what constitutes a good proof.  For example:
\begin{itemize}
\item Does every conclusion result from proper implication from existing definitions, axioms, and previously-proved
  conclusions?
\item Is direct proof, contrapositive proof, or proof by contradiction the best approach for a particular problem?
\item Are all subset and equality relationships properly proved via membership implication?
\item Are all necessary cases included and stated in a mutually exclusive manner?
\item Are all equivalences proved in a proper circular fashion?
\item Are key and reused conclusions highlighted in lemmas?
\end{itemize}
In short, Professor Suh was looking for a similar framework for the more general concept of design.

In a desire to not hinder the creative element needed for design, yet provide some methodology to distinguish bad
designs from good designs from better designs, the framework shown in Figure \ref{fig:design} establishes the
overall framework for axiomatic design.

\begin{figure}[h]
  \label{fig:design}
  \begin{center}
    \scalebox{0.75}{
      \begin{tikzpicture}[>=latex',every text node part/.style={align=center}]
        \node (cn) [draw,terminal] at (0,0) {customer \\ needs};
        \node (pd) [draw,process,right=of cn] {problem \\ definition};
        \node (cp) [draw,process,right=2cm of pd] {creative \\ process};
        \node (ap) [draw,process,right=2cm of cp] {analytical \\ process};
        \node (uc) [draw,process,right=of ap] {ultimate \\ check};
        \node (fd) [draw,terminal,right=of uc] {final \\ design};
        \draw [->] (cn) -- (pd);
        \draw [->] (pd) -- node [auto] {FRs} (cp);
        \draw [->] (cp) -- node [auto] {candidate \\ design} (ap);
        \draw [->] (ap) -- (uc);
        \draw [->] (uc) -- (fd);
        \draw [->] (ap) -- ($(ap) + (0,-1.5cm)$) -| (cp);
      \end{tikzpicture}
    }
  \end{center}
  \caption{Axiomatic Design Framework}
\end{figure}

Design starts with the desire to fulfill a clear customer need.  In this sense, the term \emph{customer} refers to
any entity that expresses a need, and can be as varied as individuals, organizations, or society.  The only
requirement is that the need can be clearly expressed.  The first job of the designer is to determine how these
needs will be meet by generating a list of \emph{functional requirements} (FRs).  It is this list of FRs that
determine exactly \emph{what} is to be accomplished.

Once the set of FRs has been determined, the designer begins the creative process by mapping the FRs into solutions
that are embodied in so-called \emph{design parameters} (DPs).  The design parameters contain all of the
information on \emph{how} the various FRs are fulfilled: parts lists, drawings, specifications, tolerances, etc.
This process is described by Figure \ref{fig:mapping}.

\begin{figure}[h]
  \label{fig:mapping}
  \begin{center}
    \begin{tikzpicture}[>=latex',every text node part/.style={align=center}]
      \node (fs) [draw,cloud] at (0,0) {FR1 \\ FR2 \\ FR3 \\ \vdots};
      \node [below=1ex of fs] {functional \\ space};
      \node (ps) [draw,cloud,right=1.5in of fs] {DP1 \\ DP2 \\ DP3 \\ \vdots};
      \node [below=1ex of ps] {physical \\ space};
      \draw [->] (fs) to [bend left=45] (ps);
      \draw [->] (fs) to [bend left=30] (ps);
      \draw [->] (fs) to [bend right=30] (ps);
      \draw [->] (fs) to [bend right=45] (ps);
      \draw [->] (fs) -- node [auto] {mapping} (ps);
    \end{tikzpicture}
  \end{center}
  \caption{Mapping FRs to DPs}
\end{figure}

Note that the FRs exist in a design-agnostic \emph{functional space} and the DPs exist in a solution-specific
physical space.  It is the designer's job to provide the most efficient mapping between the two spaces.  But since
the mapping process is non-unique, there needs to be a method to compare plausible designs.  The axiomatic design
framework includes two main axioms and their corollaries that define properties that are common to all good
designs.  These axioms are applied in the analytical phase to determine the quality of a design, defined as the
chances for success of the design based on its information content.  Thus, different designs can be compared based
on how well they fulfill the requirements of the axioms.  These axioms: the independence axiom and the information
axiom, are described in the following sections.

\subsection{The Independence Axiom}

The independence axiom states that the functional requirements (FRs) should be independent 

\begin{figure}
  \label{fig:coupling}
  \scalebox{0.75}{
    \begin{minipage}{3in}
      \begin{center}
        \[\begin{bmatrix}
        FR1 \\ FR2 \\ FR3
        \end{bmatrix}=\begin{bmatrix}
        X & 0 & 0 \\
        0 & X & 0 \\
        0 & 0 & X \\
        \end{bmatrix}\begin{bmatrix}
          DP1 \\ DP2 \\ DP3
        \end{bmatrix}\]
        Uncoupled
      \end{center}
    \end{minipage}
    \begin{minipage}{3in}
      \begin{center}
        \[\begin{bmatrix}
        FR1 \\ FR2 \\ FR3
        \end{bmatrix}=\begin{bmatrix}
        X & 0 & 0 \\
        X & X & 0 \\
        X & X & X \\
        \end{bmatrix}\begin{bmatrix}
          DP1 \\ DP2 \\ DP3
        \end{bmatrix}\]
        Decoupled
      \end{center}
    \end{minipage}
    \begin{minipage}{3in}
      \begin{center}
        \[\begin{bmatrix}
        FR1 \\ FR2 \\ FR3
        \end{bmatrix}=\begin{bmatrix}
        X & X & X \\
        X & X & X \\
        X & X & X \\
        \end{bmatrix}\begin{bmatrix}
          DP1 \\ DP2 \\ DP3
        \end{bmatrix}\]
        Coupled
      \end{center}
    \end{minipage}
  }
  \caption{Coupling Examples}
\end{figure}

\subsection{The Information Axiom}

\subsection{Part Consolidation}

One particular area of design focus is the number of parts in a product design:
\begin{quote}
  Poorly designed products often cost more because they use more materials or parts than do well-designed products.
  They are often difficult to manufacture and maintain.
\end{quote}
It is in this area, the number of parts, that this research desires to provide designers with a tool that they can
use during the design phase to make a determination of the minimum number of parts needed to realize a particular
design.  As will be shown, this is accomplished by proposing an algorithm that addresses the NP-hard problem of
finding the chromatic number of a graph.

\section{Axiomatic Design}

The axiomatic design framework was developed in the late \(20^{th}\) century by Professor Nam P. Suh while at MIT
and the NSF \cite{suh}.  This was in response to concern in the engineering community that \emph{design} was being
practiced almost exclusively as an ad-hoc creative endeavor with very little in the way of scientific discipline.
In the words of Professor Suh:
\begin{quote}
  It [design] might have preceding the development of natural sciences by scores of centuries.  Yet, to this day,
  design is being done intuitively as an art.  It is one of the few technical areas where experience is more
  important than formal education.
\end{quote}
It is important to note that Professor Suh was not making these claims in an educational vacuum, but in the shadow
of several recent major design failures such as the Union Carbide plant disaster in India, nuclear power plant
accidents at Three Mile Island and Chenobyl, and the Challenger space shuttle O-ring failure.  Furthermore,
Professor Suh asserts that design-related issues resulting in production problems and operating failures were
increasingly happening in everything from consumer products to big-ticket items.

The following sections provide an overview of axiomatic design as specified in detail by Professor Suh in \cite{suh},
and summarized in \cite{cavallaro,jahanbekam,suh2}.  Following is a description of how the proposed algorithm can be
a helpful tool to a designer using the axiomatic design framework.

\subsection{Design}

\emph{Design} is defined as the process by which it is determined \emph{what} needs to be achieved and then
\emph{how} to achieve it.  Thus, the decisions on what do to are just as important as how to do it.  Creativity is
the process by which experience and intuition are used to generate solutions to perceived needs.  This includes
pattern matching to and adapting existing solutions and synthesizing new solutions.  Thus, creativity plays a vital
role in design.  Since different designers may approach the same problem differently, their level of creativity may
lead to very different, yet plausible, solutions.  Thus, there needs to be a design-agnostic method for comparing
different designs with the goal of selecting the best one.

This discussion will sound very familiar to mathematicians, since creativity is very important for solving math
problems, and in particular, for writing proofs.  Starting with the work of Peano in the \(19^{th}\) century, the
field of mathematics has established various tests on what constitutes a good proof.  For example:
\begin{itemize}
\item Does every conclusion result by proper implication from existing definitions, axioms, and previously-proved
  conclusions?
\item Is direct proof, contrapositive proof, proof by contradiction, or proof by induction the best approach for a
  particular problem?
\item Do proofs by induction contain clear basic, assumptive, and inductive steps?
\item Are all subset and equality relationships properly proved via membership implication?
\item Are all necessary cases included and stated in a mutually exclusive manner?
\item Are all equivalences proved in a proper circular fashion?
\item Are key and reused conclusions highlighted in lemmas?
\end{itemize}
In short, Professor Suh was looking for a similar framework for the more general concept of design.

\subsection{The Framework}

In a desire to not hinder the creative element needed for design, yet provide some methodology to distinguish bad
designs from good designs from better designs, the diagram in Figure \ref{fig:design} establishes the overall
framework for axiomatic design.

\begin{figure}[h]
  \label{fig:design}
  \begin{center}
    \scalebox{0.75}{
      \begin{tikzpicture}[>=latex',every text node part/.style={align=center}]
        \node (cn) [draw,terminal] at (0,0) {customer \\ needs};
        \node (pd) [draw,process,right=of cn] {problem \\ definition};
        \node (cp) [draw,process,right=2cm of pd] {creative \\ process};
        \node (ap) [draw,process,right=2cm of cp] {analytical \\ process};
        \node (uc) [draw,process,right=of ap] {ultimate \\ check};
        \node (fd) [draw,terminal,right=of uc] {final \\ design};
        \draw [->] (cn) -- (pd);
        \draw [->] (pd) -- node [auto] {FRs} (cp);
        \draw [->] (cp) -- node [auto] {candidate \\ design} (ap);
        \draw [->] (ap) -- (uc);
        \draw [->] (uc) -- (fd);
        \draw [->] (ap) -- ($(ap) + (0,-1.5cm)$) -| (cp);
      \end{tikzpicture}
    }
  \end{center}
  \caption{Axiomatic Design Framework}
\end{figure}

Design starts with the desire to fulfill a clear set of \emph{customer needs}.  The term \emph{customer} refers to
any entity that expresses needs, and can be as varied as individuals, organizations, or society.  The designer, in
the \emph{problem definition} phase, determines how these customer needs will be meet by generating a list of
\emph{functional requirements} (FRs).  It is this list of FRs that determines exactly \emph{what} is to be
accomplished.

Once the set of FRs has been determined, the designer begins the \emph{creative process} by mapping the FRs into
solutions that are embodied in so-called \emph{design parameters} (DPs).  The design parameters contain all of the
information on \emph{how} the various FRs are fulfilled: parts lists, drawings, specifications, etc.  This process
is represented by Figure \ref{fig:mapping}.

\begin{figure}[h]
  \label{fig:mapping}
  \begin{center}
    \begin{tikzpicture}[>=latex',every text node part/.style={align=center}]
      \node (fs) [draw,cloud] at (0,0) {FR1 \\ FR2 \\ FR3 \\ \vdots};
      \node [below=1ex of fs] {functional \\ space};
      \node (ps) [draw,cloud,right=1.5in of fs] {DP1 \\ DP2 \\ DP3 \\ \vdots};
      \node [below=1ex of ps] {physical \\ space};
      \draw [->] (fs) to [bend left=45] (ps);
      \draw [->] (fs) to [bend left=30] (ps);
      \draw [->] (fs) to [bend right=30] (ps);
      \draw [->] (fs) to [bend right=45] (ps);
      \draw [->] (fs) -- node [auto] {mapping} (ps);
    \end{tikzpicture}
  \end{center}
  \caption{Mapping FRs to DPs}
\end{figure}

The FRs exist in a design-agnostic \emph{functional space} and the DPs exist in a solution-specific
\emph{physical space}.  It is the designer's job to provide the most efficient mapping between the two spaces.

The FR/DP mapping is described by the \emph{design equation}, which is shown in Equation \ref{equ:design}.
\begin{equation}
  \label{equ:design}
  [\text{FR}]=[\text{A}][\text{DP}]
\end{equation}
The design equation is a matrix equation that maps a vector of \(m\) FRs to a vector of \(n\) DPs via an \(m\times
n\) design matrix A.  As will be shown, good designs require \(m=n\), and thus the design matrix is a square
\(n\times n\) matrix.  A full discussion of the design matrix element values is beyond the scope of this research.
Instead, we will use the following two values:
\[A_{ij}=\begin{cases}
X, & \text{FR}_i\ \text{depends on DP}_j \\
0, & \text{FR}_i\ \text{does not depend on DP}_j
\end{cases}\]

Since the FR/DP mapping is non-unique, there needs to be a method to compare different plausible designs so that
the best design can be selected as the final design.  Thus, the framework in Figure \ref{fig:design} includes an
\emph{analytical process} where designs are judged by a set of axioms, corollaries, and theorems that specify the
properties common to all good designs.  Once the best design, according to this analysis, is selected, it undergoes
an \emph{ultimate check} to make sure that it sufficiently meets all of the customer's needs.  If so, then that
design is selected as the final design.

\subsection{The Axioms}

The independence axiom imposes a restriction on the FR/DP mapping:

\begin{axiom}[Independence]
  \label{axm:independ}
  An optimal design always maintains the independence of the FRs.  This means that the FRs and DPs are related in
  such a way that a specific DP can be adjusted to satisfy its corresponding FR without affecting other FRs.
\end{axiom}

\begin{figure}[h]
  \label{fig:coupling}
  \scalebox{0.75}{
    \begin{minipage}{3in}
      \begin{center}
        \[\begin{bmatrix}
        FR1 \\ FR2 \\ FR3
        \end{bmatrix}=\begin{bmatrix}
        X & 0 & 0 \\
        0 & X & 0 \\
        0 & 0 & X \\
        \end{bmatrix}\begin{bmatrix}
          DP1 \\ DP2 \\ DP3
        \end{bmatrix}\]
        Uncoupled
      \end{center}
    \end{minipage}
    \begin{minipage}{3in}
      \begin{center}
        \[\begin{bmatrix}
        FR1 \\ FR2 \\ FR3
        \end{bmatrix}=\begin{bmatrix}
        X & 0 & 0 \\
        X & X & 0 \\
        X & X & X \\
        \end{bmatrix}\begin{bmatrix}
          DP1 \\ DP2 \\ DP3
        \end{bmatrix}\]
        Decoupled
      \end{center}
    \end{minipage}
    \begin{minipage}{3in}
      \begin{center}
        \[\begin{bmatrix}
        FR1 \\ FR2 \\ FR3
        \end{bmatrix}=\begin{bmatrix}
        X & X & X \\
        X & X & X \\
        X & X & X \\
        \end{bmatrix}\begin{bmatrix}
          DP1 \\ DP2 \\ DP3
        \end{bmatrix}\]
        Coupled
      \end{center}
    \end{minipage}
  }
  \caption{Coupling Examples}
\end{figure}

\subsection{Part Consolidation}

One particular area of design focus is the number of parts in a product design:
\begin{quote}
  Poorly designed products often cost more because they use more materials or parts than do well-designed products.
  They are often difficult to manufacture and maintain.
\end{quote}
It is in this area, the number of parts, that this research desires to provide designers with a tool that they can
use during the design phase to make a determination of the minimum number of parts needed to realize a particular
design.  As will be shown, this is accomplished by proposing an algorithm that addresses the NP-hard problem of
finding the chromatic number of a graph.

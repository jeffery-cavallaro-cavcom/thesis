\section{Graph Theory}

This section presents the concepts, definitions, and theorems from the field of graph theory that are needed in the
development of the proposed algorithm.  This material is primarily taken from the textbooks used
\cite{chartrand,west} and class notes compiled by the author during his undergraduate and graduate graph theory
classes at SJSU.

\subsection{Simple Graphs}

The problem of part consolidation is best served by a class of graphs called \emph{simple graphs}:

\begin{definition}[Simple Graph]
  A \emph{simple graph} is a mathematical object represented by a tuple \(G=(V,E,\ldots)\) consisting of a
  non-empty and finite set of \emph{vertices} (also called \emph{nodes}) \(V(G)\), a finite and possibly empty set
  of edges \(E(G)\), and zero of more relations.  Each edge is represented by a two-element subset of \(V(G)\)
  called the \emph{endpoints} of the edge:
  \[E(G)\subseteq\ps_2\left(V(G)\right)\]
  Each relation has \(V(G)\) or \(E(G)\) as its domain and is used to associated vertices or edges with
  problem-specific attributes.
\end{definition}

For the remainder of this work, the use of the term ``graph'' implies a ``simple graph.''

The choice of two-element subsets of \(V(G)\) for the edges has certain ramifications that are indeed characteristics
that differentiate a simple graph from other classes of graphs:
\begin{enumerate}
\item Every two vertices of a graph are the endpoints of at most one edge; there are no so-called
  \emph{multiple} edges between two vertices.
\item The two endpoint vertices of an edge are always distinct; there are no so-called \emph{loop} edges on a
  single vertex.
\item The two endpoint vertices are unordered, suggesting that an edge provides a bidirectional connection between
  its endpoint vertices.
\end{enumerate}

A part consolidation problem can be represented by a graph whose vertices are the functional requirements (FRs) of
the design and whose edges indicate which endpoint FRs should never be combined into a single part.

Graphs are often portrayed visually using labeled or filled circles for the vertices and lines for the edges such
that each edge line is drawn between its two endpoint vertices.  An example graph is shown in Figure
\ref{fig:exgraph}.

\begin{figure}[h]
  \label{fig:exgraph}
  \begin{minipage}{3in}
    \vspace{0in}
    \begin{center}
      \begin{tikzpicture}[node distance=1cm,every node/.style={labeled node}]
        \node (E) at (0,0) {\(e\)};
        \node (A) [above left=of E] {\(a\)};
        \node (B) [above right=of E] {\(b\)};
        \node (C) [below right=of E] {\(c\)};
        \node (D) [below left=of E] {\(d\)};
        \draw (A) edge (B);
        \draw (B) edge (E);
        \draw (E) edge (A);
        \draw (A) edge (D);
      \end{tikzpicture}
    \end{center}
  \end{minipage}
  \begin{minipage}{3in}
    \vspace{0in}
    \begin{center}
      \begin{tikzpicture}[node distance=1.75cm,every node/.style={unlabeled node}]
        \node (E) at (0,0) {};
        \node (A) [above left=of E] {};
        \node (B) [above right=of E] {};
        \node (C) [below right=of E] {};
        \node (D) [below left=of E] {};
        \draw (A) edge (B);
        \draw (B) edge (E);
        \draw (E) edge (A);
        \draw (A) edge (D);
      \end{tikzpicture}
    \end{center}
  \end{minipage}
  \begin{gather*}
    V=V(G)=\set{a,b,c,d,e} \\
    E=E(G)=\set[\big]{\set{a,b},\set{a,d},\set{a,e},\set{b,e}}
  \end{gather*}
  \caption{An Example Graph (labeled and unlabeled)}
\end{figure}

\begin{samepage}
  When referring to the edges in a graph, the following common notation will be used:

  \begin{notation}[Edge]
    The edge \(\set{u,v}\) is represented by the simple juxtaposition \(uv\) or \(vu\).
  \end{notation}
\end{samepage}

Note that there is no requirement that every vertex in a graph be an endpoint to some edge:

\begin{definition}[Isolated Vertex]
  Let \(G\) be a graph and let \(u\in V(G)\).  To say that \(u\) is an \emph{isolated} vertex means that it is not
  an endpoint for any edge in \(E(G)\):
  \[\forall\,e\in E(G),u\notin e\]
\end{definition}

In the example graph of Figure \ref{fig:exgraph}, notice that vertex \(c\) is an isolated vertex.

When two vertices are the endpoints of the same edge the vertices are said to be \emph{adjacent} or are called
\emph{neighbors}:

\begin{definition}[Adjacent Vertices]
  Let \(G\) be a graph and let \(u,v\in V(G)\).  To say that \(u\) and \(v\) are \emph{adjacent} vertices, also
  called \emph{neighbors}, means that they are the endpoints of some edge \(e\in E(G)\):
  \[\exists\,e\in E(G),e=uv\]
  The edge \(e\) is said to \emph{join} its two endpoint vertices \(u\) and \(v\).  Furthermore, the edge \(e\) is
  said to be \emph{incident} to its endpoint vertices \(u\) and \(v\).
\end{definition}

In the example graph of Figure \ref{fig:exgraph}, notice that vertex \(a\) is adjacent to vertices \(b\), \(e\),
and \(d\); and vertex \(b\) is adjacent to vertex \(e\).

We can also speak of adjacent edges, which are edges that share an endpoint:

\begin{definition}[Adjacent Edges]
  Let \(G\) be a graph and left \(e,f\in E(G)\).  To say that \(e\) and \(f\) are \emph{adjacent} edges means that
  they share some endpoint \(v\in E(G)\):
  \[\exists\,v\in V(G),e\cap f=\set{v}\]
  or similarly:
  \[\abs{e\cap f}=1\]
\end{definition}

Note that two edges in a simple graph can only share one endpoint; otherwise, the two edges would be multiple edges,
which are not allowed in simple graphs.

In the example graph of Figure \ref{fig:exgraph}, notice that \(ab\) is adjacent to \(ad\), \(ae\), and \(be\); and
\(ae\) is adjacent to \(be\).

\subsection{Order and Size}

Two of the most important characteristics of a graph are the number of vertices in the graph, called the \emph{order}
of the graph, and the number of edges in the graph, called the \emph{size} of the graph:

\begin{definition}[Order]
  Let \(G\) be a graph.  The \emph{order} of \(G\), denoted by \(n\) or \(n(G)\), is the number of vertices in
  \(G\):
  \[n=n(G)=\abs{V(G)}\]
\end{definition}

\begin{definition}[Size]
  Let \(G\) be a graph.  The \emph{size} of \(G\), denoted by \(m\) or \(m(G)\), is the number of edges in \(G\):
  \[m=m(G)=\abs{E(G)}\]
\end{definition}

In the example graph of Figure \ref{fig:exgraph}, notice that \(n=5\) and \(m=4\).

Since every two vertices can have at most one edge between them, the number of edges has an upper bound:

\begin{theorem}
  Let \(G\) be a graph of order \(n\) and size \(m\):
  \[m\le\frac{n(n-1)}{2}\]
\end{theorem}

\begin{proof}
  Since each pair of distinct vertices in \(V(G)\) can have zero or one edges joining them, the maximum number of
  possible edges is \(\binom{n}{2}\), and so:
  \[m\le\binom{n}{2}=\frac{n!}{2!(n-2)!}=\frac{n(n-1)}{2}\]
\end{proof}

Some choices of graph order and size lead to certain degenerate cases that serve as important termination cases for
the the proposed algorithm:

\begin{definition}[Degenerate Cases]
  \begin{itemize}[left=0pt]
  \item[]
  \item The \emph{null} graph is the non-graph with no vertices \((n=m=0)\).
  \item The \emph{trivial} graph is the graph with exactly one vertex and no edges \((n=1,m=0)\).  Otherwise, the
    graph is \emph{non-trivial}.
    \item An \emph{empty} graph is a graph with possibly some isolated vertices but with no edges \((m=0)\).
  \end{itemize}
\end{definition}

Hence, both the null and trivial graphs are empty.

\subsection{Graph Tuple Relations}

Various problems in graph theory require that vertices and edges be assigned values of particular attributes.  This
is accomplished by adding relations to the graph tuple that map the vertices and/or edges to their attribute values.
Note that there are no particular limitations on the nature of such a relation --- everything from a basic relation
to a bijective function is possible, depending on the problem.

In practice, when a graph theory problem requires a particular vertex or edge attribute, the presence of some
corresponding relation \(\sR\) is assumed and we say something like, ``vertex \(v\) has attribute \(a\),'' instead
of the more formal, ``vertex \(v\) has attribute \(\sR(v)\).''

The following sections describe the two relations used by the proposed algorithm.

\subsubsection{Labels}

One of the possible relations in a graph tuple is a bijective function that assigns each vertex an identifying label.
When such a function is present, the graph is said to be a \emph{labeled} graph:

\begin{definition}[Labeled Graph]
  To say that a graph \(G\) is \emph{labeled} means that its vertices are considered to be distinct and are
  assigned identifying names (labels) by adding a bijective labeling function to the graph tuple:
  \[\ell:V(G)\to L\]
  where \(L\) is a set of labels (names).  Otherwise, the vertices are considered to be identical (only the
  structure of the graph matters) and the graph is \emph{unlabeled}.
\end{definition}

The vertices in a labeled graph are typically draw as open circles containing the corresponding labels, whereas the
vertices in an unlabeled graph are typically drawn as filled circles.  This is demonstrated in the example graph of
Figure \ref{fig:exgraph}: the graph on the left is labeled and the graph on the right is unlabeled.

Since the labeling function \(\ell\) is bijective, a vertex \(v\in V(G)\) with label ``a'' can be identified by
\(v\) or \(\ell^{-1}(a)\).  In practice, the presence of a labeling function is assumed for a labeled graph and so
a vertex is freely identified by its label.  This is important to note when a proof includes a phrase such as,
``let \(v\in V(G)\ldots\)'' since \(v\) may be a reference to any vertex in \(V(G)\) or may call out a specific
vertex by its label; the intention is usually clear from the context.

The design graphs that act as the inputs to the proposed algorithm are labeled graphs, where the labels represent
the various functional requirements:
\[FR_1,FR_2,FR_3,\ldots,FR_n\]

\subsubsection{Vertex Color}

Other graph theory problems require that the graph's vertex set be distributed into some number of sets based on
some problem-specific criteria.  Usually, this distribution is a true partition (no empty sets), but this is not
required depending on the problem.  One popular method of performing this distribution is by adding a
\emph{coloring} function to the graph tuple:
\[c:V(G)\to C\]
where \(C\) is a set of \emph{colors}; vertices with the same color are assigned to the same set in the
distribution.  Although the elements of \(C\) are usually actual colors (red, green, blue, etc.), a graph coloring
problem is free to select any value type for the color attribute.  Note that there is no assumption that \(c\) is
surjective, so the codomain \(C\) may contain unused colors, which corresponds to empty sets in the distribution.

The most popular coloring scheme for a graph requires that adjacent vertices be assigned different colors:

\begin{definition}[Proper Coloring]
  A coloring \(c\) on a graph \(G\) is called \emph{proper} when no two adjacent vertices are assigned the same color:
  \[\forall\,u,v\in V(G),uv\in E(G)\implies c(u)\ne c(v)\]
  A proper coloring \(c\) with \(\abs{C}=k\) is called a \emph{\coloring{k}} of \(G\) and \(G\) is said to be
  \emph{\colorable{k}}, meaning the actual coloring (range of \(c\)) uses \emph{at most} \(k\) colors.
\end{definition}

An example of a \coloring{4} is shown in Figure \ref{fig:exproper}.

\begin{figure}[h]
  \label{fig:exproper}
  \begin{minipage}[t]{3in}
    \begin{center}
      \vspace{0in}
      \begin{tikzpicture}
        \colorlet{c1}{green!50!white}
        \colorlet{c2}{blue!50!white}
        \colorlet{c3}{red!50!white}
        \colorlet{c4}{orange!50!white}
        \begin{scope}[node distance=2cm,every node/.style={labeled node}]
          \node (E) [fill=c3] at (0,0) {\(e\)};
          \node (A) [above left=of E,fill=c1] {\(a\)};
          \node (B) [above right=of E,fill=c2] {\(b\)};
          \node (C) [below right=of E,fill=c1] {\(c\)};
          \node (D) [below left=of E,fill=c4] {\(d\)};
        \end{scope}
        \draw (A) edge (B);
        \draw (B) edge (E);
        \draw (E) edge (A);
        \draw (A) edge (D);
        \draw (E) edge (C);
      \end{tikzpicture}
    \end{center}
  \end{minipage}
  \begin{minipage}[t]{3in}
    \begin{gather*}
      c(a)=green \\
      c(b)=blue \\
      c(c)=green \\
      c(d)=orange \\
      c(e)=red
    \end{gather*}
  \end{minipage}
  \caption{A Graph with a \coloring{4}}
\end{figure}

Since there is no requirement that a coloring \(c\) be surjective, the codomain \(C\) may contain unused colors.
For example, the codomain of the coloring shown in Figure \ref{fig:exproper} might be:
\[C=\set{green,blue,red,orange}\]
and hence \(c\) is surjective and \(G\) is \colorable{4}.  But we can always add an unused color to \(C\):
\[C=\set{green,blue,red,orange,brown}\]
Now, \(c\) is no longer surjective, and according to the definition: \(G\) is \colorable{5} --- the coloring \(c\)
uses at most 5 colors (actually only 4), which is the cardinality of the codomain.

Thus, we can make statement in Proposition \ref{prop:coloring}.

\begin{proposition}
  \label{prop:coloring}
  Let \(G\) be a graph:
  \begin{quote}
    \(G\) is \colorable{k} \(\implies G\) is \colorable{(k+1)}
  \end{quote}
\end{proposition}

By inductive application of Proposition \ref{prop:coloring}, one can arrive at the conclusion in Proposition
\ref{prop:coloring2}.

\begin{proposition}
  \label{prop:coloring2}
  Let \(G\) be a graph:
  \begin{quote}
    \(G\) is \colorable{k} \(\implies G\) is \colorable{(k+r)} for some \(r\in\N\).
  \end{quote}
\end{proposition}

Furthermore, for a graph \(G\) of order \(n\), if \(n\le k\) we can conclude that \(G\) is \colorable{k}, since
there are sufficient colors such that each vertex can be assigned its own color.  This is stated in Proposition
\ref{prop:coloring3}, which will turn out to be an important termination case for the proposed algorithm.

\begin{proposition}
  \label{prop:coloring3}
  Let \(G\) be a graph of order \(n\) and let \(k\in\N\):
  \begin{quote}
    If \(n\le k\) then \(G\) is \colorable{k}.
  \end{quote}
\end{proposition}

Since \(k\in\N\), by the well-ordering principle, there exists some minimum \(k\) such that a graph \(G\) is
\colorable{k}:

\begin{definition}[Chromatic Coloring]
  The minimum \(k\) such that a graph \(G\) is \colorable{k} is called the \emph{chromatic number} of \(G\), denoted
  by \(\X(G)\).  A \coloring{k} for a graph \(G\) where \(k=\X(G)\) is called a \emph{\chromatic{k}} coloring.
\end{definition}

Returning to the example \coloring{4} of Figure \ref{fig:exproper}, note that vertex \(d\) can be colored blue and
then orange can be excluded from the codomain, resulting in a \coloring{3}.  This is shown in Figure
\ref{fig:exchromatic}.  Since there is no way to use less than 3 colors to obtain a proper coloring of the graph,
the coloring is \chromatic{3}.  Note that when a coloring is chromatic, there are no unused colors (empty sets) and
hence the distribution is a true partition.

\begin{figure}[h]
  \label{fig:exchromatic}
  \begin{minipage}[t]{3in}
    \begin{center}
      \vspace{0in}
      \begin{tikzpicture}
        \colorlet{c1}{green!50!white}
        \colorlet{c2}{blue!50!white}
        \colorlet{c3}{red!50!white}
        \begin{scope}[node distance=2cm,every node/.style={labeled node}]
          \node (E) [fill=c3] at (0,0) {\(e\)};
          \node (A) [above left=of E,fill=c1] {\(a\)};
          \node (B) [above right=of E,fill=c2] {\(b\)};
          \node (C) [below right=of E,fill=c1] {\(c\)};
          \node (D) [below left=of E,fill=c2] {\(d\)};
        \end{scope}
        \draw (A) edge (B);
        \draw (B) edge (E);
        \draw (E) edge (A);
        \draw (A) edge (D);
        \draw (E) edge (C);
      \end{tikzpicture}
    \end{center}
  \end{minipage}
  \begin{minipage}[t]{3in}
    \begin{gather*}
      c(a)=green \\
      c(b)=blue \\
      c(c)=green \\
      c(d)=blue \\
      c(e)=red
    \end{gather*}
  \end{minipage}
  \caption{A Graph with a Chromatic \coloring{3}}
\end{figure}

The primary purpose of a \coloring{k} of a graph \(G\) is to distribute the vertices of \(G\) into \(k\) so-called
\emph{independent} (some possibly empty) sets:

\begin{definition}[Independent Set]
  Let \(G\) be a graph and let \(S\subseteq V(G)\).  To say that \(S\) is an \emph{independent} set means that all of
  the vertices in \(S\) are non-adjacent in \(G\):
  \[\forall\,u,v\in S,uv\notin E(G)\]
\end{definition}

Since a \chromatic{k} coloring of a graph \(G\) is surjective, there are no unused colors (empty sets) and so the
coloring partitions the vertices of \(G\) into exactly \(k\) independent sets.  The goal of the proposed algorithm
is to find a chromatic coloring of a design graph so that the resulting independent sets indicate how to
consolidate the FRs into a minimum number of parts: one part per independent set.

\subsection{Subgraphs}

The basic strategy of the proposed algorithm is to arrive at a solution by mutating an input graph into simpler
graphs such that a solution is more easily determined.  The algorithm utilizes three particular mutators: vertex
deletion, edge addition, and vertex contraction.  Before describing these mutators, it will be helpful to describe
what is meant by graph equality and a \emph{subgraph} of a graph.

\subsubsection{Graph Equality}

Graph equality follows from equality of the vertex and edge sets:

\begin{definition}[Graph Equality]
  Let \(G\) and \(H\) be graphs.  To say that \(G\) is equal to \(H\), denoted \(G=H\), means that \(V(G)=V(H)\)
  and \(E(G)=E(H)\).
\end{definition}

Note that this definition of equality ignores any additional relations that may be added to the graph tuples since
those relations tend to be added by specific problems and do not reflect the actual parts of the graphs.

\subsubsection{Subgraphs}

Since graph equality follows from vertex and edge set equality, there should also be a concept of a \emph{subgraph}
resulting from the subsets of those sets:

\begin{definition}[Subgraph]
  Let \(G\) and \(H\) be two graphs:
  \begin{itemize}
  \item To say that \(H\) is a \emph{subgraph} of \(G\), denoted \(H\subseteq G\), means that \(V(H)\subseteq V(G)\)
    and \(E(H)\subseteq E(G)\).
  \item To say that \(H\) is a \emph{proper subgraph} of \(G\), denoted \(H\subset G\), means that \(H\subseteq G\)
    but \(H\ne G\): \(V(H)\subset V(G)\) or \(E(H)\subset E(G)\).
  \item To say that \(H\) is a \emph{spanning subgraph} of \(G\) means that \(H\) is a subgraph of \(G\) such that
    \(V(H)=V(G)\) and \(E(H)\subseteq E(G)\).
  \end{itemize}
\end{definition}

Thus, given a graph \(G\) and a subgraph \(H\), there should be a sequence of zero or more vertex and/or edge
removals to obtain \(H\) from \(G\).  Likewise, there should be a sequence of zero or more vertex and/or edge
additions to obtain \(G\) from \(H\).  If \(H\) is a proper subgraph of \(G\) then \(H\) and \(G\) differ by at
least one removed vertex or one removed edge.  If \(H\) is a spanning subgraph of \(G\) then \(H\) contains all of
the vertices in \(G\) but may differ by removed edges only.  Per the definition, a graph is always a subgraph of
itself \((G\subseteq G)\) and the null graph is a subgraph of every graph.

The concept of subgraphs is demonstrated by graphs \(G\), \(H\), and \(F\) in Figure \ref{fig:subgraphs}.  \(H\) is
a proper subgraph of \(G\) by removing vertices \(c\) and \(d\) and edges \(ad\) and \(be\).  \(F\) is a proper
spanning subgraph of \(G\) because \(F\) contains all of the vertices in \(G\) but is missing edges \(ab\) and
\(be\).

\begin{figure}[h]
  \label{fig:subgraphs}
  \begin{minipage}{2in}
    \begin{center}
      \begin{tikzpicture}[node distance=1cm,every node/.style={labeled node}]
        \node (E) at (0,0) {\(e\)};
        \node (A) [above left=of E] {\(a\)};
        \node (B) [above right=of E] {\(b\)};
        \node (C) [below right=of E] {\(c\)};
        \node (D) [below left=of E] {\(d\)};
        \draw (A) edge (B);
        \draw (B) edge (E);
        \draw (E) edge (A);
        \draw (A) edge (D);
      \end{tikzpicture}

      \bigskip

      \(G\)
    \end{center}
  \end{minipage}
  \begin{minipage}{2in}
    \begin{center}
      \begin{tikzpicture}[node distance=1cm,every node/.style={labeled node}]
        \node (E) at (0,0) {\(e\)};
        \node (A) [above left=of E] {\(a\)};
        \node (B) [above right=of E] {\(b\)};
        \node (C) [below right=of E,color=white] {};
        \node (D) [below left=of E,color=white] {};
        \draw (A) edge (B);
        \draw (E) edge (A);
      \end{tikzpicture}

      \bigskip

      \(H\subset G\) (proper)
    \end{center}
  \end{minipage}
  \begin{minipage}{2in}
    \begin{center}
      \begin{tikzpicture}[node distance=1cm,every node/.style={labeled node}]
        \node (E) at (0,0) {\(e\)};
        \node (A) [above left=of E] {\(a\)};
        \node (B) [above right=of E] {\(b\)};
        \node (C) [below right=of E] {\(c\)};
        \node (D) [below left=of E] {\(d\)};
        \draw (E) edge (A);
        \draw (A) edge (D);
      \end{tikzpicture}

      \bigskip

      \(F\subset G\) (spanning)
    \end{center}
  \end{minipage}
  \caption{Subgraph Examples}
\end{figure}

\subsubsection{Induced Subgraphs}

An \emph{induced} subgraph is a special type of subgraph:

\begin{definition}[Induced Subgraph]
  Let \(G\) be a graph and let \(S\) be a non-empty subset of \(V(G)\).  The subgraph of \(G\) \emph{induced} by
  \(S\), denoted \(G[S]\), is a subgraph \(H\) such that:
  \begin{itemize}
  \item \(V(H)=S\)
  \item \(u,v\in V(H)\) and \(uv\in E(G)\implies uv\in E(H)\)
  \end{itemize}
  Such a subgraph \(H\) is called an \emph{induced subgraph} of \(G\).
\end{definition}

Note that when a vertex is removed to make an induced subgraph then all of that vertex's incident edges are also
removed.  However, for every pair of vertices included in an induced subgraph, if the vertices are the endpoints of
a particular edge then that edge must also be included in the subgraph.  In the examples of Figure
\ref{fig:subgraphs}, \(H\) is not an induced subgraph of \(G\) because it is missing edge \(be\).  Likewise, a
proper spanning subgraph like \(F\) can never be induced due to missing edges.  In fact, the only induced spanning
subgraph of a graph is the graph itself.  Figure \ref{fig:induced} adds edge \(be\) so that \(H\) is now an induced
subgraph of \(G\).

\begin{figure}[h]
  \label{fig:induced}
  \begin{minipage}{3in}
    \begin{center}
      \begin{tikzpicture}[node distance=1cm,every node/.style={labeled node}]
        \node (E) at (0,0) {\(e\)};
        \node (A) [above left=of E] {\(a\)};
        \node (B) [above right=of E] {\(b\)};
        \node (C) [below right=of E] {\(c\)};
        \node (D) [below left=of E] {\(d\)};
        \draw (A) edge (B);
        \draw (B) edge (E);
        \draw (E) edge (A);
        \draw (A) edge (D);
      \end{tikzpicture}

      \bigskip

      \(G\)
    \end{center}
  \end{minipage}
  \begin{minipage}{3in}
    \begin{center}
      \begin{tikzpicture}[node distance=1cm,every node/.style={labeled node}]
        \node (E) at (0,0) {\(e\)};
        \node (A) [above left=of E] {\(a\)};
        \node (B) [above right=of E] {\(b\)};
        \node (C) [below right=of E,color=white] {};
        \node (D) [below left=of E,color=white] {};
        \draw (A) edge (B);
        \draw (B) edge (E);
        \draw (E) edge (A);
      \end{tikzpicture}

      \bigskip

      \(H=G[\set{a,b,e}]\)
    \end{center}
  \end{minipage}
  \caption{Induced Subgraph Example}
\end{figure}

\subsection{Mutators}

The following sections describe the graph mutators used by the proposed algorithm.

\subsubsection{Vertex Removal}

Let \(G\) be a graph and let \(S\subseteq V(G)\).  The induced subgraph obtained by removing all of the vertices in
\(S\) (and their incident edges) is denoted by:
\[G-S=G[V(G)-S]\]
If \(S\ne\emptyset\) then \(G-S\) is a proper subgraph of \(G\).  If \(S=V(G)\) then the result is the null graph.

Figure \ref{fig:vremove} shows an example of vertex removal: vertices \(c\) and \(e\) are removed, along with their
incident edges \(ae\) and \(be\).

\begin{figure}[h]
  \label{fig:vremove}
  \begin{minipage}{3in}
    \begin{center}
      \begin{tikzpicture}[node distance=1cm,every node/.style={labeled node}]
        \node [red] (E) at (0,0) {\(e\)};
        \node (A) [above left=of E] {\(a\)};
        \node (B) [above right=of E] {\(b\)};
        \node [red] (C) [below right=of E] {\(c\)};
        \node (D) [below left=of E] {\(d\)};
        \draw (A) edge (B);
        \draw [red] (B) edge (E);
        \draw [red] (E) edge (A);
        \draw (A) edge (D);
      \end{tikzpicture}

      \bigskip

      \(G\)
    \end{center}
  \end{minipage}
  \begin{minipage}{3in}
    \begin{center}
      \begin{tikzpicture}[node distance=1cm,every node/.style={labeled node}]
        \node (A) [above left=of E] {\(a\)};
        \node (B) [above right=of E] {\(b\)};
        \node (C) [below right=of E,white] {\(c\)};
        \node (D) [below left=of E] {\(d\)};
        \draw (A) edge (B);
        \draw (A) edge (D);
      \end{tikzpicture}

      \bigskip

      \(G-\set{c,e}\)
    \end{center}
  \end{minipage}
  \caption{Vertex Removal Example}
\end{figure}

If \(\abs{S}=1\) then an alternate syntax can be used.  Assume \(v\in V(G)\):
\[G-v=G-\set{v}\]

The proposed algorithm uses vertex removal to simplify a \colorable{k} graph into a simpler subgraph that is still
\colorable{k}.

\subsubsection{Edge Addition}

Let \(G\) be a graph and let \(u,v\in V(G)\) such that \(uv\notin E(G)\).  The graph \(G+uv\) is the graph with the
same vertices as \(G\) and with edge set \(E(G)\cup\set{uv}\).  Note that \(G\) is a proper spanning subgraph of
\(G+uv\).

Figure \ref{fig:eadd} shows an example of edge addition: edge \(cd\) is added.

\begin{figure}[h]
  \label{fig:eadd}
  \begin{minipage}{3in}
    \begin{center}
      \begin{tikzpicture}[node distance=1cm,every node/.style={labeled node}]
        \node (E) at (0,0) {\(e\)};
        \node (A) [above left=of E] {\(a\)};
        \node (B) [above right=of E] {\(b\)};
        \node (C) [below right=of E] {\(c\)};
        \node (D) [below left=of E] {\(d\)};
        \draw (A) edge (B);
        \draw (B) edge (E);
        \draw (E) edge (A);
        \draw (A) edge (D);
      \end{tikzpicture}

      \bigskip

      \(G\)
    \end{center}
  \end{minipage}
  \begin{minipage}{3in}
    \begin{center}
      \begin{tikzpicture}[node distance=1cm,every node/.style={labeled node}]
        \node (E) at (0,0) {\(e\)};
        \node (A) [above left=of E] {\(a\)};
        \node (B) [above right=of E] {\(b\)};
        \node (C) [below right=of E] {\(c\)};
        \node (D) [below left=of E] {\(d\)};
        \draw (A) edge (B);
        \draw (B) edge (E);
        \draw (E) edge (A);
        \draw (A) edge (D);
        \draw [green] (C) edge (D);
      \end{tikzpicture}

      \bigskip

      \(G+cd\)
    \end{center}
  \end{minipage}
  \caption{Edge Addition Example}
\end{figure}

The proposed algorithm uses edge addition to prevent two non-adjacent FRs from being consolidated into the same
part.

\subsubsection{Edge Removal}

The proposed algorithm does not use edge removal; however, a number of related algorithms do rely on this mutator
so it is presented here.  Let \(G\) be a graph and let \(X\subseteq E(G)\).  The spanning subgraph obtained by
removing all of the edges in \(X\) is denoted by:
\[G-X=H\left(V(G),E(G)-X\right)\]
Thus, only edges are remove --- no vertices are removed.  If \(X\ne\emptyset\) then \(G-X\) is a proper subgraph of
\(G\).  If \(X=E(G)\) then the result is an empty graph (no edges).

Figure \ref{fig:eremove} shows an example of edge removal: edges \(ae\) and \(be\) are removed.

\begin{figure}[h]
  \label{fig:eremove}
  \begin{minipage}{3in}
    \begin{center}
      \begin{tikzpicture}[node distance=1cm,every node/.style={labeled node}]
        \node (E) at (0,0) {\(e\)};
        \node (A) [above left=of E] {\(a\)};
        \node (B) [above right=of E] {\(b\)};
        \node (C) [below right=of E] {\(c\)};
        \node (D) [below left=of E] {\(d\)};
        \draw (A) edge (B);
        \draw [red] (B) edge (E);
        \draw [red] (E) edge (A);
        \draw (A) edge (D);
      \end{tikzpicture}

      \bigskip

      \(G\)
    \end{center}
  \end{minipage}
  \begin{minipage}{3in}
    \begin{center}
      \begin{tikzpicture}[node distance=1cm,every node/.style={labeled node}]
        \node (E) at (0,0) {\(e\)};
        \node (A) [above left=of E] {\(a\)};
        \node (B) [above right=of E] {\(b\)};
        \node (C) [below right=of E] {\(c\)};
        \node (D) [below left=of E] {\(d\)};
        \draw (A) edge (B);
        \draw (A) edge (D);
      \end{tikzpicture}

      \bigskip

      \(G-\set{ae,be}\)
    \end{center}
  \end{minipage}
  \caption{Edge Removal Example}
\end{figure}

If \(\abs{X}=1\) then an alternate syntax can be used.  Assume \(e\in E(G)\):
\[G-e=G-\set{e}\]

\subsubsection{Vertex Contraction}

Vertex contraction is a bit different because it does not involve subgraphs.  Let \(G\) be a graph and let \(u,v\in
V(G)\).  The graph \(G\cdot uv\) is constructed by identifying \(u\) and \(v\) as one vertex (i.e., merging them).
Any edge between the two vertices is discarded.  Any other edges that were incident to the two vertices become
incident to the new single vertex.  Note that this may require supression of multiple edges to preserve the nature
of a simple graph.

Figure \ref{fig:contract} shows an example of vertex contraction: vertices \(a\) and \(b\) are contracted into a
single vertex.  Since edges \(ae\) and \(be\) would result in multiple edges between \(a\) and \(e\), one of the
edges is discarded.  Edges \(bc\) and \(bd\) also become incident to the single vertex.

\begin{figure}[h]
  \label{fig:contract}
  \begin{minipage}{3in}
    \begin{center}
      \begin{tikzpicture}[every node/.style={labeled node}]
        \cycleVnodes{\(a\),\(b\),\(c\),\(d\),\(e\)}{(0,0)}{1in}{90}{}
        \draw (1) edge (5);
        \draw (2) edge (3) edge (4) edge (5);
        \draw (3) edge (4);
        \draw (4) edge (5);
        \draw [dashed,red,->] (2) edge (1);
      \end{tikzpicture}

      \bigskip

      \(G\)
    \end{center}
  \end{minipage}
  \begin{minipage}{3in}
    \begin{center} 
      \begin{tikzpicture}
        \begin{scope}[every node/.style={coordinate}]
          \cycleNnodes{5}{(0,0)}{1in}{90}{}
        \end{scope}
        \begin{scope}[every node/.style={labeled node}]
          \node (AB) at (1) {\(ab\)};
          \node (C) at (3) {\(c\)};
          \node (D) at (4) {\(d\)};
          \node (E) at (5) {\(e\)};
        \end{scope}
        \draw (AB) edge (C) edge (D) edge (E);
        \draw (C) edge (D);
        \draw (D) edge (E);
      \end{tikzpicture}

      \bigskip

      \(G\cdot ab\)
    \end{center}
  \end{minipage}
  \caption{Vertex Contraction Example}
\end{figure}

For the operation \(G\cdot uv\), if \(uv\in E(G)\) then the operation is also referred to as \emph{edge
  contraction}.  If \(uv\notin E(G)\) then the operation is also referred to as \emph{vertex identification}.  The
proposed algorithm uses vertex identification to consolidate two non-adjacent FRs into the same part.

\subsection{Connected Graphs}

The edges of a graph suggest the ability to ``walk'' from one vertex to another along the edges.  A graph where this
is possible for any two vertices is called a \emph{connected} graph.  The concept of connectedness is an
important topic in graph theory; however, an ideal coloring algorithm should work regardless of the connected
nature of an input graph.  The concept of connectedness and how it impacts coloring is described in this section.

\subsubsection{Walks}

The undirected edges in a simple graph suggest bidirectional connectivity between their endpoint vertices.  This
leads to the idea of ``traveling'' between two vertices in a graph by following the edges joining intermediate
adjacent vertices.  Such a journey is referred to as a \emph{walk}:

\begin{definition}[Walk]
  A \(u-v\) \emph{walk} \(W\) in a graph \(G\) is a finite sequence of vertices \(w_i\in V(G)\) starting with
  \(u=w_0\) and ending with \(v=w_k\):
  \[W=(u=w_0,w_1,\ldots,w_k=v)\]
  such that \(w_iw_{i+1}\in E(G)\) for \(0\le i<k\).

  To say that \(W\) is \emph{open} means that \(u\ne v\).  To say that \(W\) is \emph{closed} means that \(u=v\).
  The \emph{length} \(k\) of \(W\) is the number of edges traversed: \(k=\abs{W}\).

  A \emph{trivial} walk is a walk of zero length --- i.e, a single vertex: \(W=(u)\).
\end{definition}

The bidirectional nature of the edges in a simple graph suggests the following proposition:

\begin{proposition}
  Let \(G\) be a graph and let \(u-v\) be a walk of length \(k\) in \(G\).  \(G\) contains a \(v-u\) walk of length
  \(k\) in \(G\) by traversing \(u-v\) in the opposite direction.
\end{proposition}

An example of two walks of length \(4\) is shown in Figure \ref{fig:walks}.  Note that \(W_1\) is an open walk
because it starts and ends on distinct vertices, whereas \(W_2\) is a closed walk because it starts and ends on the
same vertex.

\begin{figure}[h]
  \label{fig:walks}
  \begin{minipage}{3in}
    \begin{center}
      \begin{tikzpicture}[every node/.style={labeled node}]
        \cycleVnodes{\(a\),\(b\),\(c\),\(d\),\(e\)}{(0,0)}{0.75in}{90}{}
        \draw (1) edge (2) edge (3) edge (4) edge (5);
        \draw (2) edge (3) edge (5);
        \draw (3) edge (4);
      \end{tikzpicture}
    \end{center}
  \end{minipage}
  \begin{minipage}{3in}
    \(W_1=(a,b,e,a,c)\ \text{is open}\)

    \(W_2=(a,e,b,c,a)\ \text{is closed}\)

    \bigskip

    \(\abs{W_1}=\abs{W_2}=4\)
  \end{minipage}
  \caption{Open and Closed Walks in a Graph}
\end{figure}

Note that in the general case, vertices and edges are allowed to be repeated during a walk.  Certain special walks
can be defined by restricting such repeats:

\begin{definition}[Special Walks]
  \begin{tabular}{lll}
    \emph{trail} & An open walk with no repeating edges & \((a,b,c,a,e)\) \\
    \\
    \emph{path} & A trail with no repeating vertices & \((a,e,b,c)\) \\
    \\
    \emph{circuit} & A closed trail & \((a,b,e,a,c,d,a)\) \\
    \\
    \emph{cycle} & A closed path & \((a,e,b,c,a)\)
  \end{tabular}
\end{definition}

The example special walks stated above refer to the graph in Figure \ref{fig:walks}.

When discussing the connectedness of a graph, the main concern is the existence of paths between vertices:

\begin{definition}[Connected Vertices]
  Let \(G\) be a graph and let \(u,v\in V(G)\).  To say that \(u\) and \(v\) are \emph{connected} means that \(G\)
  contains a \(u-v\) path.
\end{definition}

But if there exists a \(u-v\) walk in a graph \(G\), does this also mean that there exists a \(u-v\) path in \(G\)
--- i.e. a walk with no repeating edges or vertices?  The answer is yes, as shown by the following theorem:

\begin{theorem}
  Let \(G\) be a graph and let \(u,v\in V(G)\).  If \(G\) contains a \(u-v\) walk of length \(k\) then \(G\)
  contains a \(u-v\) path of length \(\ell\le k\).
\end{theorem}

\begin{proof}
  Assume that \(G\) contains at lease one \(u-v\) walk of length \(k\) and consider the set of all possible \(u-v\)
  walks in \(G\); their lengths form a non-empty set of positive integers.  By the well-ordering principle, there
  exists a \(u-v\) walk \(P\) of minimal length \(\ell\le k\):
  \[P=(u=w_0,\ldots,w_{\ell}=v)\]
  We claim that \(P\) is a path.

  Assume by way of contradiction that \(P\) is not a path, and thus \(P\) has at least one repeating vertex.  Let
  \(w_i=w_j\) for some \(0\le i<j\le\ell\) be such a repeating vertex.  There are two possibilities:
  
  \begin{description}
  \item Case 1: The walk ends on a repeated vertex (\(j=\ell\)).  This is demonstrated in Figure \ref{fig:rend}.

    \begin{figure}[h]
      \label{fig:rend}
      \begin{center}
        \begin{tikzpicture}
          \colorlet{cin}{green}
          \begin{scope}[every node/.style={unlabeled node},node distance=1in]
            \node (w0) at (0,0) {};
            \node (w1) [right=of w0] {};
            \node (wi) [right=of w1] {};
            \node (wi1) [right=of wi] {};
            \node (wj2) [right=of wi1] {};
            \node (wj1) [below=of wi] {};
          \end{scope}
          \draw [cin] (w0) edge (w1);
          \draw [dashed,cin] (w1) edge (wi);
          \draw (wi) edge (wi1);
          \draw [dashed] (wi1) edge (wj2);
          \draw (wj2) edge (wj1);
          \draw (wj1) edge (wi);
          \node [above] at (w0) {\(u=w_0\)};
          \node [above] at (w1) {\(w_1\)};
          \node [above] at (wi) {\(w_i=w_j=w_{\ell}=v\)};
          \node [above] at (wi1) {\(w_{i+1}\)};
          \node [above] at (wj2) {\(w_{j-2}\)};
          \node [below] at (wj1) {\(w_{j-1}\)};
        \end{tikzpicture}
      \end{center}
      \caption{Repeated Vertex at End Case}
    \end{figure}

    Let \(P'=(u=w_0,w_1,\ldots,w_i=v)\) be the walk shown in green in the figure.  \(P'\) is a \(u-v\) walk of
    length \(i<\ell\) in \(G\).

  \item Case 2: A repeated vertex occurs inside the walk (\(j<\ell\)).  This is demonstrated in Figure
    \ref{fig:rmiddle}.
    
    \begin{figure}[h]
      \label{fig:rmiddle}
      \begin{center}
        \begin{tikzpicture}
          \colorlet{cin}{green}
          \begin{scope}[every node/.style={unlabeled node},node distance=1in]
            \node (w0) at (0,0) {};
            \node (w1) [right=of w0] {};
            \node (wi) [right=of w1] {};
            \node (wip1) [below right=of wi] {};
            \node (wjm1) [below left=of wi] {};
            \node (wjp1) [right=of wi] {};
            \node (wl) [right=of wjp1] {};
          \end{scope}
          \draw [cin] (w0) edge (w1);
          \draw [dotted,cin] (w1) edge (wi);
          \draw (wi) edge (wip1);
          \draw [dotted] (wip1) edge (wjm1);
          \draw (wjm1) edge (wi);
          \draw [cin] (wi) edge (wjp1);
          \draw [dotted,cin] (wjp1) edge (wl);
          \draw [dotted,cin] (wjp1) edge (wl);
          \node [above] at (w0) {\(u=w_0\)};
          \node [above] at (w1) {\(w_1\)};
          \node [above] at (wi) {\(w_i=w_j\)};
          \node [below] at (wip1) {\(w_{i+1}\)};
          \node [below] at (wjm1) {\(w_{j-1}\)};
          \node [above] at (wjp1) {\(w_{j+1}\)};
          \node [above] at (wl) {\(w_{\ell}=v\)};
        \end{tikzpicture}
      \end{center}
      \caption{Repeated Vertex Inside Case}
    \end{figure}

  Let \(P'=(u=w_0,w_1,\ldots,w_i,w_{j+1},\ldots,w_{\ell}=v)\) be the walk shown in green in the figure.  \(P'\) is
  a \(u-v\) walk of length \(\ell-(j-i)<\ell\) in \(G\).
  \end{description}

  Both cases contradict the minimality of the length of \(P\).

  \(\therefore P\) is a \(u-v\) path of length \(\ell\le k\) in \(G\).
\end{proof}

\subsubsection{Connected}

A \emph{connected} graph is a graph whose vertices are all connected:

\begin{definition}[Connected Graph]
  To say that a graph \(G\) is \emph{connected} means that for all \(u,v\in V(G)\) there exists a \(u-v\) path.
  Otherwise, \(G\) is said to be \emph{disconnected}.
\end{definition}

Examples of connected and disconnected graphs are shown in figure \ref{fig:connect}.

\begin{figure}[h]
  \label{fig:connect}
  \begin{minipage}[t]{2.5in}
    \begin{center}
      \begin{tikzpicture}[every node/.style={labeled node}]
        \cycleVnodes{\(a\),\(b\),\(c\),\(d\)}{(0,0)}{0.75in}{135}{}
        \draw (1) edge (2) edge (3) edge (4);
      \end{tikzpicture}

      \bigskip

      \begin{tabular}{c}
        \((a,b)\) \\
        \((a,c)\) \\
        \((a,d)\) \\
        \((b,a,c)\) \\
        \((b,a,d)\) \\
        \((c,a,d)\)
      \end{tabular}

      \bigskip
      
      CONNECTED
    \end{center}
  \end{minipage}
  \begin{minipage}[t]{3in}
    \begin{center}
      \begin{tikzpicture}[every node/.style={labeled node},node distance=2cm]
        \cycleVnodes{\(a\),\(b\),\(c\)}{(0,0)}{0.75in}{90}{l}
        \draw (l1) edge (l2) edge (l3);
        \draw (l2) edge (l3);
        \pathVnodes{\(d\),\(e\)}{(3,0.5)}{right}{r};
        \draw (r1) edge (r2);
      \end{tikzpicture}

      \bigskip
      
      No path from any of \(a,b,c\) to any of \(d,e\)

      \bigskip

      DISCONNECTED
    \end{center}
  \end{minipage}
  \caption{Connected and Disconnected Graphs}
\end{figure}

By definition, the trivial graph is connected since the single vertex is connected to itself by a trivial path (of
length \(0\)).

\subsubsection{Components}

It would seem that a disconnected graph is composed of some number of connected subgraphs that partition the
graph's vertex set under a connected equivalence relation.  Each such subgraph is called a \emph{component} of the
graph:

\begin{definition}[Component]
  Let \(G\) be a graph and let \(\SG\) be the set of all connected subgraphs of \(G\).  To say that a graph
  \(H\in\SG\) is a \emph{component} of a \(G\) means that \(H\) is not a subgraph of any other connected subgraph
  of \(\SG\):
  \[\forall\,F\in\SG-\set{H},H\not\subset F\]
  The number of distinct components in \(G\) is denoted by:
  \[k=k(G)\]
  For a connected graph: \(k(G)=1\).
\end{definition}

Each component of a graph \(G\) is denoted by \(G_i\) where \(1\le i\le k(G)\).  We also use union notation to
denote that \(G\) is composed of its component parts:
\[G=\bigcup_{0\le i\le k(G)}G_i\]
Furthermore the \(G_i\) are induced by the vertex equivalence classes of the connectedness relation:

\begin{theorem}
  Let \(G\) be a graph with component \(G_i\).  \(G_i\) is an induced subgraph of \(G\).
\end{theorem}

\begin{proof}
  By definition, \(G_i\) is a maximal connected subgraph of \(G\).  So assume by way of contradiction that
  \(G_i\) is not an induced subgraph of \(G\).  Thus, \(G_i\) is missing some edges that when added would result in a
  connected induced subgraph \(H\) of \(G\).  But then \(G_i\subset H\), contradicting the maximality of \(G_i\).

  \(\therefore G_i\) is an induced subgraph of \(G\).
\end{proof}

\subsubsection{Impact on Coloring}

The impact of disconnectedness on coloring depends on the selected algorithm.  One might assume that the selected
algorithm should be run on each component individually in order to determine each \(\X(G_i)\) and then use Proposition
\ref{prop:coloring2} to conclude that the maximum such value is sufficient for \(\X(G)\):
\[\X(G)=\max_{1\le i\le k(G)}\X(G_i)\]
For example, consider the disconnected graph in Figure \ref{fig:connect}.  The graph contains two components, so
number the components from left-to-right:
\begin{gather*}
  \X(G_1)=3 \\
  \X(G_2)=2 \\
  \\
  \X(G)=\max\{3,2\}=3
\end{gather*}

Using this technique requires application of an initial algorithm to partition the graph into components.  Such an
algorithm is well-known and is described by Hopcroft and Tarjan, 1973 \cite{hopcroft}.  The algorithm is recursive.
It starts by pushing a randomly selected vertex on the stack and walking the vertex's incident edges, removing each
edge as it is traversed.  As each unmarked vertex is encountered, it is assigned to the current component.
Vertices with incident edges are pushed onto the stack and newly isolated vertices are popped off the stack.  Once
the stack is empty, any previously unmarked vertex is selected to start the next component and the process
continues until all vertices are marked.  This algorithm has a runtime complexity of \(\BO(\max(n,m))\).

Alternatively, an ideal coloring algorithm could be run on the entire graph at once regardless of the number of
components in the graph.  The proposed algorithm is such a solution, and therefore saves the needless work of
partitioning the graph into components first.

\subsection{Vertex Degree}

Besides a graph's order and size, the next most important parameter is the so-called \emph{degree} of each vertex.
In order to define the degree of a vertex, we need to define what is meant by a vertex's \emph{neighborhood} first:

\begin{definition}[Neighbor]
  Let \(G\) be a graph and let \(u,v\in V(G)\).  To say that \(u\) is a \emph{neighbor} of \(v\) (and vice-versa)
  means that \(uv\in E(G)\).
\end{definition}

Thus, neighbor vertices are adjacent.  Note that for simple graphs, a vertex is never a neighbor of itself.

The set of all neighbors for a vertex is referred to as the vertex's neighborhood:

\begin{definition}[Neighborhood]
  Let \(G\) be a graph and let \(u\in V(G)\).  The \emph{neighborhood} of \(u\), denoted by \(N(u)\), is the set of
  all neighbors of \(u\) in \(G\):
  \[N(u)=\setb{v\in V(G)}{uv\in E(G)}\]
\end{definition}

The degree of a vertex is then defined to be the cardinality of its neighborhood:

\begin{definition}[Degree]
  Let \(G\) be a graph and let \(u\in G\).  The \emph{degree} of \(u\), denoted by \(\deg_G(u)\) or \(\deg(u)\), is
  the cardinality of the neighborhood of \(u\):
  \[\deg(u)=\abs{N(u)}\]
\end{definition}

Note that the degree of a vertex can be viewed as the number of neighbor vertices or the number of incident edges.

When considering the degrees of all the vertices in a graph, the following limits are helpful:

\begin{notation}
  Let \(G\) be a graph:
  \begin{gather*}
    \d(G)=\min_{v\in V(G)}\deg(v) \\
    \D(G)=\max_{v\in V(G)}\deg(v)
  \end{gather*}
\end{notation}

And so, for a graph \(G\) or order \(n\), it must be the case that for every vertex \(v\in G\):
\[0\le\d(G)\le\deg(v)\le\D(G)\le n-1\]

Intuitively, as \(\d(G)\) increases, a graph becomes denser (more edges) resulting in more adjacencies, making it
harder to find a proper coloring at lower values of \(k\).

Vertices can be classified based on their degree:

\begin{definition}[Vertex Types]
  Let \(G\) be a graph of order \(n\) and let \(u\in V(G)\):
  \begin{quote}
    \begin{tabular}{|c|l|}
      \hline
      \(\deg(u)\) & TYPE \\
      \hline
      \(0\) & isolated \\
      \(1\) & pendant, end, leaf \\
      \(n-1\) & universal \\
      even & even \\
      odd & odd \\
      \hline
    \end{tabular}
  \end{quote}
\end{definition}

The degrees of the vertices in a graph and the number of edges in the graph are related by the so-called First
Theorem of Graph Theory:

\begin{theorem}[First Theorem of Graph Theory]
  Let \(G\) be a graph of size \(m\):
  \[\sum_{v\in V(G)}\deg(v)=2m\]
\end{theorem}

\begin{proof}
  When summing all the degrees, each edge is counted twice: once for each endpoint.
\end{proof}

These concepts are demonstrated by the graph in Figure \ref{fig:degree}; note that the sum of the vertex degrees is
30, which is twice the number of edges in the graph.

\begin{figure}[h]
  \label{fig:degree}
  \begin{minipage}[t]{3in}
    \begin{center}
      \vspace{0pt}
      \begin{tikzpicture}[every node/.style={labeled node}]
        \cycleV{\(v_2\),\(v_3\),\(v_4\),\(v_5\),\(v_6\),\(v_7\)}{(0,0)}{1in}{0}{c};
        \node (v1) at (0,0) {\(v_1\)};
        \node (v8) at (150:1.25in) {\(v8\)};
        \foreach \i in {c1,c2,c3,c4,c5,c6}{
          \draw (v1) edge (\i);
        }
        \draw (c1) edge (c5);
        \draw (c2) edge (c4);
        \draw (v1) edge (v8);
      \end{tikzpicture}

      \bigskip

      \begin{tabular}{ll}
        \(n=8\) & \(m=15=\frac{30}{2}\) \\
        \(\d(G)=1\) & \(\D(G)=7\) \\
        \(\diam(G)=2\)
      \end{tabular}
    \end{center}
  \end{minipage}
  \begin{minipage}[t]{3in}
    \vspace{0pt}
    \begin{tabular}{c|c|l}
      vertex & degree & type \\
      \hline
      \(v_1\) & 7 & universal,odd \\
      \(v_2\) & 4 & even \\
      \(v_3\) & 4 & even \\
      \(v_4\) & 3 & odd \\
      \(v_5\) & 4 & even \\
      \(v_6\) & 4 & even \\
      \(v_7\) & 3 & odd \\
      \(v_8\) & 1 & pendant,odd \\
      \hline
      total & 30 &
    \end{tabular}
  \end{minipage}
  \caption{Vertex Degrees}
\end{figure}

\subsection{Special Graphs}

We conclude this introductory section on graph theory with a discussion of some special classes of graphs that are
important to the execution of the proposed algorithm.

\subsubsection{Empty Graphs}

An \emph{empty} graph of order \(n\), denoted by \(E_n\), is a graph with one or more vertices (\(n>1\)) and no
edges (\(m=0\)).  An empty graph is connected iff \(n=1\).  Examples of empty graphs are shown in Figure
\ref{fig:empty}.

\begin{figure}[h]
  \label{fig:empty}
  \begin{minipage}{1.5in}
    \begin{center}
      \begin{tikzpicture}[every node/.style=unlabeled node]
        \node at (0,0) {};
      \end{tikzpicture}

      \bigskip

      \(E_1\)
    \end{center}
  \end{minipage}
  \begin{minipage}{2.5in}
    \begin{center}
      \begin{tikzpicture}[every node/.style=unlabeled node]
        \pathNnodes{4}{(0,0)}{right}{};
      \end{tikzpicture}

      \bigskip

      \(E_4\)
    \end{center}
  \end{minipage}
  \begin{minipage}{2in}
    \begin{center}
      \begin{tikzpicture}[every node/.style=unlabeled node]
        \pathNnodes{3}{(0,0)}{right}{};
        \pathNnodes{3}{(0,1)}{right}{};
        \pathNnodes{3}{(0,2)}{right}{};
      \end{tikzpicture}

      \bigskip

      \(E_9\)
    \end{center}
  \end{minipage}
  \caption{Empty Graphs}
\end{figure}

The null graph (\(n=0\)) is denoted by \(E_0\) and is defined to be \chromatic{0}.  All other empty graphs are
\chromatic{1} and thus are important termination conditions for the proposed algorithm.

\subsubsection{Paths}

A \emph{path} graph of order \(n\) and length \(n-1\), denoted by \(P_n\), is a connected graph consisting of a
single open path.  Examples of path graphs are shown in Figure \ref{fig:path}.

\begin{figure}[h]
  \label{fig:path}
  \begin{minipage}{1.5in}
    \begin{center}
      \begin{tikzpicture}[every node/.style=unlabeled node]
        \node at (0,0) {};
      \end{tikzpicture}

      \bigskip

      \(P_1\)
    \end{center}
  \end{minipage}
  \begin{minipage}{2.5in}
    \begin{center}
      \begin{tikzpicture}[every node/.style=unlabeled node]
        \pathN{4}{(0,0)}{right}{};
      \end{tikzpicture}

      \bigskip

      \(P_4\)
    \end{center}
  \end{minipage}
  \begin{minipage}{2in}
    \begin{center}
      \begin{tikzpicture}[every node/.style=unlabeled node]
        \pathNnodes{3}{(0,0)}{right}{b};
        \pathNnodes{3}{(0,1)}{right}{m};
        \pathNnodes{3}{(0,2)}{right}{t};
        \draw (t1) to (t2) to (t3) to (m3) to (m2) to (m1) to (b1) to (b2) to (b3);
      \end{tikzpicture}

      \bigskip

      \(P_9\)
    \end{center}
  \end{minipage}
  \caption{Path Graphs}
\end{figure}

Note that \(P_1=E_1\) is \chromatic{1}, whereas \(P_n,n>1\) is \chromatic{2}.

Paths are not particularly important to the proposed algorithm; however, they play a part in the definition of
cycles.

\subsubsection{Cycles}

A \emph{cycle} graph of order \(n\) and length \(n\) for \(n\ge3\), denoted by \(C_n\), is a connected graph
consisting of a single closed path.  When \(n\) is odd then \(C_n\) is called an \emph{odd} cycle and when \(n\) is
even then \(C_n\) is called an \emph{even} cycle.

Examples of cycle graphs are shown in Figure \ref{fig:cycle}.

\begin{figure}[h]
  \label{fig:cycle}
  \begin{minipage}{2in}
    \begin{center}
      \begin{tikzpicture}[every node/.style=unlabeled node]
        \cycleN{3}{(0,0)}{0.5in}{90}{};
      \end{tikzpicture}

      \bigskip

      \(C_3\) (odd)
    \end{center}
  \end{minipage}
  \begin{minipage}{2in}
    \begin{center}
      \begin{tikzpicture}[every node/.style=unlabeled node]
        \cycleN{4}{(0,0)}{0.5in}{135}{};
      \end{tikzpicture}

      \bigskip

      \(C_4\) (even)
    \end{center}
  \end{minipage}
  \begin{minipage}{2in}
    \begin{center}
      \begin{tikzpicture}[every node/.style=unlabeled node]
        \cycleN{9}{(0,0)}{0.5in}{90}{};
      \end{tikzpicture}

      \bigskip

      \(C_9\) (odd)
    \end{center}
  \end{minipage}
  \caption{Cycle Graphs}
\end{figure}

Note that even cycles are \(2\)-chromatic; however, odd cycles are \(3\)-chromatic.

Cycles are not particularly important to the proposed algorithm; however, they play a part in the definition of
trees, which are important to the later Zykov analysis of coloring algorithms.

\subsubsection{Complete Graphs}

A \emph{complete} graph of order \(n\) and size \(\frac{n(n-1)}{2}\), denoted by \(K_n\), is a connected graph that
contains every possible edge:
\[E(G)=\ps_2(V(G))\]
and thus all \(n\) vertices are adjacent to each other.

Examples of complete graphs are shown in Figure \ref{fig:complete}.

\begin{figure}[h]
  \label{fig:complete}
  \begin{minipage}{2in}
    \begin{center}
      \begin{tikzpicture}[every node/.style=unlabeled node]
        \node at (0,0) {};
      \end{tikzpicture}

      \bigskip

      \(K_1\)
    \end{center}
  \end{minipage}
  \begin{minipage}{2in}
    \begin{center}
      \begin{tikzpicture}[every node/.style=unlabeled node]
        \completeN{4}{(0,0)}{0.5in}{135}{};
      \end{tikzpicture}

      \bigskip

      \(K_4\)
    \end{center}
  \end{minipage}
  \begin{minipage}{2in}
    \begin{center}
      \begin{tikzpicture}[every node/.style=unlabeled node]
        \completeN{9}{(0,0)}{0.5in}{90}{};
      \end{tikzpicture}

      \bigskip

      \(K_9\)
    \end{center}
  \end{minipage}
  \caption{Complete Graphs}
\end{figure}

Note that \(K_1=P_1=E_1\).

Since all of the vertices in a complete graph are adjacent to each other, each vertex requires a separate color in
order to achieve a proper coloring.  Thus, \(K_n\) is \(n\)-chromatic and is also an important termination
condition for the proposed algorithm.

\subsubsection{Trees}

A \emph{tree} is a connected graph that contains no cycles as subgraphs.  Typically, one vertex of the tree is
selected as the \emph{root} vertex and then the tree is depicted in layers that contain vertices that are
equidistant from the root vertex.  Note that the bottom layer is composed entirely of pendant vertices, but pendant
vertices can exist in the other layers as well.  Such pendant vertices are usually referred to as \emph{leaves} in
this context.

An example tree is shown in Figure \ref{fig:tree}.  The root vertex \(r\) is shown in red and the leaf vertices
\(b,e,g,h,i,j,k\) are shown in green.

\begin{figure}
  \label{fig:tree}
  \begin{center}
    \begin{tikzpicture}[every node/.style={labeled node}]
      \colorlet{cr}{red!25!white}
      \colorlet{cl}{green!25!white}
      \node [fill=cr] (r) at (0,0) {\(r\)};
      \node (v11) at (-3,-2) {\(a\)};
      \node [fill=cl] (v12) at (0,-2) {\(b\)};
      \node (v13) at (3,-2) {\(c\)};
      \node (v21) at (-4.5,-4) {\(d\)};
      \node [fill=cl] (v22) at (-3,-4) {\(e\)};
      \node (v23) at (-1.5,-4) {\(f\)};
      \node [fill=cl] (v24) at (1.75,-4) {\(g\)};
      \node [fill=cl] (v25) at (4.25,-4) {\(h\)};
      \node [fill=cl] (v31) at (-4.5,-6) {\(i\)};
      \node [fill=cl] (v32) at (-2.25,-6) {\(j\)};
      \node [fill=cl] (v33) at (-0.75,-6) {\(k\)};
      \draw (r) edge (v11) edge (v12) edge (v13);
      \draw (v11) edge (v21) edge (v22) edge (v23);
      \draw (v21) edge (v31);
      \draw (v23) edge (v32) edge (v33);
      \draw (v13) edge (v24) edge (v25);
    \end{tikzpicture}
  \end{center}
  \caption{A Tree Organized from Root to Leaves}
\end{figure}

Trees are important because they can be used to represent the ``paper tape'' for recursive Turing machine-type
algorithms.  Each vertex represents a state of the problem and the edges represent ``movement'' of the tape to
select the current state.  All states can be visited using a so-called \emph{depth-first} walk.  In the example
in Figure \ref{fig:tree}, such a depth-first walk would be:
\[(r,a,d,i,d,a,e,a,f,j,f,k,f,a,r,b,r,c,g,c,h,c,r)\]
Note that this walk guarantees that each vertex is visited at least once.

When such a tree is applied to the problem of exhaustively finding the chromatic number of a graph, the tree is
called a \emph{Zykov} tree.  These concepts are described in detail in the next section.

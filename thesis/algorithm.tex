\section{The Proposed Algorithm}

The exhaustive algorithm described in the previous section subjects each pair of non-adjacent vertices in a graph
\(G\) to the two choices of like (vertex contraction) or different (edge addition) color assignment in a recursive
manner.  The leaves of the resulting Zykov tree that tracks these choices represent complete graphs that describe
all of the possible \(k\) colorings of \(G\), with the complete graphs of smallest order representing chromatic
colorings of \(G\).

The major advantages of the exhaustive algorithm are that it isn't dependent on the structure of the graph (e.g.,
connectedness), an example of a chromatic coloring is readily available, and the fact that the algorithm, due to
its Turing machine nature, can be coded rather easily to run on a computer.  Its major disadvantage is its high
runtime complexity due to the need to generate the entire Zykov tree using an exponentially growing number of
recursive calls.

Thus, the goals of the proposed algorithm are as follows:
\begin{enumerate}
\item It does not depend on the structure of a graph.
\item An example of a chromatic coloring should be easily available.
\item It can be easily coded for execution on a computer.
\item It has significantly better runtime complexity than the exhaustive algorithm.
\end{enumerate}

To accomplish these goals, the proposed algorithm takes a somewhat different tack: it loops on successively higher
values of \(k\).  For each candidate \(k\) value, a graph is assumed to be \colorable{k} and a modified version of
the exhaustive algorithm is executed to either prove or disprove this assumption.  Since a candidate \(k\) value is
known, certain reversible steps can be applied to mutate \(G\) into simpler graphs with the same colorability and
test for very early termination of the current Zykov tree.  The first \(k\) for which \(G\) (or one of its
simplifications) is found to be \(k\)-colorable is the chromatic number of \(G\).  As will be shown, the tradeoff
of looping on \(k\) and very shallow execution of each corresponding Zykov tree far outweighs the need to generate
an entire Zykov tree, resulting in a much better runtime complexity.

One slight disadvantage of the proposed algorithm is that whereas the exhaustive algorithm readily provides
examples of actual chromatic colorings, the proposed algorithm requires a reverse traversal of its reversible steps
in order to construct such a coloring.  Nevertheless, this technique still has the advantage over common coloring
algorithms such as greedy coloring because it is not heuristic.  However, as was stated earlier, during the
axiomatic design phase it is more important to know the minimum number of parts as opposed to actual functional
requirement allocation to those parts.  Thus, a little effort to reconstruct a chromatic coloring after the fact is
not of major importance.

This algorithm was first proposed by the author and his advisor in collaboration with a team of mechanical
engineering researchers from SUNY Buffalo \cite{cavallaro}.  It accepts a graph \(G\) as input, provides \(\X(G)\)
as output, and is composed of an outer loop on values of \(k\) and a subroutine called by the outer loop to
determine if \(G\) is \colorable{k}.  The outer loop and called subroutine are summarized in the following
sections.  A complete description of the theorems that support the various steps in algorithm and the application
of the algorithm to a sample graph then follow.

\subsection{Outer Loop}

The outer loop accepts a graph \(G\) as input and returns \(\X(G)\).  It initially checks for some degenerate cases
and then loops on increasing values of \(k\).  For each value of \(k\) the called subroutine executes the modified
exhaustive algorithm to determine if \(G\) is \colorable{k}.  The first such successful return identifies
\(\X(G)\).

The steps of the outer loop are as follows:

\begin{enumerate}
\item \label{step:null} If \(n=0\) then return \(0\), thus handling the degenerate case of a null graph.

\item \label{step:one} If \(m=0\) then return \(1\), thus handling the degenerate case of an empty graph.

\item \label{step:init} Initialize \(k\) to \(2\).

\item \label{step:inner} Call the subroutine to determine if \(G\) is \colorable{k}.  The subroutine returns a
  possibly simplified \(G\) called \(G'\) and a boolean value \(R\) that reports the result of the test.

\item \label{step:call} If \(G\) is \colorable{k} (\(R=\) true) then return \(k\).

\item \label{step:newg} Otherwise, replace \(G\) with \(G'\).  As will be seen, doing this avoids needless
  reapplication of certain steps in the called subroutine.

\item \label{step:incr} Increment \(k\).

\item \label{step:loop} Go to step \ref{step:inner}.
\end{enumerate}

A flowchart of these steps is shown in Figure \ref{fig:outer}.

\begin{figure}[h]
  \label{fig:outer}
  \begin{center}
    \scalebox{0.75}{
      \begin{tikzpicture}[>=latex']
        \node (start) [draw,terminal] at (0,0) {START};
        \node (nullcheck) [draw,decision,below=of start] {\(n=0\)?};
        \node (nulldone) [draw,terminal,right=of nullcheck] {RETURN \(0\)};
        \node (emptycheck) [draw,decision,below=of nullcheck] {\(m=0\)?};
        \node (emptydone) [draw,terminal,right=of emptycheck] {RETURN \(1\)};
        \node (kinit) [draw,process,below=of emptycheck] {\(k=2\)};
        \node (kcheck) [draw,predproc,below=of kinit] {CALL \(G,k\)};
        \node (iskcolor) [draw,decision,below=of kcheck] {\(R=\)TRUE?};
        \node (kdone) [draw,terminal,right=of iskcolor] {RETURN \(k\)};
        \node (newg) [draw,process,below=of iskcolor] {\(G=G'\)};
        \node (kinc) [draw,process,below=of newg] {\(k=k+1\)};
        \node (belowinc) [coordinate,below=0.5cm of kinc] {};
        \node (leftinc) [coordinate,left=3cm of belowinc] {};
        \draw [->] (start) -- node [auto] {\(G\)} (nullcheck);
        \draw [->] (nullcheck) -- node [auto] {YES} (nulldone);
        \draw [->] (nullcheck) -- node [auto] {NO} (emptycheck);
        \draw [->] (emptycheck) -- node [auto] {YES} (emptydone);
        \draw [->] (emptycheck) -- node [auto] {NO} (kinit);
        \draw [->] (kinit) -- (kcheck);
        \draw [->] (kcheck) -- node [auto] {\(G',R\)} (iskcolor);
        \draw [->] (iskcolor) -- node [auto] {YES} (kdone);
        \draw [->] (iskcolor) -- node [auto] {NO} (newg);
        \draw [->] (newg) -- (kinc);
        \draw [->] (kinc) -- (belowinc) -- (leftinc) |- (kcheck);
      \end{tikzpicture}
    }
  \end{center}
  \caption{Algorithm Outer Loop}
\end{figure}

Note that the outer loop is guaranteed to complete because \(k\) will eventually be greater than or equal to \(n\),
and thus by Proposition \ref{prop:coloring3}, the current state of \(G\) is at least \colorable{k}, causing the
called subroutine to return true.

\subsection{Called Subroutine}

The called subroutine executes a modified version of the exhaustive algorithm that determines whether a graph is
\colorable{k}.  It accepts the current state of \(G\) of order \(n\) and size \(m\) and the current value of
\(k\ge2\) as inputs.  It returns a possibly simplified version of \(G\) and a boolean value indicating whether or
not \(G\) is \colorable{k}.  Internally, various tests are applied to trim the corresponding Zykov tree or abandon
it all together based on the current value of \(k\).

The steps of the called subroutine and references to their associated theorems are as follows:

\begin{enumerate}
\item \label{step:check} If \(n\le k\) then return true (Proposition \ref{prop:coloring3}).

\item \label{step:dencalc} Calculate a maximum edge threshold:
  \[a=\frac{n^2(k-1)}{2k}\]

\item \label{step:density} If \(m>a\) then return false (Corollary \ref{cor:density}).

\item \label{step:smallcalc} Construct the set \(X\) of all vertices with degree less than \(k\):
  \[X=\setb{v\in V(G)}{\deg(v)<k}\]

\item \label{step:small} \(X\ne\emptyset\) then replace \(G\) with \(G-X\) and go to step \ref{step:check}
  (Corollary \ref{cor:lowdeg}).

\item \label{step:neighbor} If \(G\) has vertices \(u\) and \(v\) such that \(N(u)\subseteq N(v)\) then replace
  \(G\) with \(G-u\) and go to step \ref{step:check} (Theorem \ref{thm:subset}).

\item \label{step:select} Select two vertices \(u,v\in V(G)\) with the smallest number of common neighbors and let
  \[b=\abs{N(u)\cap N(v)}\]

\item \label{step:neighcalc} Calculate an upper bound for the minimum number of common neighbors for all vertices
  in \(G\):
  \[c=n-2-\frac{n-2}{k-1}\]

\item \label{step:common} If \(b>c\) then return false (Corollary \ref{cor:inter}).

\item \label{step:select2} Select two non-adjacent vertices \(u,v\in V(G)\) with the smallest number of common
  neighbors.  It will be shown below that such a pair of vertices is guaranteed to exist in the current state of
  \(G\).

\item \label{step:call1} Assume that \(u\) and \(v\) are assigned the same color by letting \(G'=G\cdot uv\).
  Recursively call this routine to see if \(G'\) is \colorable{k}.  If so, then return true (Theorem
  \ref{thm:recurse}).

\item \label{step:call2} Assume that \(u\) and \(v\) are assigned different colors by letting \(G'=G+uv\).
  Recursively call this routine to see if \(G'\) is \colorable{k}.  If so, then return true (Theorem
  \ref{thm:recurse}).

\item \label{step:fail} Since neither of the assumptions in steps \ref{step:call1} and \ref{step:call2} hold,
  conclude that \(G\) is not \(k\)-colorable and return false.
\end{enumerate}

A flowchart of these steps is shown in Figure \ref{fig:called}.

\begin{figure}[h]
  \label{fig:called}
  \begin{center}
    \scalebox{0.7}{
      \begin{tikzpicture}[>=latex']
        \node (start) [draw,terminal] at (0,0) {START};
        \node (donecheck) [draw,decision,below=of start] {\(n\le k\)?};
        \node (done) [draw,terminal,right=of donecheck] {RETURN \(G\),TRUE};
        \node (edgecalc) [draw,process,below=of donecheck] {\(a=\frac{n^2(k-1)}{2k}\)};
        \node (edgecheck) [draw,decision,below=of edgecalc] {\(m>a\)?};
        \node (edgefail) [draw,terminal,right=of edgecheck] {RETURN \(G\),FALSE};
        \node (nodecalc) [draw,process,below=of edgecheck] {\(X=\setb{v\in V(G)}{\deg(v)<k}\)};
        \node (nodecheck) [draw,decision,below=of nodecalc] {\(X\ne\emptyset\)?};
        \node (remnode) [draw,process,left=of nodecheck] {\(G=G-X\)};
        \node (join) [coordinate] at ($(remnode)-(2.5cm,0)$) {};
        \node (subcheck) [draw,decision,below=of nodecheck] {\(N(u)\subseteq N(v)\)?};
        \node (remsub) [draw,process,left=of subcheck] {\(G=G-u\)};
        \node (mininter) [draw,process,below=of subcheck] {\(\displaystyle b=\min_{u,v\in V(G)}\abs{N(u)\cap N(v)}\)};
        \node (intercalc) [draw,process,below=of mininter] {\(c=n-2-\frac{n-2}{k-1}\)};
        \node (intercheck) [draw,decision,below=of intercalc] {\(b>c\)?};
        \node (interfail) [draw,terminal,right=of intercheck] {RETURN \(G\),FALSE};
        \node (finduv) [draw,process,right=2.5cm of done] {\(\displaystyle \min_{uv\notin E(G)}\abs{N(u)\cap N(v)}\)};
        \node (save1) [draw,process,below=of finduv] {\(G'=G\cdot uv\)};
        \node (call1) [draw,predproc,below=of save1] {CALL \(G',k\)};
        \node (check1) [draw,decision,below=of call1] {\(R=\)TRUE?};
        \node (done1) [draw,terminal,right=of check1] {RETURN \(G\),TRUE};
        \node (save2) [draw,process,below=of check1] {\(G'=G+uv\)};
        \node (call2) [draw,predproc,below=of save2] {CALL \(G',k\)};
        \node (check2) [draw,decision,below=of call2] {\(R=\)TRUE?};
        \node (done2) [draw,terminal,right=of check2] {RETURN \(G\),TRUE};
        \node (fail) [draw,terminal,below=of check2] {RETURN \(G\),FALSE};
        \draw [->] (start) -- node [auto] {\(G,k\)} (donecheck);
        \draw [->] (donecheck) -- node [auto] {YES} (done);
        \draw [->] (donecheck) -- node [auto] {NO} (edgecalc);
        \draw [->] (edgecalc) -- (edgecheck);
        \draw [->] (edgecheck) -- node [auto] {YES} (edgefail);
        \draw [->] (edgecheck) -- node [auto] {NO} (nodecalc);
        \draw [->] (nodecalc) -- (nodecheck);
        \draw [->] (nodecheck) -- node [auto] {YES} (remnode);
        \draw [->] (remnode) -- (join) |- (donecheck);
        \draw [->] (nodecheck) -- node [auto] {NO} (subcheck);
        \draw [->] (subcheck) -- node [auto] {YES} (remsub);
        \draw (remsub) -| (join);
        \draw [->] (subcheck) -- node [auto] {NO} (mininter);
        \draw [->] (mininter) -- (intercalc);
        \draw [->] (intercalc) -- (intercheck);
        \draw [->] (intercheck) -- node [auto] {YES} (interfail);
        \draw [->] (intercheck) -- node [auto] {NO} ($(intercheck)-(0,2cm)$) -- ++(7cm,0) |- (finduv);
        \draw [->] (finduv) -- (save1);
        \draw [->] (save1) -- (call1);
        \draw [->] (call1) -- node [auto] {\(G'',R\)} (check1);
        \draw [->] (check1) -- node [auto] {YES} (done1);
        \draw [->] (check1) -- node [auto] {NO} (save2);
        \draw [->] (save2) -- (call2);
        \draw [->] (call2) -- node [auto] {\(G'',R\)} (check2);
        \draw [->] (check2) -- node [auto] {YES} (done2);
        \draw [->] (check2) -- node [auto] {NO} (fail);
      \end{tikzpicture}
    }
  \end{center}
  \caption{Algorithm Called Subroutine}
\end{figure}

Step \ref{step:check} of the called subroutine is the success termination condition.  Success occurs when \(G\) is
simplified by removing sufficent vertices (steps \ref{step:smallcalc}--\ref{step:neighbor}) or when the outer loop
has sufficiently incremented \(k\) (step \ref{step:incr}) such that \(n\le k\).

Steps \ref{step:smallcalc}--\ref{step:neighbor} of the called subroutine attempt to remove vertices to achieve a
simpler graph that is equivalently \colorable{k}.  Each time a vertex is removed, the subtrees associated with that
vertex in the corresponding Zykov tree for \(G\) are ignored.  Since these same steps would just be repeated for
\(k+1\), the subroutine returns the current state of the possibly simplified \(G\) to the outer loop as a starting
point for the next candidate value of \(k\).

Steps \ref{step:dencalc}--\ref{step:density} and \ref{step:select}--\ref{step:common} of the called subroutine
apply tests that attempt to disprove that the current state of \(G\) is \colorable{k} for the current value of
\(k\).  If so, then the current Zykov tree is abandoned and the subroutine returned false.  This allows the outer
loop to continue with \(k+1\).

The remaining steps of the called subroutine constitute the recursive portion of the modified exhaustive algorithm.
The subroutine is guaranteed to return because either there will be sufficient vertex contractions such that \(n\le
k\), resulting in a true return, or sufficient edge additions such that the graph becomes complete and is rejected
by step \ref{step:common}, resulting in a false return.  Note that in the event of a false return, any
modifications to the current state of \(G\) resulting from the recursive calls are not returned to the outer loop.

\clearpage

\subsection{Supporting Theorems}

This section contains the theorems that support the steps in the called subroutine.  Remember that the success check
of step \ref{step:check} is already supported by Proposition \ref{prop:coloring3}.

\subsubsection{Maximum Edge Threshold}

The maximum edge threshold test of steps \ref{step:dencalc} and \ref{step:density} is supported by Theorem
\ref{thm:density}.

\begin{theorem}[Maximum Edge Threshold]
  \label{thm:density}
  Let \(G\) be a graph of order \(n\) and size \(m\) and let \(k\in\N\).  If \(G\) is \colorable{k} then:
  \[m\le\frac{n^2(k-1)}{2k}\]
\end{theorem}

\begin{proof}
  Assume \(G\) is \colorable{k}.  This means that \(V(G)\) can be distributed into \(k\) independent (some possibly
  empty) subsets.  Call these subsets \(A_1,\ldots A_k\) and let \(a_i=\abs*{A_i}\).  Thus, each \(v\in A_i\) can
  be adjacent to at most \(n-a_i\) other vertices in \(G\), and hence the maximum number of edges incident to
  vertices in \(A_i\) is given by: \(a_i(n-a_i)=na_i-a_i^2\).  Now, using Theorem \ref{thm:first}, the maximum
  number of edges in \(G\) is given by:
  \[m\le\frac{1}{2}\sum_{i=1}^k(na_i-a_i^2)\]
  with the constraint:
  \[\sum_{i=1}^ka_i=n\]
  This problem can be solved using the Lagrange multiplier technique.  We start by defining:
  \begin{align*}
    F(a_1,\ldots,a_k) &= f(a_1,\ldots,a_k)-\l g(a_1,\ldots,a_k) \\
    &= \frac{1}{2}\sum_{i=1}^k(na_i-a_i^2)-\l\sum_{i=1}^ka_i \\
    &= \sum_{i=1}^k\left(\frac{1}{2}na_i-\frac{1}{2}a_i^2-\l a_i\right)
  \end{align*}
  Now, optimize by taking the gradient and setting the result equal to the zero vector:
  \[\vec{\nabla}F=\sum_{i=1}^k(\frac{n}{2}-a_i-\l)\hat{a_i}=\vec{0}\]
  This results in a system of \(k\) equations of the form:
  \[\frac{n}{2}-a_i-\l=0\]
  And so:
  \[a_i=\frac{n}{2}-\l\]
  Plugging this result back into the contraint:
  \[\sum_{i=1}^ka_i=\sum_{i=1}^k\left(\frac{n}{2}-\l\right)=k\left(\frac{n}{2}-\l\right)=n\]
  Solving for \(\l\) yields:
  \[\l=\frac{n}{2}-\frac{n}{k}\]
  And finally, to get \(a_i\) in terms of \(n\) and \(k\):
  \[a_i=\frac{n}{2}-\left(\frac{n}{2}-\frac{n}{k}\right)=\frac{n}{k}\]
  Therefore:
  \[m\le\frac{1}{2}\sum_{i=1}^k\left[n\left(\frac{n}{k}\right)-\left(\frac{n}{k}\right)^2\right]=
  \frac{k}{2}\left(\frac{n^2k-n^2}{k^2}\right)=\frac{n^2(k-1)}{2k}\]
\end{proof}

The called subroutine actually uses the contrapositive of this result, as stated in Corollary \ref{cor:density}.

\begin{corollary}
  \label{cor:density}
  Let \(G\) be a graph of order \(n\) and size \(m\) and let \(k\in\N\).  If:
  \[m>\frac{n^2(k-1)}{2k}\]
  then \(G\) is not \colorable{k}.
\end{corollary}

Corollary \ref{cor:density} is demonstrated by Figure \ref{fig:density}.  The shown graph \(G\) has \(n=4\),
\(m=5\), and \(\X(G)=3\).  Testing for \(k=2\):
\[a=\frac{4^2(2-1)}{2\cdot2}=4\]
But \(m=5>4=a\) and so we can conclude that \(G\) is not \colorable{2}.  However, testing for \(k=3\);
\[a=\frac{4^2(3-1)}{2\cdot3}=5.3\]
So \(m=5\ngtr5.3=a\) and thus \(G\) \emph{may} be \(3\)-colorable, since this test only provides a necessary and
not a sufficient condition.

\begin{figure}[h]
  \label{fig:density}
  \begin{center}
    \begin{tikzpicture}
      \colorlet{c1}{green!25!white}
      \colorlet{c2}{blue!25!white}
      \colorlet{c3}{red!25!white}
      \begin{scope}[every node/.style={coordinate}]
        \cycleNnodes{4}{(0,0)}{0.75in}{135}{c};
      \end{scope}
      \begin{scope}[every node/.style={labeled node}]
        \node [fill=c1] (a) at (c1) {\(a\)};
        \node [fill=c2] (b) at (c2) {\(b\)};
        \node [fill=c1] (c) at (c3) {\(c\)};
        \node [fill=c3] (d) at (c4) {\(d\)};
      \end{scope}
      \draw (a) edge (b) edge (d);
      \draw (b) edge (c) edge (d);
      \draw (c) edge (d);
    \end{tikzpicture}

    \bigskip

    \(G\)
  \end{center}
  \caption{Corollary \ref{cor:density} Example}
\end{figure}

In fact, the the test of Corollary \ref{cor:density} will always fail for a complete graph when \(k<n\).  Since
\(k,n>0\):
\begin{gather*}
  \frac{n(n-1)}{2}>\frac{n^2(k-1)}{2k} \\
  kn(n-1)>n^2(k-1) \\
  kn^2-kn>kn^2-n^2 \\
  kn<n^2 \\
  k<n
\end{gather*}

\subsubsection{Vertex Removal}

The theorems that support vertex removal make use of Lemma \ref{lem:remone}.

\begin{lemma}
  \label{lem:remone}
  Let \(G\) be a graph and let \(v\in V(G)\).  If \(G\) is \colorable{k} then \(G-v\) is also \colorable{k}.
\end{lemma}

\begin{proof}
  Assume \(G\) is \colorable{k} and let \(c:V(G)\to C\) be such a coloring.  This means that \(\abs{C}=k\) and for
  all \(uw\in E(G)\) such that \(v\notin uw\) it is the case that \(c(u)=c_1\) and \(c(w)=c_2\) and \(c_1\ne c_2\)
  for some \(c_1,c_2\in C\).

  Now, consider the restricted function \(c'=\restrict{c}{V(G-v)}\).  It is still the case that \(c'(u)=c_1\) and
  \(c'(w)=c_2\) and \(c_1\ne c_2\), so \(c'\) is still proper with \(\abs{C}=k\).

  \(\therefore G-v\) is \colorable{k}.
\end{proof}

Lemma \ref{lem:remone} is demonstrated in Figure \ref{fig:remone}.  No matter which vertex is removed, the resulting
subgraph is still properly colored using at most four colors.

\begin{figure}[h]
  \label{fig:remone}
  \begin{minipage}{1.25in}
    \begin{center}
      \scalebox{0.75}{
        \begin{tikzpicture}
          \colorlet{c1}{green!25!white}
          \colorlet{c2}{blue!25!white}
          \colorlet{c3}{red!25!white}
          \colorlet{c4}{yellow!25!white}
          \begin{scope}[every node/.style={coordinate}]
            \cycleNnodes{4}{(0,0)}{0.5in}{135}{x};
          \end{scope}
          \begin{scope} [every node/.style={labeled node}]
            \node [fill=c1] (v1) at (x1) {\(a\)};
            \node [fill=c2] (v2) at (x2) {\(b\)};
            \node [fill=c3] (v3) at (x3) {\(c\)};
            \node [fill=c4] (v4) at (x4) {\(d\)};
          \end{scope}
          \draw (v1) edge (v2) edge (v3) edge (v4);
          \draw (v2) edge (v3) edge (v4);
          \draw (v3) edge (v4);
        \end{tikzpicture}
      }

      \bigskip

      \(G\)
    \end{center}
  \end{minipage}
  \begin{minipage}{1.25in}
    \begin{center}
      \scalebox{0.75}{
        \begin{tikzpicture}
          \colorlet{c1}{green!25!white}
          \colorlet{c2}{blue!25!white}
          \colorlet{c3}{red!25!white}
          \colorlet{c4}{yellow!25!white}
          \begin{scope}[every node/.style={coordinate}]
            \cycleNnodes{4}{(0,0)}{0.5in}{135}{x};
          \end{scope}
          \begin{scope} [every node/.style={labeled node}]
            \node [fill=c2] (v2) at (x2) {\(b\)};
            \node [fill=c3] (v3) at (x3) {\(c\)};
            \node [fill=c4] (v4) at (x4) {\(d\)};
          \end{scope}
          \draw (v2) edge (v3) edge (v4);
          \draw (v3) edge (v4);
        \end{tikzpicture}
      }

      \bigskip

      \(G-a\)
    \end{center}
  \end{minipage}
  \begin{minipage}{1.25in}
    \begin{center}
      \scalebox{0.75}{
        \begin{tikzpicture}
          \colorlet{c1}{green!25!white}
          \colorlet{c2}{blue!25!white}
          \colorlet{c3}{red!25!white}
          \colorlet{c4}{yellow!25!white}
          \begin{scope}[every node/.style={coordinate}]
            \cycleNnodes{4}{(0,0)}{0.5in}{135}{x};
          \end{scope}
          \begin{scope} [every node/.style={labeled node}]
            \node [fill=c1] (v1) at (x1) {\(a\)};
            \node [fill=c3] (v3) at (x3) {\(c\)};
            \node [fill=c4] (v4) at (x4) {\(d\)};
          \end{scope}
          \draw (v1) edge (v3) edge (v4);
          \draw (v3) edge (v4);
        \end{tikzpicture}
      }

      \bigskip

      \(G-b\)
    \end{center}
  \end{minipage}
  \begin{minipage}{1.25in}
    \begin{center}
      \scalebox{0.75}{
        \begin{tikzpicture}
          \colorlet{c1}{green!25!white}
          \colorlet{c2}{blue!25!white}
          \colorlet{c3}{red!25!white}
          \colorlet{c4}{yellow!25!white}
          \begin{scope}[every node/.style={coordinate}]
            \cycleNnodes{4}{(0,0)}{0.5in}{135}{x};
          \end{scope}
          \begin{scope} [every node/.style={labeled node}]
            \node [fill=c1] (v1) at (x1) {\(a\)};
            \node [fill=c2] (v2) at (x2) {\(b\)};
            \node [fill=c4] (v4) at (x4) {\(d\)};
          \end{scope}
          \draw (v1) edge (v2) edge (v4);
          \draw (v2) edge (v4);
        \end{tikzpicture}
      }

      \bigskip

      \(G-c\)
    \end{center}
  \end{minipage}
  \begin{minipage}{1.25in}
    \begin{center}
      \scalebox{0.75}{
        \begin{tikzpicture}
          \colorlet{c1}{green!25!white}
          \colorlet{c2}{blue!25!white}
          \colorlet{c3}{red!25!white}
          \colorlet{c4}{yellow!25!white}
          \begin{scope}[every node/.style={coordinate}]
            \cycleNnodes{4}{(0,0)}{0.5in}{135}{x};
          \end{scope}
          \begin{scope} [every node/.style={labeled node}]
            \node [fill=c1] (v1) at (x1) {\(a\)};
            \node [fill=c2] (v2) at (x2) {\(b\)};
            \node [fill=c3] (v3) at (x3) {\(c\)};
          \end{scope}
          \draw (v1) edge (v2) edge (v3);
          \draw (v2) edge (v3);
        \end{tikzpicture}
      }

      \bigskip

      \(G-d\)
    \end{center}
  \end{minipage}
  \caption{Lemma \ref{lem:remone} Example}
\end{figure}

Steps \ref{step:smallcalc} and \ref{step:small} remove vertices with degrees less than \(k\).  This is supported by
Theorem \ref{thm:lowdeg}.

\begin{theorem}
  \label{thm:lowdeg}
  Let \(G\) be a graph and let \(v\in V(G)\) such that \(\deg(v)<k\) for some \(k\in\N\).  \(G\) is \colorable{k} iff
  \(G-v\) is \colorable{k}.
\end{theorem}

\begin{proof}
  Assume \(G\) is \colorable{k}.  Therefore, by Lemma \ref{lem:remone}, \(G-v\) is also \colorable{k}.

  For the converse, assume that \(G-v\) is \(k\)-colorable.  By assumption, \(\deg(v)<k\), meaning \(v\) has at
  most \(k-1\) neighbors, using at most \(k-1\) colors.  Thus, there is always an additional color available for
  \(v\).  So extend \(G-v\) to \(G\) and color \(v\) with one of the available \(k-\deg(v)\) colors.  The result is
  a proper coloring of \(G\) using at most \(k\) colors.

  \(\therefore G\) is \colorable{k}.
\end{proof}

Theorem \ref{thm:lowdeg} is demonstrated in Figure \ref{fig:lowdeg} for \(k=4\) and \(\deg(v)=3\).

\begin{figure}[h]
  \label{fig:lowdeg}
  \begin{center}
    \begin{tikzpicture}
      \colorlet{c1}{green!25!white}
      \colorlet{c2}{blue!25!white}
      \colorlet{c3}{red!25!white}
      \colorlet{c4}{yellow!25!white}
      \begin{scope}[every node/.style={coordinate}]
        \cycleNnodes{4}{(0,0)}{1in}{135}{c};
      \end{scope}
      \begin{scope}[every node/.style={labeled node}]
        \node [fill=c1] (a) at (c1) {\(a\)};
        \node [fill=c2] (v) at (c2) {\(v\)};
        \node [fill=c3] (b) at (c3) {\(b\)};
        \node [fill=c4] (c) at (c4) {\(c\)};
        \node [fill=c2,below left=of a] (d) {\(d\)};
      \end{scope}
      \draw (a) edge (b) edge (c) edge (d);
      \draw [dashed,red] (v) edge (a) edge (b) edge (c);
      \draw (b) edge (c);
      \draw (c) edge (d);
    \end{tikzpicture}
  \end{center}
  \caption{Theorem \ref{thm:lowdeg} Example}
\end{figure}

The called subroutine actually removes all such vertices at once, which is supported by the inductive proof in
Corollary \ref{cor:lowdeg}.

\begin{corollary}
  \label{cor:lowdeg}
  Let \(G\) be a graph of order \(n\) and let \(X=\setb{v\in V(G)}{\deg(v)<k}\) for some \(k\in\N\).  \(G\) is
  \colorable{k} iff \(G-X\) is \colorable{k}.
\end{corollary}

\begin{proof}
  (by induction on \(\abs{X}\))
  \begin{description}
  \item[Base Case:] Let \(\abs{X}=0\).

    But \(G-X=G\) (trivial case).

  \item[Inductive Assumption:] Let \(\abs{X}=r\).

    Assume \(G\) is \colorable{k} iff \(G-X\) is \colorable{k}.

  \item[Inductive Step:] Consider \(\abs{X}=r+1\).
    
    Since \(\abs{X}=r+1>0\), there exists \(v\in X\) such that \(\deg(v)<k\).  Let \(Y=X-\set{v}\) and note that
    \(\abs{Y}=\abs{X}-1=(r+1)-1=r\).  So, \(G\) is \colorable{k} iff \(G-v\) is \colorable{k} (Theorem
    \ref{thm:lowdeg}) iff \((G-v)-Y\) is \colorable{k} (inductive assumption).
  \end{description}

  Therefore, by the principle of induction, \(G\) is \colorable{k} iff \(G-X\) is \colorable{k}.
\end{proof}

Returning to the example in Figure \ref{fig:lowdeg}, note that \(X=\set{v,b}\) is the set of all vertices with
degree less than \(4\) and so both could be removed at once in accordance with Corollary \ref{cor:lowdeg}.
Furthermore, after these vertices are removed, the remaining vertices \(a\), \(b\), and \(c\) will all have degree
\(2\).  Since \(2<4\), the remaining vertices are subsequently removed, leaving \(n=0<4\), indicating that the
graph is indeed \colorable{4}.  This iterative collapsing of a graph is an ideal situation.

Step \ref{step:neighbor} removes vertices whose neighborhoods are subsets of other vertices.  This is supported by
Theorem \ref{thm:subset}.

\begin{theorem}
  \label{thm:subset}
  Let \(G\) be a graph and let \(u,v\in V(G)\) such that \(N(u)\subseteq N(v)\).  \(G\) is \colorable{k} iff \(G-u\)
  is \colorable{k}.
\end{theorem}

\begin{proof}
  Assume \(G\) is \colorable{k}.  Therefore, by Lemma \ref{lem:remone}, \(G-u\) is also \colorable{k}.

  For the converse, assume that \(G-v\) is \(k\)-colorable.  Since \(N(u)\subseteq N(v)\) and (by definition)
  \(u\notin N(u)\), it must be the case that \(u\notin N(v)\) and hence \(uv\notin E(G)\).  Thus \(u\) and \(v\)
  are allowed to have the same color.  Furthermore, since every vertex adjacent to \(u\) is also adjacent to \(v\),
  none of these vertices can have the same color as \(v\).  So extend \(G-u\) to \(G\) and color \(u\) with the
  same color as \(v\).  The result is a proper coloring of \(G\) using at most \(k\) colors.

  \(\therefore G\) is \colorable{k}.
\end{proof}

Theorem \ref{thm:subset} is demonstrated in Figure \ref{fig:subset}.  Since \(N(u)\subseteq N(v)\), \(G\) and
\(G-u\) are equivalently colorable.  Futhermore, once \(u\) is removed, the degrees of vertices \(a\) and \(c\)
will have degree \(2\).  So if \(k=3\), those two vertices are subsequently removed by step \ref{step:small}.  The
remaining graph is of order \(2\), which then passes the success check of step \ref{step:check} because \(2<3\), so
indeed the graph is \colorable{3}.  Once again, this iterative removal of vertices is very powerful.

\begin{figure}[h]
  \label{fig:subset}
  \begin{center}
    \begin{tikzpicture}[every node/.style={labeled node}]
      \colorlet{c1}{green!25!white}
      \colorlet{c2}{blue!25!white}
      \colorlet{c3}{red!25!white}
      \node [fill=c1] (b) at (0,0) {\(b\)};
      \node [fill=c2] (a) [above=of b] {\(a\)};
      \node [fill=c2] (c) [below=of b] {\(c\)};
      \node [fill=c3] (u) [left=of b] {\(u\)};
      \node [fill=c3] (v) [right=of b] {\(v\)};
      \draw [dashed,red] (u) edge (a) edge (b);
      \draw (v) edge (a) edge (b) edge (c);
      \draw (a) edge (b);
      \draw (c) edge (b);
    \end{tikzpicture}
  \end{center}
  \caption{Theorem \ref{thm:subset} Example}
\end{figure}

\subsubsection{Minimum Shared Neighbor Upper Bound}

Steps \ref{step:select}--\ref{step:common} establish an upper bound for the minimum shared neighbor count between
any two vertices in a graph that is assumed to be \colorable{k}.  This limit is dependent on the following facts
that are guaranteed by previous steps:
\begin{enumerate}
\item \(2\le k<n\)
\item There are no \(u,v\in V(G)\) such that \(N(u)\subseteq N(v)\)
\end{enumerate}
The supporting theorem uses these facts along with Lemma \ref{lem:neighbor} in its proof.

\begin{lemma}
  \label{lem:neighbor}
  Let \(G\) be a graph and let \(S\) be a non-empty independent subset of \(V(G)\).  If there exists a vertex
  \(v\in S\) such that \(v\) is adjacent to all vertices in \(V(G)-S\) (i.e., \(N(v)=V(G)-S\)) then for all
  vertices \(u\in S\) it is the case that \(N(u)\subseteq N(v)\).
\end{lemma}

\begin{proof}
  Assume such a \(v\) exists and then assume that \(u\in S\).  If \(u=v\) then (trivially) \(N(v)=N(v)\), so assume
  \(u\ne v\).  Furthermore, since \(u,v\in S\) and \(S\) is independent (by assumption), it must be the case that
  \(u\) and \(v\) are not neighbors.

  \begin{description}
  \item[Case 1:] \(N(u)=\emptyset\).
      
    Therefore, by definition, \(N(u)=\emptyset\subseteq N(v)\).

  \item[Case 2:] \(N(u)\ne\emptyset\).

    Assume \(w\in N(u)\).  This means that \(w\) is adjacent to \(u\) and hence \(w\notin S\), since \(S\) is an
    independent set.  So \(w\in V(G)-S\) and thus, by assumption, \(v\) is adjacent to \(w\) and we can conclude
    that \(w\in N(v)\).  Therefore \(N(u)\subseteq N(v)\).
  \end{description}

  Therefore, for all \(u\in S\), \(N(u)\subseteq N(v)\).
\end{proof}

Lemma \ref{lem:neighbor} is demonstrated in Figure \ref{fig:neighbor}.  Note that since \(v\in S\) is adjacent to
every vertex in \(V(G)-S\), vertex \(u\in S\) can't help but be adjacent to some subset of \(N(v)\).

\begin{figure}[h]
  \label{fig:neighbor}
  \begin{center}
    \begin{tikzpicture}
      \draw (0in,0) ellipse (0.5in and 1in);
      \draw (3in,0) ellipse (1.5in and 1in);
      \begin{scope}[every node/.style={unlabeled node}]
        \node (v) at (0in,0.25in) {};
        \node (u) at (0in,-0.25in) {};
        \node (w1) at (2in,0) {};
        \node (w2) at (2.5in,0) {};
        \node (w3) at (3in,0) {};
        \node (w4) at (3.5in,0) {};
        \node (w5) at (4in,0) {};
      \end{scope}
      \node [left=1ex of v] {\(v\)};
      \node [left=1ex of u] {\(u\)};
      \node at (0,-1.25in) {\(S\)};
      \node at (3in,-1.25in) {\(V(G)-S\)};
      \draw (v) [bend left] edge (w1);
      \draw (v) [bend left] edge (w2);
      \draw (v) [bend left] edge (w3);
      \draw (v) [bend left] edge (w4);
      \draw (v) [bend left] edge (w5);
      \draw (u) [bend right] edge (w2);
      \draw (u) [bend right] edge (w4);
      \draw (u) [bend right] edge (w5);
    \end{tikzpicture}
  \end{center}
  \caption{Lemma \ref{lem:neighbor} Example}
\end{figure}

Theorem \ref{thm:inter} establishes the desired upper bound.

\begin{theorem}
  \label{thm:inter}
  Let \(G\) be a graph of order \(n\) and size \(m\) such that there are no \(u,v\in V(G)\) where \(N(u)\subseteq
  N(v)\), and let \(k\in\N\) such that \(2\le k<n\).  If \(G\) is \colorable{k} then there exists two vertices
  \(w,z\in V(G)\) such that:
  \[\abs{N(w)\cap N(z)}\le n-2-\frac{n-2}{k-1}\]
\end{theorem}

\begin{proof}
  Assume \(G\) is \colorable{k}.  This means that \(V(G)\) can be distributed into \(k\) independent (some possibly
  empty) subsets \(A_1,\ldots,A_k\) such that \(a_i=\abs*{A_i}\) and \(a_1\ge a_2\ge\cdots\ge a_k\).  Since
  \(n>k\), by the pigeonhole principle, it must be the case that \(a_1\ge2\).  Assume \(v\in A_1\).

  First, assume by way of contradiction (ABC) that \(v\) is adjacent to all other vertices in \(V(G)-A_1\).  Since
  \(a_1\ge2\), there exists \(u\in A_1\) such that \(u\ne v\) and \(u\) is not adjacent to \(v\).  Thus, by Lemma
  \ref{lem:neighbor}, \(N(u)\subseteq N(v)\), contradicting the assumption.  Note that this contradiction also
  eliminates the degenerate case where \(A_1=V(G)\); however, this case does not occur here because the graph would
  be an empty graph and would have been eliminated by previous steps.  Therefore, there exists some \(v'\in
  V(G)-A_1\) such that \(v\) is not adjacent to \(v'\).  Assume \(v'\in A_i\) for some \(i\) such that \(1<i\le
  k\):

  \begin{description}
  \item [Case 1:] \(a_i=1\)

    By the pigeonhole principle:
    \[a_1\ge\ceil*{\frac{n-1}{k-1}}\ge\frac{n-1}{k-1}\]
    Now, assume by way of contradiction (ABC) that \(v'\) is adjacent to all vertices in \(V(G)-A_1-A_i\) and
    assume \(u\in N(v)\).  Then it must be the case that \(u\in V(G)-A_1-A_i\), and so \(u\) is adjacent to \(v'\),
    and thus \(u\in N(v')\).  Therefore \(N(v)\subseteq N(v')\), which contradicts the assumption.  This situation
    is demonstrated by Figure \ref{fig:aione}.

    \begin{figure}[h]
      \label{fig:aione}
      \begin{center}
        \begin{tikzpicture}
          \draw (0,0) ellipse (0.5in and 1in);
          \draw (2in,0) ellipse (0.5in and 1in);
          \draw (4in,0) ellipse (0.5in and 1in);
          \begin{scope}[every node/.style={unlabeled node}]
            \node (v) at (0,0) {};
            \node (u1) at (2in,0.5in) {};
            \node (u2) at (2in,0.25in) {};
            \node (u3) at (2in,0in) {};
            \node (u4) at (2in,-0.25in) {};
            \node (u5) at (2in,-0.5in) {};
            \node (vp) at (4in,0) {};
          \end{scope}
          \node [left=1ex of v] {\(v\)};
          \node [right=1ex of vp] {\(v'\)};
          \node at (0,-1.25in) {\(A_1\)};
          \node at (2in,-1.25in) {\(V(G)-A_1-A_i\)};
          \node at (4in,-1.25in) {\(A_i\)};
          \draw (v) edge (u1) edge (u3) edge (u4);
          \draw (vp) edge (u1) edge (u2) edge (u3) edge (u4) edge (u5);
        \end{tikzpicture}
      \end{center}
      \caption{Case \(a_i=1\) Contradiction}
    \end{figure}

    So there must exist some \(u\in V(G)-A_1-A_i\) such that \(u\) is not adjacent to \(v'\).  This results in the
    upper bound:
    \[\abs{N(v)\cap N(v')}\le n-\abs{\set{u,v'}}-a_1\le n-2-\frac{n-1}{k-1}\]
    Note that since \(v\in A_1\), it is already counted in \(a_1\).  Comparing this bound to the desired bound:
    \[\left(n-2-\frac{n-2}{k-1}\right)-\left(n-2-\frac{n-1}{k-1}\right)=\frac{(n-1)-(n-2)}{k-1}=\frac{1}{k-1}>0\]
    for \(k\ge2\).  Thus the new bound is tighter and so:
    \[\abs{N(v)\cap N(v')}\le n-2-\frac{n-1}{k-1}\le n-2-\frac{n-2}{k-1}\]
    
  \item [Case 2:] \(a_i=2\)

    By the pigeonhole principle:
    \[a_1\ge\ceil*{\frac{n-2}{k-1}}\ge\frac{n-2}{k-1}\]
    This results in the upper bound:
    \[\abs{N(v)\cap N(v')}\le n-a_i-a_1\le n-2-\frac{n-2}{k-1}\]
    
  \item [Case 3:] \(a_i\ge3\)

    By the pigeonhole principle:
    \[a_1\ge\ceil*{\frac{n-3}{k-1}}\ge\frac{n-3}{k-1}\]
    This results in the upper bound:
    \[\abs{N(v)\cap N(v')}\le n-a_i-a_1\le n-3-\frac{n-3}{k-1}\]
    Comparing this to the desired bound:
    \[\left(n-2-\frac{n-2}{k-1}\right)-\left(n-3-\frac{n-3}{k-1}\right)=1-\frac{(n-3)-(n-2)}{k-1}=1-\frac{1}{k-1}>0\]
    for \(k\ge2\).  Thus the new bound is tighter and so:
    \[\abs{N(v)\cap N(v')}\le n-3-\frac{n-3}{k-1}\le n-2-\frac{n-2}{k-1}\]
  \end{description}

  Therefore, there exists \(w,v\in V(G)\) such that:
  \[\abs{N(w)\cap N(z)}\le n-2-\frac{n-2}{k-1}\]
\end{proof}

The called subroutine actually uses the contrapositive of this result, as stated in Corollary \ref{cor:inter}.

\begin{corollary}
  \label{cor:inter}
  Let \(G\) be a graph of order \(n\) and size \(m\) such that there are no \(u,v\in V(G)\) where
  \(N(u)\subseteq N(v)\), and let \(k\in\N\) such that \(2\le k<n\).  If for all \(w,z\in V(G)\) it is the case
  that:
  \[\abs{N(w)\cap N(z)}>n-2-\frac{n-2}{k-1}\]
  then \(G\) is not \colorable{k}.
\end{corollary}

Corollary \ref{cor:inter} is demonstrated in Figure \ref{fig:inter}.  The shown graph has \(n=5\), is \chromatic{3},
and has:
\[\min_{u,v\in V(G)}\abs{N(u)\cap N(v)}=1\]
Testing for \(k=2\):
\[5-2-\frac{5-2}{2-1}=0\]
But \(1>0\) and so we can conclude that \(G\) is not \colorable{2}.  However, testing for \(k=3\):
\[5-2-\frac{5-2}{3-1}=\frac{3}{2}\]
So \(1\ngtr\frac{3}{2}\) and thus \(G\) \emph{may} be \colorable{3}, since this test only provides a necessary and
not a sufficient condition.

\begin{figure}[h]
  \label{fig:inter}
  \begin{center}
    \begin{tikzpicture}
      \colorlet{c1}{green!25!white}
      \colorlet{c2}{blue!25!white}
      \colorlet{c3}{red!25!white}
      \begin{scope}[every node/.style={labeled node}]
        \node [fill=c1] (a) at (0,0) {\(a\)};
        \node [fill=c2,above left=of a] (b) {\(b\)};
        \node [fill=c3,above right=of a] (c) {\(c\)};
        \node [fill=c2,below left=of a] (d) {\(d\)};
        \node [fill=c3,below right=of a] (e) {\(e\)};
      \end{scope}
      \draw (a) -- (b) -- (c) -- (a) -- (d) -- (e) -- (a);
    \end{tikzpicture}

    \bigskip

    \(G\)
  \end{center}
  \caption{Corollary \ref{cor:inter} Example}
\end{figure}

\subsubsection{Recursive Steps}

If no more vertices can be removed and the failure tests pass then the called subroutine reverts to the exhaustive
method.  Step \ref{step:select2} selects two non-adjacent vertices with the smallest number of shared neighbors.
Such a pair must exist.  Otherwise, the current state of \(G\) is complete, which would have been eliminated by
step \ref{step:density}.  The first recursive call (step \ref{step:call1}) assumes that the two selected vertices
have the same color, so they are contracted.  The second recursive call (step \ref{step:call2}) assumes that the
two selected vertices have different colors, so they are joined by an added edge.  Each call starts a new branch of
the Zykov tree corresponding to the current value of \(k\).  If either call returns true then it can be concluded
that the input graph was indeed \colorable{k}.  Otherwise, it can be concluded that the input graph is not
\colorable{k} and the called subroutine returns the state of \(G\) prior to the recursive calls to the outer loop.

These steps are supported by Theorem \ref{thm:recurse}.

\begin{theorem}
  \label{thm:recurse}
  Let \(G\) be a graph of order \(n>=2\) and let \(u,v\in G\) such that \(u\) and \(v\) are not adjacent.  \(G\) is
  \colorable{k} iff \(G\cdot uv\) or \(G+uv\) is \colorable{k}.
\end{theorem}

\begin{proof}
  \begin{description}
  \item[]
  \item[\(\implies\)] Assume \(G\) is \colorable{k}.

    \begin{description}
    \item [Case a:] \(u\) and \(v\) have the same color.

      Then \(\forall\,w\in N(u)\cup N(v)\) it must be the case that \(w\) is a different color than the color of
      \(u\) and \(v\).  Let \(v'\) be the contracted vertex, so that \(N(v')=N(u)\cup N(v)\), and color \(v'\) with
      the same color as \(u\) and \(v\).  The result is a proper \coloring{k} of \(G\cdot uv\).

      \(\therefore G\cdot uv\) is \colorable{k}.
      
    \item [Case b:] \(u\) and \(v\) have the different colors in \(c\).

      By adding edge \(uv\), \(u\) and \(v\) become adjacent and thus must have different colors.  Therefore, \(u\)
      and \(v\) can retain their same colors.  The result is a proper \coloring{k} of \(G+uv\).
    \end{description}

    \(\therefore G\cdot uv\) or \(G+uv\) is \colorable{k}.
    
  \item[\(\impliedby\)] Assume \(G\cdot uv\) or \(G+uv\) is \colorable{k}.

    \begin{description}
    \item [Case a:] \(G\cdot uv\) is \colorable{k}.

      Let \(v'\) be the contracted vertex with some assigned color.  It must be the case that \(\forall\,w\in
      N(v'), w\) has a different color than \(v'\).  Expand \(G\cdot uv\) to \(G\) and color \(u\) and \(v\) with
      the same color as \(v'\).  The result is a proper \coloring{k} of \(G\).
      
    \item [Case b:] \(G+uv\) is \colorable{k}.

      Remove edge \(uv\).  Since \(u\) and \(v\) are no longer adjacent, there are no requirements on their colors.
      Thus, they can retain their original colors.  The result is a proper \coloring{k} of \(G\).
    \end{description}

    \(\therefore G\) is \colorable{k}.
  \end{description}
\end{proof}

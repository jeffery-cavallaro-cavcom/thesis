\section{The Proposed Algorithm}

The exhaustive algorithm described in the previous section subjects each pair of non-adjacent vertices in a graph
\(G\) to the two choices of like (vertex contraction) or different (edge addition) color assignment in a recursive
manner.  The leaves of the resulting Zykov tree represent complete graphs that describe all of the possible \(k\)
colorings of \(G\), with the complete graphs of smallest order representing chromatic colorings of \(G\).

The major advantages of the exhaustive algorithm are that it isn't dependent on whether or not a graph is
connected, an example of a chromatic coloring is readily available, and the fact that the algorithm, due to its
Turing machine nature, can be coded rather easily to run on a computer.  Its major disadvantage is its high runtime
complexity due to the need to generate the entire Zykov tree using an exponentially growing number of recursive
calls.

Thus, the goals of the proposed algorithm are as follows:
\begin{enumerate}
\item It does not depend on whether a graph is connected or not.
\item An example of a chromatic coloring should be easily available.
\item It can be easily coded for execution on a computer.
\item It has significantly better runtime complexity than the exhaustive algorithm.
\end{enumerate}

To accomplish these goals, the proposed algorithm takes a somewhat different tack: it loops on successively higher
values of \(k\).  For each candidate \(k\) value, a graph is assumed to be \colorable{k} and a modified version of
the exhaustive algorithm is executed to either prove or disprove this assumption.  Since a candidate \(k\) value is
known, certain reversible steps can be applied to mutate \(G\) into simpler graphs with the same colorability and
test for very early termination of the current Zykov tree.  The first \(k\) for which \(G\) (or one of its
simplifications) is found to be \(k\)-colorable is the chromatic number of \(G\).  As will be shown, the tradeoff
of looping on \(k\) and very shallow execution of each corresponding Zykov tree far outweighs the need to generate
an entire Zykov tree, resulting in a much better runtime complexity.

One slight disadvantage of the proposed algorithm is that whereas the exhaustive algorithm readily provides
examples of actual chromatic colorings, the proposed algorithm requires a reverse traversal of its reversible steps
in order to construct such a coloring.  Nevertheless, this technique still has the advantage over common coloring
algorithms such as greedy coloring because it is not heuristic.  However, as was stated earlier, during the
axiomatic design phase it is more important to know the minimum number of parts as opposed to actual functional
requirement allocation to those parts.

This algorithm was first proposed by the author and his advisor in collaboration with a team of mechanical
engineering researchers from SUNY Buffalo \cite{cavallaro}.  It accepts a graph \(G\) as input, provides \(\X(G)\)
as output, and is composed of an outer loop on values of \(k\) and a subroutine called by the outer loop to
determine if \(G\) is \colorable{k}.  The outer loop and called subroutine are summarized in the following
sections.  A complete description of the theorems that support the various steps in algorithm and the application
of the algorithm to a sample graph then follow.

\subsection{Outer Loop}

The outer loop accepts a graph \(G\) as input and returns \(\X(G)\).  It initially checks for some edge cases and
then loops on increasing values of \(k\).  For each value of \(k\) the called subroutine executes the modified
exhaustive algorithm to determine if \(G\) is \colorable{k}.  The first such successful return identifies
\(\X(G)\).

The steps of the outer loop are as follows:

\begin{enumerate}
\item \label{step:null} If \(n=0\) then return \(0\), thus handling the edge case of a null graph.

\item \label{step:one} If \(m=0\) then return \(1\), thus handling the edge case of an empty graph.

\item \label{step:init} Initialize \(k\) to \(2\).

\item \label{step:inner} Call the subroutine to determine if \(G\) is \colorable{k}.  The subroutine returns a
  possibly simplified \(G\) called \(G'\) and a boolean value \(R\) that reports the result of the test.

\item \label{step:call} If \(G\) is \colorable{k} (\(R=\) true) then return \(k\).

\item \label{step:newg} Otherwise, replace \(G\) with \(G'\).  As will be seen, doing this avoids needless
  reapplication of certain steps in the called subroutine.

\item \label{step:incr} Increment \(k\).

\item \label{step:loop} Go to step \ref{step:inner}.
\end{enumerate}

A flowchart of these steps is shown in Figure \ref{fig:outer}.

\begin{figure}[h]
  \label{fig:outer}
  \begin{center}
    \scalebox{0.75}{
      \begin{tikzpicture}[>=latex']
        \node (start) [draw,terminal] at (0,0) {START};
        \node (nullcheck) [draw,decision,below=of start] {\(n=0\)?};
        \node (nulldone) [draw,terminal,right=of nullcheck] {RETURN \(0\)};
        \node (emptycheck) [draw,decision,below=of nullcheck] {\(m=0\)?};
        \node (emptydone) [draw,terminal,right=of emptycheck] {RETURN \(1\)};
        \node (kinit) [draw,process,below=of emptycheck] {\(k=2\)};
        \node (kcheck) [draw,predproc,below=of kinit] {CALL \(G,k\)};
        \node (iskcolor) [draw,decision,below=of kcheck] {\(R=\)TRUE?};
        \node (kdone) [draw,terminal,right=of iskcolor] {RETURN \(k\)};
        \node (newg) [draw,process,below=of iskcolor] {\(G=G'\)};
        \node (kinc) [draw,process,below=of newg] {\(k=k+1\)};
        \node (belowinc) [coordinate,below=0.5cm of kinc] {};
        \node (leftinc) [coordinate,left=3cm of belowinc] {};
        \draw [->] (start) -- node [auto] {\(G\)} (nullcheck);
        \draw [->] (nullcheck) -- node [auto] {YES} (nulldone);
        \draw [->] (nullcheck) -- node [auto] {NO} (emptycheck);
        \draw [->] (emptycheck) -- node [auto] {YES} (emptydone);
        \draw [->] (emptycheck) -- node [auto] {NO} (kinit);
        \draw [->] (kinit) -- (kcheck);
        \draw [->] (kcheck) -- node [auto] {\(G',R\)} (iskcolor);
        \draw [->] (iskcolor) -- node [auto] {YES} (kdone);
        \draw [->] (iskcolor) -- node [auto] {NO} (newg);
        \draw [->] (newg) -- (kinc);
        \draw [->] (kinc) -- (belowinc) -- (leftinc) |- (kcheck);
      \end{tikzpicture}
    }
  \end{center}
  \caption{Algorithm Outer Loop}
\end{figure}

Note that the outer loop is guaranteed to complete because \(k\) will eventually be greater than or equal to \(n\),
and thus by Proposition \ref{prop:coloring3}, the current state of \(G\) is at least \colorable{k}, causing the
called subroutine to return true.

\subsection{Called Subroutine}

The called subroutine executes a modified version of the exhaustive algorithm that determines whether a graph is
\colorable{k}.  It accepts the current state of \(G\) of order \(n\) and size \(m\) and the current value of
\(k\ge2\) as inputs.  It returns a possibly simplified version of \(G\) and a boolean value indicating whether or
not \(G\) is \colorable{k}.  Internally, various tests are applied to trim the corresponding Zykov tree or abandon
it all together based on the current value of \(k\).

The steps of the called subroutine and their associated theorems are as follows:

\begin{enumerate}
\item \label{step:check} If \(n\le k\) then return true (Proposition \ref{prop:coloring3}).  This success condition
  is achieved whenever \(G\) is simplified by removing sufficent vertices (steps \ref{step:small} and
  \ref{step:neighbor}) or when the outer loop has sufficiently incremented \(k\) (step \ref{step:incr}).

\item \label{step:dencalc} Calculate a maximum edge threshold:
  \[a=\frac{n^2(k-1)}{2k}\]

\item \label{step:density} If \(m>a\) then return false (Lemma \ref{cor:density}).

\item \label{step:smallcalc} Construct the set \(X\) of all vertices with degree less than \(k\):
  \[X=\setb{v\in V(G)}{\deg(v)<k}\]

\item \label{step:small} \(X\ne\emptyset\) then replace \(G\) with \(G-X\) and go to step \ref{step:check} (Corollary
  \ref{cor:lowdeg}).

\item \label{step:neighbor} If \(G\) has vertices \(u\) and \(v\) such that \(N(u)\subseteq N(v)\) then replace
  \(G\) with \(G-u\) and go to step \ref{step:check} (Lemma \ref{lem:subset}).

\item \label{step:select} Select two vertices \(u,v\in V(G)\) with the smallest number of common neighbors and let
  \[b=\abs{N(u)\cap N(v)}\]

\item \label{step:neighcalc} Calculate an upper bound for the minimum number of common neighbors for all vertices
  in \(G\):
  \[c=n-2-\frac{n-2}{k-1}\]

\item \label{step:common} If \(b>c\) then return false (Corollary \ref{cor:inter}).

\item \label{step:select2} Select two non-adjacent vertices \(u,v\in V(G)\) with the smallest number of common
  neighbors.  It will be shown below that such a pair of vertices is guaranteed to exist in the current state of
  \(G\).

\item \label{step:call1} Assume that \(u\) and \(v\) are assigned the same color by letting \(G'=G\cdot uv\).
  Recursively call this routine to see if \(G'\) is \colorable{k}.  If so, then return true (Lemma
  \ref{lem:recurse}).

\item \label{step:call2} Assume that \(u\) and \(v\) are assigned different colors by letting \(G'=G+uv\).
  Recursively call this routine to see if \(G'\) is \colorable{k}.  If so, then return true (Lemma
  \ref{lem:recurse}).

\item \label{step:fail} Since neither of the assumptions in steps \ref{step:call1} and \ref{step:call2} hold,
  conclude that \(G\) is not \(k\)-colorable and return false.
\end{enumerate}

A flowchart of these steps is shown in Figure \ref{fig:called}.

\begin{figure}[h]
  \label{fig:called}
  \begin{center}
    \scalebox{0.7}{
      \begin{tikzpicture}[>=latex']
        \node (start) [draw,terminal] at (0,0) {START};
        \node (donecheck) [draw,decision,below=of start] {\(n\le k\)?};
        \node (done) [draw,terminal,right=of donecheck] {RETURN \(G\),TRUE};
        \node (edgecalc) [draw,process,below=of donecheck] {\(a=\frac{n^2(k-1)}{2k}\)};
        \node (edgecheck) [draw,decision,below=of edgecalc] {\(m>a\)?};
        \node (edgefail) [draw,terminal,right=of edgecheck] {RETURN \(G\),FALSE};
        \node (nodecalc) [draw,process,below=of edgecheck] {\(X=\setb{v\in V(G)}{\deg(v)<k}\)};
        \node (nodecheck) [draw,decision,below=of nodecalc] {\(X\ne\emptyset\)?};
        \node (remnode) [draw,process,left=of nodecheck] {\(G=G-X\)};
        \node (join) [coordinate] at ($(remnode)-(2.5cm,0)$) {};
        \node (subcheck) [draw,decision,below=of nodecheck] {\(N(u)\subseteq N(v)\)?};
        \node (remsub) [draw,process,left=of subcheck] {\(G=G-u\)};
        \node (mininter) [draw,process,below=of subcheck] {\(\displaystyle b=\min_{u,v\in V(G)}\abs{N(u)\cap N(v)}\)};
        \node (intercalc) [draw,process,below=of mininter] {\(c=n-2-\frac{n-2}{k-1}\)};
        \node (intercheck) [draw,decision,below=of intercalc] {\(b>c\)?};
        \node (interfail) [draw,terminal,right=of intercheck] {RETURN \(G\),FALSE};
        \node (finduv) [draw,process,right=2.5cm of done] {\(\displaystyle \min_{uv\notin E(G)}\abs{N(u)\cap N(v)}\)};
        \node (save1) [draw,process,below=of finduv] {\(G'=G\cdot uv\)};
        \node (call1) [draw,predproc,below=of save1] {CALL \(G',k\)};
        \node (check1) [draw,decision,below=of call1] {\(R=\)TRUE?};
        \node (done1) [draw,terminal,right=of check1] {RETURN \(G\),TRUE};
        \node (save2) [draw,process,below=of check1] {\(G'=G+uv\)};
        \node (call2) [draw,predproc,below=of save2] {CALL \(G',k\)};
        \node (check2) [draw,decision,below=of call2] {\(R=\)TRUE?};
        \node (done2) [draw,terminal,right=of check2] {RETURN \(G\),TRUE};
        \node (fail) [draw,terminal,below=of check2] {RETURN \(G\),FALSE};
        \draw [->] (start) -- node [auto] {\(G,k\)} (donecheck);
        \draw [->] (donecheck) -- node [auto] {YES} (done);
        \draw [->] (donecheck) -- node [auto] {NO} (edgecalc);
        \draw [->] (edgecalc) -- (edgecheck);
        \draw [->] (edgecheck) -- node [auto] {YES} (edgefail);
        \draw [->] (edgecheck) -- node [auto] {NO} (nodecalc);
        \draw [->] (nodecalc) -- (nodecheck);
        \draw [->] (nodecheck) -- node [auto] {YES} (remnode);
        \draw [->] (remnode) -- (join) |- (donecheck);
        \draw [->] (nodecheck) -- node [auto] {NO} (subcheck);
        \draw [->] (subcheck) -- node [auto] {YES} (remsub);
        \draw (remsub) -| (join);
        \draw [->] (subcheck) -- node [auto] {NO} (mininter);
        \draw [->] (mininter) -- (intercalc);
        \draw [->] (intercalc) -- (intercheck);
        \draw [->] (intercheck) -- node [auto] {YES} (interfail);
        \draw [->] (intercheck) -- node [auto] {NO} ($(intercheck)-(0,2cm)$) -- ++(7cm,0) |- (finduv);
        \draw [->] (finduv) -- (save1);
        \draw [->] (save1) -- (call1);
        \draw [->] (call1) -- node [auto] {\(G'',R\)} (check1);
        \draw [->] (check1) -- node [auto] {YES} (done1);
        \draw [->] (check1) -- node [auto] {NO} (save2);
        \draw [->] (save2) -- (call2);
        \draw [->] (call2) -- node [auto] {\(G'',R\)} (check2);
        \draw [->] (check2) -- node [auto] {YES} (done2);
        \draw [->] (check2) -- node [auto] {NO} (fail);
      \end{tikzpicture}
    }
  \end{center}
  \caption{Algorithm Called Subroutine}
\end{figure}

Steps \ref{step:check}--\ref{step:neighbor} of the subroutine attempt to remove vertices to achieve a simpler graph
that is also \colorable{k}.  Note that each time a vertex is removed, the subtrees associated with that vertex in
the corresponding Zykov tree for \(G\) are ignored.  Since these same steps would just be repeated for \(k+1\), the
subroutine returns the current state of the possibly simplified \(G\) to the outer loop as a starting point for the
next candidate value of \(k\).

Steps \ref{step:density} and \ref{step:common} of the subroutine apply tests that attempt to disprove that the
current state of \(G\) is \colorable{k} for the current value of \(k\).  If so, then the current Zykov tree is
abandoned and the subroutine returned false.  This allows the outer loop to continue with \(k+1\).

Step \ref{step:select2} of the subroutine assumes that there exists at least one pair of non-adjacent vertices.
The previous steps in the algorithm guarantee that this is indeed the case.  Otherwise, the current state of \(G\)
at step \ref{step:select} would be a complete graph and thus \(b=n-2\).  Comparing this to the value of \(c\) in
step \ref{step:neighcalc}:
\[b-c=(n-2)-\left(n-2-\frac{n-2}{k-1}\right)=\frac{n-2}{k-1}\]
with \(k\ge2\).  We can also conclude that \(n>2\) at this point, since any graph of order \(1\) or \(2\) would
have been wiped out by vertex reduction in step \ref{step:small}.  And so \(b-c>0\), causing step \ref{step:common}
to eliminated any complete graph.  Therefore, the current state of \(G\) is not complete and so there exists at
least one pair of non-adjacent vertices.

The remaining steps of the subroutine constitute the recursive portion of the modified exhaustive algorithm.  The
subroutine is guaranteed to return because either there will be sufficient contractions such that \(n\le k\),
resulting in a true return, or sufficient edge additions such that the graph becomes complete and is rejected by
step \ref{step:common}, resulting in a false return.

\clearpage

\subsection{Supporting Theorems}

This section contains the theorems that support the steps in the called subroutine.  Remember that the success check
of step \ref{step:check} is already supported by Proposition \ref{prop:coloring3}.

The maximum edge threshold test of steps \ref{step:dencalc} and \ref{step:density} are supported by Theorem
\ref{thm:density} and Corollary \ref{cor:density}.

\begin{theorem}[Maximum Edge Threshold]
  \label{thm:density}
  Let \(G\) be a graph of order \(n\) and size \(m\) and let \(k\in\N\).  If \(G\) is \colorable{k} then:
  \[m\le\frac{n^2(k-1)}{2k}\]
\end{theorem}

\begin{proof}
  Assume \(G\) is \colorable{k}.  This means that \(V(G)\) can be distributed into \(k\) independent (some possibly
  empty) subsets.  Call these subsets \(A_1,\ldots A_k\) and let \(a_i=\abs*{A_i}\).  Thus, each \(v\in A_i\) can
  be adjacent to at most \(n-a_i\) other vertices in \(G\), and hence the maximum number of edges incident to
  vertices in \(A_i\) is given by: \(a_i(n-a_i)=na_i-a_i^2\).  Now, using Theorem \ref{thm:first}, the maximum
  number of edges in \(G\) is given by:
  \[m\le\frac{1}{2}\sum_{i=1}^k(na_i-a_i^2)\]
  with the constraint:
  \[\sum_{i=1}^ka_i=n\]
  This problem can be solved using the Lagrange multiplier technique.  We start by defining:
  \begin{align*}
    F(a_1,\ldots,a_k) &= f(a_1,\ldots,a_k)-\l g(a_1,\ldots,a_k) \\
    &= \frac{1}{2}\sum_{i=1}^k(na_i-a_i^2)-\l\sum_{i=1}^ka_i \\
    &= \sum_{i=1}^k\left(\frac{1}{2}na_i-\frac{1}{2}a_i^2-\l a_i\right)
  \end{align*}
  Now, optimize by taking the gradient and setting the result equal to the zero vector:
  \[\vec{\nabla}F=\sum_{i=1}^k(\frac{n}{2}-a_i-\l)\hat{a_i}=\vec{0}\]
  This results in a system of \(k\) equations of the form:
  \[\frac{n}{2}-a_i-\l=0\]
  And so:
  \[a_i=\frac{n}{2}-\l\]
  Plugging this result back into the contraint:
  \[\sum_{i=1}^ka_i=\sum_{i=1}^k\left(\frac{n}{2}-\l\right)=k\left(\frac{n}{2}-\l\right)=n\]
  Solving for \(\l\) yields:
  \[\l=\frac{n}{2}-\frac{n}{k}\]
  And finally, to get \(a_i\) in terms of \(n\) and \(k\):
  \[a_i=\frac{n}{2}-\left(\frac{n}{2}-\frac{n}{k}\right)=\frac{n}{k}\]
  Therefore:
  \[m\le\frac{1}{2}\sum_{i=1}^k\left[n\left(\frac{n}{k}\right)-\left(\frac{n}{k}\right)^2\right]=
  \frac{k}{2}\left(\frac{n^2k-n^2}{k^2}\right)=\frac{n^2(k-1)}{2k}\]
\end{proof}

The called subroutine actually uses the contrapositive of this result, as stated in Corollary \ref{cor:density}.

\begin{corollary}
  \label{cor:density}
  Let \(G\) be a graph of order \(n\) and size \(m\) and let \(k\in\N\).  If:
  \[m>\frac{n^2(k-1)}{2k}\]
  then \(G\) is not \colorable{k}.
\end{corollary}

The theorems that support vertex removal make use of Lemma \ref{lem:remone}.

\begin{lemma}
  \label{lem:remone}
  Let \(G\) be a graph and let \(v\in V(G)\).  If \(G\) is \colorable{k} then \(G-v\) is also \colorable{k}.
\end{lemma}

\begin{proof}
  Assume \(G\) is \colorable{k} and let \(c:V(G)\to C\) be such a coloring.  This means that \(\abs{C}=k\) and for
  all \(uw\in E(G)\) such that \(v\notin uw\) it is the case that \(c(u)=c_1\) and \(c(w)=c_2\) and \(c_1\ne c_2\)
  for some \(c_1,c_2\in C\).

  Now, consider the restricted function \(c'=\restrict{c}{V(G-v)}\).  It is still the case that \(c'(u)=c_1\) and
  \(c'(w)=c_2\) and \(c_1\ne c_2\), so \(c'\) is still proper with \(\abs{C}=k\).

  \(\therefore G-v\) is \colorable{k}.
\end{proof}

Lemma \ref{lem:remone} is demonstrated in Figure \ref{fig:remone}.  No matter which vertex is removed, the resulting
subgraph is still properly colored using at most four colors.

\begin{figure}
  \label{fig:remone}
  \begin{minipage}{1.25in}
    \begin{center}
      \scalebox{0.75}{
        \begin{tikzpicture}
          \colorlet{c1}{green!25!white}
          \colorlet{c2}{blue!25!white}
          \colorlet{c3}{red!25!white}
          \colorlet{c4}{yellow!25!white}
          \begin{scope}[every node/.style={coordinate}]
            \cycleNnodes{4}{(0,0)}{0.5in}{135}{x};
          \end{scope}
          \begin{scope} [every node/.style={labeled node}]
            \node [fill=c1] (v1) at (x1) {\(a\)};
            \node [fill=c2] (v2) at (x2) {\(b\)};
            \node [fill=c3] (v3) at (x3) {\(c\)};
            \node [fill=c4] (v4) at (x4) {\(d\)};
          \end{scope}
          \draw (v1) edge (v2) edge (v3) edge (v4);
          \draw (v2) edge (v3) edge (v4);
          \draw (v3) edge (v4);
        \end{tikzpicture}
      }

      \bigskip

      \(G\)
    \end{center}
  \end{minipage}
  \begin{minipage}{1.25in}
    \begin{center}
      \scalebox{0.75}{
        \begin{tikzpicture}
          \colorlet{c1}{green!25!white}
          \colorlet{c2}{blue!25!white}
          \colorlet{c3}{red!25!white}
          \colorlet{c4}{yellow!25!white}
          \begin{scope}[every node/.style={coordinate}]
            \cycleNnodes{4}{(0,0)}{0.5in}{135}{x};
          \end{scope}
          \begin{scope} [every node/.style={labeled node}]
            \node [fill=c2] (v2) at (x2) {\(b\)};
            \node [fill=c3] (v3) at (x3) {\(c\)};
            \node [fill=c4] (v4) at (x4) {\(d\)};
          \end{scope}
          \draw (v2) edge (v3) edge (v4);
          \draw (v3) edge (v4);
        \end{tikzpicture}
      }

      \bigskip

      \(G-a\)
    \end{center}
  \end{minipage}
  \begin{minipage}{1.25in}
    \begin{center}
      \scalebox{0.75}{
        \begin{tikzpicture}
          \colorlet{c1}{green!25!white}
          \colorlet{c2}{blue!25!white}
          \colorlet{c3}{red!25!white}
          \colorlet{c4}{yellow!25!white}
          \begin{scope}[every node/.style={coordinate}]
            \cycleNnodes{4}{(0,0)}{0.5in}{135}{x};
          \end{scope}
          \begin{scope} [every node/.style={labeled node}]
            \node [fill=c1] (v1) at (x1) {\(a\)};
            \node [fill=c3] (v3) at (x3) {\(c\)};
            \node [fill=c4] (v4) at (x4) {\(d\)};
          \end{scope}
          \draw (v1) edge (v3) edge (v4);
          \draw (v3) edge (v4);
        \end{tikzpicture}
      }

      \bigskip

      \(G-b\)
    \end{center}
  \end{minipage}
  \begin{minipage}{1.25in}
    \begin{center}
      \scalebox{0.75}{
        \begin{tikzpicture}
          \colorlet{c1}{green!25!white}
          \colorlet{c2}{blue!25!white}
          \colorlet{c3}{red!25!white}
          \colorlet{c4}{yellow!25!white}
          \begin{scope}[every node/.style={coordinate}]
            \cycleNnodes{4}{(0,0)}{0.5in}{135}{x};
          \end{scope}
          \begin{scope} [every node/.style={labeled node}]
            \node [fill=c1] (v1) at (x1) {\(a\)};
            \node [fill=c2] (v2) at (x2) {\(b\)};
            \node [fill=c4] (v4) at (x4) {\(d\)};
          \end{scope}
          \draw (v1) edge (v2) edge (v4);
          \draw (v2) edge (v4);
        \end{tikzpicture}
      }

      \bigskip

      \(G-c\)
    \end{center}
  \end{minipage}
  \begin{minipage}{1.25in}
    \begin{center}
      \scalebox{0.75}{
        \begin{tikzpicture}
          \colorlet{c1}{green!25!white}
          \colorlet{c2}{blue!25!white}
          \colorlet{c3}{red!25!white}
          \colorlet{c4}{yellow!25!white}
          \begin{scope}[every node/.style={coordinate}]
            \cycleNnodes{4}{(0,0)}{0.5in}{135}{x};
          \end{scope}
          \begin{scope} [every node/.style={labeled node}]
            \node [fill=c1] (v1) at (x1) {\(a\)};
            \node [fill=c2] (v2) at (x2) {\(b\)};
            \node [fill=c3] (v3) at (x3) {\(c\)};
          \end{scope}
          \draw (v1) edge (v2) edge (v3);
          \draw (v2) edge (v3);
        \end{tikzpicture}
      }

      \bigskip

      \(G-d\)
    \end{center}
  \end{minipage}
  \caption{Vertex Removal Example}
\end{figure}

Step \ref{step:small} removes vertices with degree less than \(k\).  This is supported by Theorem \ref{thm:lowdeg}.

\begin{theorem}
  \label{thm:lowdeg}
  Let \(G\) be a graph and let \(v\in V(G)\) such that \(\deg(v)<k\) for some \(k\in\N\).  \(G\) is \colorable{k} iff
  \(G-v\) is \colorable{k}.
\end{theorem}

\begin{proof}
  Assume \(G\) is \colorable{k}.  Therefore, by Lemma \ref{lem:remone}, \(G-v\) is also \colorable{k}.

  For the converse, assume that \(G-v\) is \(k\)-colorable.  By assumption, \(\deg(v)<k\), meaning \(v\) has at
  most \(k-1\) neighbors, using at most \(k-1\) colors.  Thus, there is always an additional color available for
  \(v\).  So extend \(G-v\) to \(G\) and color \(v\) with one of the available \(k-\deg(v)\) colors.  The result is
  a proper \coloring{\left((k-1)+1=k\right)} of \(G\).

  \(\therefore G\) is \colorable{k}.
\end{proof}

\begin{corollary}
  \label{cor:lowdeg}
  Let \(G\) be a graph of order \(n\) and let \(X=\setb{v\in V(G)}{\deg(v)<k}\) for some \(k\in\N\):
  \begin{quote}
    \(G\) is \colorable{k} \(\iff G-X\) is \colorable{k}.
  \end{quote}
\end{corollary}

\begin{proof}
  (by induction on \(\abs{X}\))
  \begin{enumerate}
  \item (Base Case) Let \(\abs{X}=0\).

    Since \(G-X=G\), \(G\) is \colorable{k} \(\iff G-X=G\) is \colorable{k}.

  \item (Inductive Assumption) Let \(\abs{X}=r\).

    Assume \(G\) is \colorable{k} \(\iff G-X\) is \colorable{k}.

  \item (Inductive Step) Consider \(\abs{X}=r+1\).
    
    Since \(\abs{X}=r+1>0\), there exists \(v\in X\) such that \(\deg(v)<k\).  Let \(Y=X-\set{v}\) and note that
    \(\abs{Y}=\abs{X}-1=(r+1)-1=r\).  So, \(G\) is \colorable{k} \(\iff G-v\) is \colorable{k} (Lemma \ref{lem:lowdeg}) \(\iff
    (G-v)-Y=G-X\) is \colorable{k} (inductive assumption).
  \end{enumerate}

  Therefore, by the principle of induction, \(G\) is \colorable{k} \(\iff G-X\) is \colorable{k}.
\end{proof}

\subsection{Neighbor Subsets}

\begin{lemma}
  \label{lem:subset}
  Let \(G\) be a graph and let \(u,v\in V(G)\) such that \(N(u)\subseteq N(v)\):
  \begin{quote}
    \(G\) is \colorable{k} \(\iff G-u\) is \colorable{k}
  \end{quote}
\end{lemma}

\begin{proof}
  \begin{description}
  \item[]
  \item[\(\implies\)] Assume \(G\) is \colorable{k}.

    \(\therefore G-u\) is \colorable{k} (Lemma \ref{lem:remone}).

  \item[\(\impliedby\)] Assume \(G-u\) is \colorable{k}.

    Since \(N(u)\subseteq N(v)\) and (by definition) \(u\notin N(u)\), it must be the case that \(u\notin N(v)\) and hence
    \(uv\notin E(G)\).  Thus \(u\) and \(v\) are allowed to have the same color.  Furthermore, since every vertex adjacent to
    \(u\) is also adjacent to \(v\), none of these vertices can have the same color as \(v\).  So extend \(G-u\) to \(G\) and
    color \(u\) with the same color as \(v\).  The result is a proper coloring of \(G\) using the same \(k\) colors.

    \(\therefore G\) is \colorable{k}.
  \end{description}
\end{proof}

\subsection{Edge Density (Tighter Bound)}

\begin{lemma}
  \label{lem:neighbor}
  Let \(G\) be a graph and let \(S\subseteq V(G)\) such that \(S\ne\emptyset\) and \(\forall\,u,v\in S,uv\notin E(G)\):
  \[\left(\exists\,v\in S,\forall\,w\in V(G)-S,vw\in E(G)\right)\implies\forall\,u\in S,N(u)\subseteq N(v)\]
\end{lemma}

\begin{proof}
  Assume \(\exists\,v\in S,\forall\,w\in V(G)-S,vw\in E(G)\).  Now, assume \(u\in S\):
  \begin{description}
  \item[Case 1:] \(N(u)=\emptyset\).
      
    Therefore, by definition, \(N(u)=\emptyset\subseteq N(v)\).

  \item[Case 2:] \(N(u)\ne\emptyset\).

    Assume \(w\in N(u)\).  This means that \(uw\in E(G)\) and hence \(w\notin S\).  So \(w\in V(G)-S\) and thus, by assumption,
    \(vw\in E(G)\). Hence \(w\in N(v)\) and therefore \(N(u)\subseteq N(v)\).
  \end{description}

  \(\therefore\forall\,u\in S,N(u)\subseteq N(v)\)
\end{proof}

\subsection*{Neighborhood Intersection Test}

\begin{lemma}
  \label{lem:inter}
  Let \(G\) be a graph of order \(n\) and size \(m\) such that \(\forall\,u,v\in V(G),N(u)\not\subseteq N(v)\) and let
  \(k\in\N\) such that \(2\le k<n\):
  \begin{quote}
    \(G\) is \colorable{k} \(\implies\exists\,w,z\in V(G)\) such that \(\abs{N(w)\cap N(z)}\le n-2-\frac{n-2}{k-1}\).
  \end{quote}
\end{lemma}

\begin{proof}
  Assume \(G\) is \colorable{k}.  This means that \(V(G)\) can be distributed into \(k\) independent (some possibly empty)
  subsets \(A_1,\ldots,A_k\) such that \(a_i=\abs*{A_i}\) and \(a_1\ge a_2\ge\cdots\ge a_k\).  Since \(n>k\), by the pigeonhole
  principle, it must be the case that \(a_1\ge2\).  Assume \(v\in A_1\).

  First, assume by way of contradiction (ABC) that \(v\) is adjacent to all other vertices in \(V(G)-A_1\).  Since \(a_1\ge2\),
  there exists \(u\in A_1\) such that \(u\ne v\) and \(u\) is not adjacent to \(v\).  Thus, \(N(u)\subseteq N(v)\) (Lemma
  \ref{lem:neighbor}), which contradicts the assumption.  Therefore, \(\exists\,v'\in V(G)-A_1\) such that \(vv'\notin E(G)\).
  Assume \(v'\in A_i\) for some \(i\) such that \(1<i\le k\):

  \begin{description}
  \item [Case 1:] \(a_i=1\)

    By the pigeonhole principle:
    \[a_1\ge\ceil*{\frac{n-1}{k-1}}\ge\frac{n-1}{k-1}\]
    Now, assume by way of contradiction (ABC) that \(v'\) is adjacent to all vertices in \(V(G)-A_1-A_i\) and assume
    \(u\in N(v)\).  Then it must be the case that \(u\in V(G)-A_1-A_i\) and so \(uv'\in E(G)\) and thus \(u\in N(v')\).
    Therefore \(N(v)\subseteq N(v')\), which contradicts the assumption.  This means that there exists some
    \(u\in V(G)-A_1-A_i\) such that \(uv'\notin E(G)\).  This results in the upper bound:
    \[\abs{N(v)\cap N(v')}\le n-2-\frac{n-1}{k-1}\]
    Comparing this to the desired bound:
    \[\left(n-2-\frac{n-2}{k-1}\right)-\left(n-2-\frac{n-1}{k-1}\right)=\frac{1}{k-1}>0\]
    for \(k\ge2\).  Thus the new bound is tighter and so:
    \[\abs{N(v)\cap N(v')}\le n-2-\frac{n-1}{k-1}\le n-2-\frac{n-2}{k-1}\]
    
  \item [Case 2:] \(a_i=2\)

    By the pigeonhole principle:
    \[a_1\ge\ceil*{\frac{n-2}{k-1}}\ge\frac{n-2}{k-1}\]
    This results in the upper bound:
    \[\abs{N(v)\cap N(v')}\le n-2-\frac{n-2}{k-1}\]
    
  \item [Case 3:] \(a_i\ge3\)

    By the pigeonhole principle:
    \[a_1\ge\ceil*{\frac{n-3}{k-1}}\ge\frac{n-3}{k-1}\]
    This results in the upper bound:
    \[\abs{N(v)\cap N(v')}\le n-3-\frac{n-3}{k-1}\]
    Comparing this to the desired bound:
    \[\left(n-2-\frac{n-2}{k-1}\right)-\left(n-3-\frac{n-3}{k-1}\right)=1-\frac{1}{k-1}\ge0\]
    for \(k\ge2\).  Thus the new bound is tighter and so:
    \[\abs{N(v)\cap N(v')}\le n-3-\frac{n-3}{k-1}\le n-2-\frac{n-2}{k-1}\]
  \end{description}

  \(\displaystyle \therefore\exists\,w,z\in V(G)\) such that \(\abs{N(w)\cap N(z)}\le n-2-\frac{n-2}{k-1}\).
\end{proof}

\begin{corollary}
  \label{cor:inter}
  Let \(G\) be a graph of order \(n\) and size \(m\) such that \(\forall\,u,v\in V(G),N(u)\not\subseteq N(v)\) and let
  \(k\in\N\) such that \(2\le k<n\):
  \begin{quote}
    \(\displaystyle\left(\forall\,w,z\in V(G),\abs{N(w)\cap N(z)}>n-2-\frac{n-2}{k-1}\right)\implies G\) is not \colorable{k}.
  \end{quote}
\end{corollary}

\subsection{Recursive Step}

\begin{lemma}
  \label{lem:recurse}
  Let \(G\) be a graph of order \(n>=2\) and let \(u,v\in G\) such that \(uv\notin E(G)\):
  \begin{quote}
    \(G\) is \colorable{k} \(\iff G\cdot uv\) or \(G+uv\) is \colorable{k}.
  \end{quote}
\end{lemma}

\begin{proof}
  \begin{description}
  \item[]
  \item[\(\implies\)] Assume \(G\) is \colorable{k}.

    \begin{description}
    \item [Case a:] \(u\) and \(v\) have the same color.

      Then \(\forall\,w\in N(u)\cup N(v)\) it must be the case that \(w\) is a different color than the color of \(u\) and
      \(v\).  Let \(v'\) be the contracted vertex, so that \(N(v')=N(u)\cup N(v)\), and color \(v'\) with the same color as
      \(u\) and \(v\).  The result is a proper \coloring{k} of \(G\cdot uv\).

      \(\therefore G\cdot uv\) is \colorable{k}.
      
    \item [Case b:] \(u\) and \(v\) have the different colors in \(c\).

      By adding edge \(uv\), \(u\) and \(v\) become adjacent and thus must have different colors.  Therefore, \(u\) and \(v\)
      can retain their same colors.  The result is a proper \coloring{k} of \(G+uv\).
    \end{description}

    \(\therefore G\cdot uv\) or \(G+uv\) is \colorable{k}.
    
  \item[\(\impliedby\)] Assume \(G\cdot uv\) or \(G+uv\) is \colorable{k}.

    \begin{description}
    \item [Case a:] \(G\cdot uv\) is \colorable{k}.

      Let \(v'\) be the contracted vertex with some assigned color.  It must be the case that \(\forall\,w\in N(v'), w\) has a
      different color than \(v'\).  Expand \(G\cdot uv\) to \(G\) and color \(u\) and \(v\) with the same color as \(v'\).
      The result is a proper \coloring{k} of \(G\).
      
    \item [Case b:] \(G+uv\) is \colorable{k}.

      Remove edge \(uv\).  Since \(u\) and \(v\) are no longer adjacent, there are no requirements on their colors.  Thus,
      they can retain their original colors.  The result is a proper \coloring{k} of \(G\).
    \end{description}

    \(\therefore G\) is \colorable{k}.
  \end{description}
\end{proof}

\section{The Proposed Algorithm}

In the exhaustive algorithm described in the previous section, each pair of non-adjacent vertices is subjected to
the two choices of like or different color assignment.  This means that 

\subsection{is-k-colorable(\(G\),\(k\))}

\(G\) may be altered by the called subroutine.

\begin{enumerate}
\item \label{step:check} If \(n\le k\) then return true (Proposition \ref{prop:success}).

\item \label{step:density} If \(\displaystyle m>\frac{n}{2k}(kn-n)\) then return false (Lemma \ref{lem:density}).

\item \label{step:small} Set \(X=\setb{v\in V(G)}{\deg(v)<k}\).  If \(\abs{X}=0\) then go to \ref{step:neighbor}.  Otherwise,
  replace \(G\) with \(G-X\) and go to \ref{step:check} (Corollary \ref{cor:lowdeg}).

\item \label{step:neighbor} If \(G\) has vertices \(u\) and \(v\) such that \(N(u)\subseteq N(v)\) then replace \(G\) with
  \(G-u\) and go to \ref{step:check} (Lemma \ref{lem:subset}).

\item \label{step:common} Select two non-adjacent vertices \(u,v\in V(G)\) with the smallest number of common neighbors.  If
  \(k\ge2\) and \(\displaystyle\abs{N(u)\cap N(v)}>n-2-\frac{n-2}{k-1}\) then return false (Corollary \ref{cor:inter}).

\item Return is-k-colorable(\(G\cdot uv\), \(k\)) or is-k-colorable(\(G+uv\), \(k\)) (Lemma \ref{lem:recurse}).

\end{enumerate}

This subroutine is guaranteed to return because either there will be sufficient contractions such that \(n\le k\) or
sufficient edge additions such that the graph becomes complete and \(m>\frac{n}{2k}(kn-n)\) for any \(k<n\).

\subsection{find-k-colorable(\(G\))}

\begin{enumerate}
\item \label{step:null} If \(n=0\) then return \(0\).

\item \label{step:init} Set \(k=1\).  

\item \label{step:call} If is-k-colorable(\(G\), \(k\)) then return \(k\).

\item \label{step:incr} \(k=k+1\)

\item \label{step:loop} Go to \ref{step:call}.
\end{enumerate}

This loop is guaranteed to complete because eventually \(k\) wiill exceed \(n\), which will cause the called subroutine to
return true.

\section{Theorems}

\subsection{Algorithm Success Condition}

\begin{proposition}
  \label{prop:success}
  Let \(G\) be a graph of order \(n\) and let \(k\in\N\cup\set{0}\):
  \begin{quote}
    \(n\le k\implies G\) is \colorable{k}.
  \end{quote}
\end{proposition}

\subsection{Edge Density Test}

\begin{lemma}
  \label{lem:density}
  Let \(G\) be a graph of order \(n\) and size \(m\):
  \begin{quote}
    \(G\) is \colorable{k} \(\displaystyle\implies m\le\frac{n}{2k}(nk-n)\).
  \end{quote}
\end{lemma}

\begin{proof}
  Assume \(G\) is \colorable{k}.  This means that \(V(G)\) can be distributed into \(k\) independent (some possibly empty)
  subsets.  Call these subsets \(A_1,\ldots A_k\) and let \(a_i=\abs*{A_i}\).  Thus, each \(v\in A_i\) can be adjacent to at
  most \(n-a_i\) other vertices in \(G\), and hence the maximum number of edges incident to vertices in \(A_i\) is given by:
  \(a_i(n-a_i)=na_i-a_i^2\).  Now, using the fundamental theorem of graph theory, the maximum number of edges in \(G\) is given
  by:
  \[m\le\frac{1}{2}\sum_{i=1}^k(na_i-a_i^2)\]
  with the constraint:
  \[\sum_{i=1}^ka_i=n\]
  This problem can be solved using the Lagrange multiplier technique:
  \begin{gather*}
    \frac{1}{2}(n-2a_i)=\l \\
    a_i=\frac{n}{2}-\l \\
    \\
    \sum_{i=1}^ka_i=\sum_{i=1}^k\left(\frac{n}{2}-\l\right)=k\left(\frac{n}{2}-\l\right)=n \\
    \l=\frac{n}{2}-\frac{n}{k} \\
    \\
    a_i=\frac{n}{2}-\left(\frac{n}{2}-\frac{n}{k}\right)=\frac{n}{k}
  \end{gather*}
  Therefore:
  \[m\le\frac{1}{2}\sum_{i=1}^k\left[n\left(\frac{n}{k}\right)-\left(\frac{n}{k}\right)^2\right]=
  \frac{k}{2}\left(\frac{n^2k-n^2}{k^2}\right)=\frac{n}{2k}(nk-n)\]
\end{proof}

\begin{corollary}
  \label{cor:density}
  Let \(G\) be a graph of order \(n\) and size \(m\) and let \(k\in\N\):
  \begin{quote}
    \(\displaystyle m>\frac{n}{2k}(nk-n)\implies G\) is not \colorable{k}.
  \end{quote}
\end{corollary}

\subsection{Degree \(n-1\) Vertices}

\begin{lemma}
  \label{lem:adjall}
  Let \(G\) be a graph of order \(n\) and let \(v\in V(G)\) such that \(\deg(v)=n-1\):
  \begin{quote}
    \(G\) is \colorable{k} \(\iff G-v\) is \colorable{(k-1)}.
  \end{quote}
\end{lemma}

\begin{proof}
  \begin{description}
  \item[]
  \item[\(\implies\)] Assume \(G\) is \colorable{k}.

    Since \(\deg(v)=n-1\), \(v\) is adjacent to all vertices in \(G-v\) and so \(v\) must have its own unique color in any
    proper \coloring{k} of \(G\).  Thus, \(G-v\) has a proper coloring using the remaining \(k-1\) colors.

    \(\therefore G-v\) is \colorable{(k-1)}.
    
  \item[\(\impliedby\)] Assume \(G-v\) is \colorable{(k-1)}.

    Extend \(G-v\) to \(G\) and color \(v\) with a unique color (i.e., not one of the original \(k-1\) colors).  This is a
    proper coloring of \(G\) using \((k-1)+1=k\) colors.

    \(\therefore G\) is \colorable{k}.
  \end{description}
\end{proof}

\begin{corollary}
  \label{cor:adjall}
  Let \(G\) be a graph of order \(n\) and let \(X=\setb{v\in V(G)}{\deg(v)=n-1}\):
  \begin{quote}
    \(G\) is \colorable{k} \(\iff G-X\) is \colorable{(k-\abs{X})}.
  \end{quote}
\end{corollary}

\begin{proof}
  (by induction on \(\abs{X}\))
  \begin{enumerate}
  \item (Base Case) Let \(\abs{X}=0\).

    Since \(G-X=G\) and \(k-0=k\), \(G\) is \colorable{k} \(\iff G-X=G\) is \colorable{(k-0=k)}.

  \item (Inductive Assumption) Let \(\abs{X}=r\).

    Assume \(G\) is \colorable{k} \(\iff G-X\) is \colorable{(k-r)}.

  \item (Inductive Step) Consider \(\abs{X}=r+1\).
    
    Since \(\abs{X}=r+1>0\), there exists \(v\in X\) such that \(\deg(v)=n-1\).  Let \(Y=X-\set{v}\) and note that
    \(\abs{Y}=\abs{X}-1=(r+1)-1=r\).  So, \(G\) is \colorable{k} \(\iff G-v\) is \colorable{(k-1)} (Lemma \ref{lem:adjall})
    \(\iff (G-v)-Y=G-X\) is \colorable{\left((k-1)-r=k-(r+1)\right)} (inductive assumption).
  \end{enumerate}

  Therefore, by the principle of induction, \(G\) is \colorable{k} \(\iff G-X\) is \colorable{(k-\abs{X})}.
\end{proof}

\subsection{Low Degree Vertices}

\begin{lemma}
  \label{lem:remone}
  Let \(G\) be a graph and let \(v\in V(G)\):
  \begin{quote}
    \(G\) is \colorable{k} \(\implies G-v\) is \colorable{k}.
  \end{quote}
\end{lemma}

\begin{proof}
  
  Assume \(G\) is \colorable{k}.

  \begin{description}
  \item[Case 1:] \(v\) has its own unique color.
      
    \(G-v\) still has a proper coloring using \(k-1\) colors and hence is \colorable{(k-1)}, and thus is \colorable{k}.

  \item[Case 2:] \(v\) shares a color with some other vertex.

    \(G-v\) still has a proper coloring using \(k\) colors and is thus still \colorable{k}.
  \end{description}

  \(\therefore G-v\) is \colorable{k}.
\end{proof}

\begin{lemma}
  \label{lem:lowdeg}
  Let \(G\) be a graph and let \(v\in V(G)\) such that \(d(v)<k\) for some \(k\in\N\):
  \begin{quote}
    \(G\) is \colorable{k} \(\iff G-v\) is \colorable{k}.
  \end{quote}
\end{lemma}

\begin{proof}
  \begin{description}
  \item[]
  \item[\(\implies\)] Assume \(G\) is \colorable{k}.

    \(\therefore G-v\) is \colorable{k} (Lemma \ref{lem:remone}).

  \item[\(\impliedby\)] Assume \(G-v\) is \(k\)-colorable.
    By assumption, \(d(v)<k\), meaning \(v\) has at most \(k-1\) neighbors, using at most \(k-1\) colors.  Thus, there is
    always an additional color available for \(v\).  So extend \(G-v\) to \(G\) and color \(v\) with one of the available
    \(k-d(v)\) colors.  The result is a proper \coloring{\left((k-1)+1=k\right)} of \(G\).

    \(\therefore G\) is \colorable{k}.
  \end{description}
\end{proof}

\begin{corollary}
  \label{cor:lowdeg}
  Let \(G\) be a graph of order \(n\) and let \(X=\setb{v\in V(G)}{\deg(v)<k}\) for some \(k\in\N\):
  \begin{quote}
    \(G\) is \colorable{k} \(\iff G-X\) is \colorable{k}.
  \end{quote}
\end{corollary}

\begin{proof}
  (by induction on \(\abs{X}\))
  \begin{enumerate}
  \item (Base Case) Let \(\abs{X}=0\).

    Since \(G-X=G\), \(G\) is \colorable{k} \(\iff G-X=G\) is \colorable{k}.

  \item (Inductive Assumption) Let \(\abs{X}=r\).

    Assume \(G\) is \colorable{k} \(\iff G-X\) is \colorable{k}.

  \item (Inductive Step) Consider \(\abs{X}=r+1\).
    
    Since \(\abs{X}=r+1>0\), there exists \(v\in X\) such that \(\deg(v)<k\).  Let \(Y=X-\set{v}\) and note that
    \(\abs{Y}=\abs{X}-1=(r+1)-1=r\).  So, \(G\) is \colorable{k} \(\iff G-v\) is \colorable{k} (Lemma \ref{lem:lowdeg}) \(\iff
    (G-v)-Y=G-X\) is \colorable{k} (inductive assumption).
  \end{enumerate}

  Therefore, by the principle of induction, \(G\) is \colorable{k} \(\iff G-X\) is \colorable{k}.
\end{proof}

\subsection{Neighbor Subsets}

\begin{lemma}
  \label{lem:subset}
  Let \(G\) be a graph and let \(u,v\in V(G)\) such that \(N(u)\subseteq N(v)\):
  \begin{quote}
    \(G\) is \colorable{k} \(\iff G-u\) is \colorable{k}
  \end{quote}
\end{lemma}

\begin{proof}
  \begin{description}
  \item[]
  \item[\(\implies\)] Assume \(G\) is \colorable{k}.

    \(\therefore G-u\) is \colorable{k} (Lemma \ref{lem:remone}).

  \item[\(\impliedby\)] Assume \(G-u\) is \colorable{k}.

    Since \(N(u)\subseteq N(v)\) and (by definition) \(u\notin N(u)\), it must be the case that \(u\notin N(v)\) and hence
    \(uv\notin E(G)\).  Thus \(u\) and \(v\) are allowed to have the same color.  Furthermore, since every vertex adjacent to
    \(u\) is also adjacent to \(v\), none of these vertices can have the same color as \(v\).  So extend \(G-u\) to \(G\) and
    color \(u\) with the same color as \(v\).  The result is a proper coloring of \(G\) using the same \(k\) colors.

    \(\therefore G\) is \colorable{k}.
  \end{description}
\end{proof}

\subsection{Edge Density (Tighter Bound)}

\begin{lemma}
  \label{lem:neighbor}
  Let \(G\) be a graph and let \(S\subseteq V(G)\) such that \(S\ne\emptyset\) and \(\forall\,u,v\in S,uv\notin E(G)\):
  \[\left(\exists\,v\in S,\forall\,w\in V(G)-S,vw\in E(G)\right)\implies\forall\,u\in S,N(u)\subseteq N(v)\]
\end{lemma}

\begin{proof}
  Assume \(\exists\,v\in S,\forall\,w\in V(G)-S,vw\in E(G)\).  Now, assume \(u\in S\):
  \begin{description}
  \item[Case 1:] \(N(u)=\emptyset\).
      
    Therefore, by definition, \(N(u)=\emptyset\subseteq N(v)\).

  \item[Case 2:] \(N(u)\ne\emptyset\).

    Assume \(w\in N(u)\).  This means that \(uw\in E(G)\) and hence \(w\notin S\).  So \(w\in V(G)-S\) and thus, by assumption,
    \(vw\in E(G)\). Hence \(w\in N(v)\) and therefore \(N(u)\subseteq N(v)\).
  \end{description}

  \(\therefore\forall\,u\in S,N(u)\subseteq N(v)\)
\end{proof}

\begin{lemma}
  \label{lem:tighter}
  Let \(G\) be a graph of order \(n\) and size \(m\) such that:
  \begin{enumerate}
  \item \(\forall\,v\in V(G),\deg(v)<n-1\)
  \item \(\forall\,u,v\in V(G),N(u)\not\subseteq N(v)\)
  \end{enumerate}
  and let \(k\in\N\):
  \begin{quote}
    \(G\) is \colorable{k} \(\displaystyle\implies m\le\frac{n}{2k}(nk-n-k)\).
  \end{quote}
\end{lemma}

\begin{proof}
  Assume \(G\) is \colorable{k}.  This means that \(V(G)\) can be distributed into \(k\) independent (some possibly empty)
  subsets.  Call these subsets \(A_1,\ldots A_k\) and let \(a_i=\abs*{A_i}\).

  First, assume by way of contradiction (ABC) that for some \(A_i\) where \(1\le i\le k\) there exists \(v\in A_i\) such that
  \(v\) is adjacent to every vertex in \(V(G)-A_i\).

  \begin{description}
  \item[Case 1:] \(a_i=1\)

    This means that \(v\) is adjacent to all of the other vertices in \(G\) and thus has degree \(n-1\), which contradicts
    the initial assumption.

  \item[Case 2:] \(a_i>1\)

    Then there exists \(u\in A_i\) such that \(u\ne v\) and \(N(u)\subseteq N(v)\) (Lemma \ref{lem:neighbor}).
  \end{description}

  \(\therefore N(u)\subseteq N(v)\), which contradicts the initial assumption, and so for each \(v\in A_i\) there must exist
  \(u\in V(G)-A_i\) such that \(uv\notin E(G)\), and hence each \(v\in A_i\) can be adjacent to at most \(n-a_i-1\) other
  vertices in \(G\).  This means that the maximum number of edges incident to vertices in \(A_i\) is given by:
  \(a_i(n-a_i-1)=na_i-a_i^2-a_i\).  Now, using the fundamental theorem of graph theory, the maximum number of edges in \(G\) is
  given by:
  \[m\le\frac{1}{2}\sum_{i=1}^k(na_i-a_i^2-a_i)\]
  with the constraint:
  \[\sum_{i=1}^ka_i=n\]
  This problem can be solved using the Lagrange multiplier technique:
  \begin{gather*}
    \frac{1}{2}(n-2a_i-1)=\l \\
    a_i=\frac{n}{2}-\frac{1}{2}-\l \\
    \\
    \sum_{i=1}^ka_i=\sum_{i=1}^k\left(\frac{n}{2}-\frac{1}{2}-\l\right)=k\left(\frac{n}{2}-\frac{1}{2}-\l\right)=n \\
    \l=\frac{n}{2}-\frac{1}{2}-\frac{n}{k} \\
    \\
    a_i=\frac{n}{2}-\frac{1}{2}-\left(\frac{n}{2}-\frac{1}{2}-\frac{n}{k}\right)=\frac{n}{k}
  \end{gather*}
  Therefore:
  \[m\le\frac{1}{2}\sum_{i=1}^k\left[n\left(\frac{n}{k}\right)-\left(\frac{n}{k}\right)^2-\frac{n}{k}\right]=
  \frac{k}{2}\left(\frac{n^2k-n^2-kn}{k^2}\right)=\frac{n}{2k}(nk-n-k)\]
\end{proof}

\begin{corollary}
  \label{cor:tighter}
  Let \(G\) be a graph of order \(n\) and size \(m\) such that:
  \begin{enumerate}
  \item \(\forall\,v\in V(G),\deg(v)<n-1\)
  \item \(\forall\,u,v\in V(G),N(u)\not\subseteq N(v)\)
  \end{enumerate}
  and let \(k\in\N\):
  \begin{quote}
    \(\displaystyle m>\frac{n}{2k}(nk-n-k)\implies G\) is not \colorable{k}.
  \end{quote}
\end{corollary}

\subsection*{Neighborhood Intersection Test}

\begin{lemma}
  \label{lem:inter}
  Let \(G\) be a graph of order \(n\) and size \(m\) such that \(\forall\,u,v\in V(G),N(u)\not\subseteq N(v)\) and let
  \(k\in\N\) such that \(2\le k<n\):
  \begin{quote}
    \(G\) is \colorable{k} \(\implies\exists\,w,z\in V(G)\) such that \(\abs{N(w)\cap N(z)}\le n-2-\frac{n-2}{k-1}\).
  \end{quote}
\end{lemma}

\begin{proof}
  Assume \(G\) is \colorable{k}.  This means that \(V(G)\) can be distributed into \(k\) independent (some possibly empty)
  subsets \(A_1,\ldots,A_k\) such that \(a_i=\abs*{A_i}\) and \(a_1\ge a_2\ge\cdots\ge a_k\).  Since \(n>k\), by the pigeonhole
  principle, it must be the case that \(a_1\ge2\).  Assume \(v\in A_1\).

  First, assume by way of contradiction (ABC) that \(v\) is adjacent to all other vertices in \(V(G)-A_1\).  Since \(a_1\ge2\),
  there exists \(u\in A_1\) such that \(u\ne v\) and \(u\) is not adjacent to \(v\).  Thus, \(N(u)\subseteq N(v)\) (Lemma
  \ref{lem:neighbor}), which contradicts the assumption.  Therefore, \(\exists\,v'\in V(G)-A_1\) such that \(vv'\notin E(G)\).
  Assume \(v'\in A_i\) for some \(i\) such that \(1<i\le k\):

  \begin{description}
  \item [Case 1:] \(a_i=1\)

    By the pigeonhole principle:
    \[a_1\ge\ceil*{\frac{n-1}{k-1}}\ge\frac{n-1}{k-1}\]
    Now, assume by way of contradiction (ABC) that \(v'\) is adjacent to all vertices in \(V(G)-A_1-A_i\) and assume
    \(u\in N(v)\).  Then it must be the case that \(u\in V(G)-A_1-A_i\) and so \(uv'\in E(G)\) and thus \(u\in N(v')\).
    Therefore \(N(v)\subseteq N(v')\), which contradicts the assumption.  This means that there exists some
    \(u\in V(G)-A_1-A_i\) such that \(uv'\notin E(G)\).  This results in the upper bound:
    \[\abs{N(v)\cap N(v')}\le n-2-\frac{n-1}{k-1}\]
    Comparing this to the desired bound:
    \[\left(n-2-\frac{n-2}{k-1}\right)-\left(n-2-\frac{n-1}{k-1}\right)=\frac{1}{k-1}>0\]
    for \(k\ge2\).  Thus the new bound is tighter and so:
    \[\abs{N(v)\cap N(v')}\le n-2-\frac{n-1}{k-1}\le n-2-\frac{n-2}{k-1}\]
    
  \item [Case 2:] \(a_i=2\)

    By the pigeonhole principle:
    \[a_1\ge\ceil*{\frac{n-2}{k-1}}\ge\frac{n-2}{k-1}\]
    This results in the upper bound:
    \[\abs{N(v)\cap N(v')}\le n-2-\frac{n-2}{k-1}\]
    
  \item [Case 3:] \(a_i\ge3\)

    By the pigeonhole principle:
    \[a_1\ge\ceil*{\frac{n-3}{k-1}}\ge\frac{n-3}{k-1}\]
    This results in the upper bound:
    \[\abs{N(v)\cap N(v')}\le n-3-\frac{n-3}{k-1}\]
    Comparing this to the desired bound:
    \[\left(n-2-\frac{n-2}{k-1}\right)-\left(n-3-\frac{n-3}{k-1}\right)=1-\frac{1}{k-1}\ge0\]
    for \(k\ge2\).  Thus the new bound is tighter and so:
    \[\abs{N(v)\cap N(v')}\le n-3-\frac{n-3}{k-1}\le n-2-\frac{n-2}{k-1}\]
  \end{description}

  \(\displaystyle \therefore\exists\,w,z\in V(G)\) such that \(\abs{N(w)\cap N(z)}\le n-2-\frac{n-2}{k-1}\).
\end{proof}

\begin{corollary}
  \label{cor:inter}
  Let \(G\) be a graph of order \(n\) and size \(m\) such that \(\forall\,u,v\in V(G),N(u)\not\subseteq N(v)\) and let
  \(k\in\N\) such that \(2\le k<n\):
  \begin{quote}
    \(\displaystyle\left(\forall\,w,z\in V(G),\abs{N(w)\cap N(z)}>n-2-\frac{n-2}{k-1}\right)\implies G\) is not \colorable{k}.
  \end{quote}
\end{corollary}

\subsection{Recursive Step}

\begin{lemma}
  \label{lem:recurse}
  Let \(G\) be a graph of order \(n>=2\) and let \(u,v\in G\) such that \(uv\notin E(G)\):
  \begin{quote}
    \(G\) is \colorable{k} \(\iff G\cdot uv\) or \(G+uv\) is \colorable{k}.
  \end{quote}
\end{lemma}

\begin{proof}
  \begin{description}
  \item[]
  \item[\(\implies\)] Assume \(G\) is \colorable{k}.

    \begin{description}
    \item [Case a:] \(u\) and \(v\) have the same color.

      Then \(\forall\,w\in N(u)\cup N(v)\) it must be the case that \(w\) is a different color than the color of \(u\) and
      \(v\).  Let \(v'\) be the contracted vertex, so that \(N(v')=N(u)\cup N(v)\), and color \(v'\) with the same color as
      \(u\) and \(v\).  The result is a proper \coloring{k} of \(G\cdot uv\).

      \(\therefore G\cdot uv\) is \colorable{k}.
      
    \item [Case b:] \(u\) and \(v\) have the different colors in \(c\).

      By adding edge \(uv\), \(u\) and \(v\) become adjacent and thus must have different colors.  Therefore, \(u\) and \(v\)
      can retain their same colors.  The result is a proper \coloring{k} of \(G+uv\).
    \end{description}

    \(\therefore G\cdot uv\) or \(G+uv\) is \colorable{k}.
    
  \item[\(\impliedby\)] Assume \(G\cdot uv\) or \(G+uv\) is \colorable{k}.

    \begin{description}
    \item [Case a:] \(G\cdot uv\) is \colorable{k}.

      Let \(v'\) be the contracted vertex with some assigned color.  It must be the case that \(\forall\,w\in N(v'), w\) has a
      different color than \(v'\).  Expand \(G\cdot uv\) to \(G\) and color \(u\) and \(v\) with the same color as \(v'\).
      The result is a proper \coloring{k} of \(G\).
      
    \item [Case b:] \(G+uv\) is \colorable{k}.

      Remove edge \(uv\).  Since \(u\) and \(v\) are no longer adjacent, there are no requirements on their colors.  Thus,
      they can retain their original colors.  The result is a proper \coloring{k} of \(G\).
    \end{description}

    \(\therefore G\) is \colorable{k}.
  \end{description}
\end{proof}
